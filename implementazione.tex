\chapter{Implementazione di Dropper}
\label{cap:implementazione}

In questo capitolo si farà un'analisi dell'attuale stato di sviluppo
di Dropper.

\section{Modulo Core}

\subsection{Interazione con l'utente}

Attualmente Dropper si interfaccia all'utente tramite alcune funzioni
della classe \lstinline{dropper}. Il costruttore di questa classe
riceve come argomento il percorso sul file system dove risiede il file
oggetto eseguibile da analizzare. Le operazioni di analisi vengono
avviate tramite la funzione \lstinline{analyze_all}. Una volta che
l'analisi è stata effettuata si può richiamare la funzione
\lstinline{build_spawn_shell_payload} che si occupera di generare la
catena vera e propria. Anche se queste sono le uniche chiamate
essenziali per utilizzare Dropper l'utente ha a disposizione ulteriori
funzioni:
\begin{itemize}

\item \lstinline{set_can_control_fd} setta il file descriptor
  controllato dall'utente (-1 se nessun file descriptor può essere
  controllato)

\item \lstinline{add_shared_object} permette di analizzare un oggetto
  che rappresenti una libreria dinamica. Dropper ricava da questo file
  gli offset relativi tra le funzioni della libreria

\item \lstinline{set_function_for_address_resolving} indica la
  funzione di riferimento da utilizzare nell'applicazione di tecniche
  che ricavino l'indirizzo di funzioni casualizzate in memoria
  partendo da valori nella GOT table (vedi sez. \ref{sec:expl}. La
  funzione indicata dev'essere già stata richiamata almeno una volta
  al momento dell'esecuzione della catena

\item \lstinline{set_writeable_area} setta l'indirizzo dell'area di
  memoria dove effettuare operazioni di scrittura. Di default questo
  valore è settato all'indirizzo di caricamento in memoria della
  sezione \lstinline{.data}

\item \lstinline{set_cmd} setta il comando da eseguire sulla macchina
  bersaglio una volta che l'exploit ha avuto successo

\end{itemize}

Quando il file eseguibile è stato analizzato e le informazioni
necessarie impostate il metodo \lstinline{build_spawn_shell_payload}
cerca di generare una catena che avvii il comando impostato.

\subsection{Analisi degli import}

Se una funzione esterna viene utilizzata dal programma, anche se il
suo indirizzo reale viene casualizzato, sarà presente un elemento
relativo a quella funzione all'interno della Procedure Linkage
Table. Questo elemento, ammesso che il binario non sia stato compilato
come indipendente dalla posizione in memoria (PIE), ha una posizione
nota. Possiamo quindi ritornare al suo indirizzo per emulare la
chiamata a quella funzione (return-to-plt). Nella GOT invece ogni
elemento inizialmente, e quindi anche staticamente nel file, contiene
l'indirizzo dell'istruzione successiva nell'elemento della PLT
relativo alla stessa funzione. Per ogni funzione la rilocazione
associata, nella sezione \lstinline{.rel.plt}, contiene nel campo
\lstinline{r_offset} l'indirizzo dell'elemento nella GOT table. A
partire da quest'ultimo, per le considerazioni appena fatte, può
essere ricavato l'indirizzo dell'elemento nella PLT. Gli indirizzi del
relativo elemento della GOT e di quello nella PLT vengono ricavati per
ogni funzione utilizzata dal file oggetto eseguibile nel metodo
\lstinline{get_imports}.

\subsection{Costruzione della catena}

Attualmente il modulo Core per prima cosa cerca di scrivere in un'area
di memoria scrivibile i dati ai quali punteranno gli argomenti
dell'invocazione della funzione \lstinline{execve}. Nel caso
quest'area di memoria non viene specificata dall'utente viene
utilizzato l'indirizzo di caricamento in memoria della sezione
\lstinline{.data}. Gli argomenti e i dati veri e propri vengono creati
dal modulo partendo dal comando che si è deciso di eseguire sulla
macchina (nella funzione \lstinline{payload_execve_args}). Se si può
controllare un file descriptor Core utilizza la funzione
\lstinline{read} per scrivere in memoria gli argomenti di
\lstinline{execve}, come descritto in \ref{sec:arc:core}. Per fare
questo viene utilizzato il metodo \lstinline{get_ret_func_chunk} del
modulo gadgets tool il quale ci consente di creare una catena che
richiami una funzione di cui si conosce l'indirizzo. È compito poi
dell'utente fare in modo che al momento della chiamata di questa
funzione il file descriptor indicato si trovi nello stato giusto in
modo da restituire i valori calcolati dal modulo. I valori da inserire
nel flusso del file descriptor vengono ricavati dall'utente attraverso
il parametro \lstinline{fd_payload} di dropper.
%%TODO
Esempio?

Attualmente è implementata una seconda strategia nel caso non sia
possibile richiamare la funzione \lstinline{read} e/o non si controlli
nessun file descriptor. Essa consiste nell'utilizzare una serie di
gadget che possano spostare valori dallo stack in una locazione di
memoria arbitraria.  Per costruire una catena del genere si chiama la
funzione \lstinline{build_memory_write} del modulo Gadgets che prende
come argomenti l'indirizzo della memoria dove scrivere i
dati. Solitamente, come vedremo in \ref{sec:memorystore}, le catene
generate in questo modo tendono ad essere piuttosto lunghe.

%%TODO inserire strategie non implementate?

Per quanto riguarda la seconda fase, la chiamata della funzione, sono
state implementate due strategie. La prima nel caso \lstinline{execve}
sia presente all'interno delle funzioni utilizzate dal programma. In
questo caso la generazione della parte restante della catena è banale
e, come visto nel paragrafo precedente, viene costruita dalla funzione
\lstinline{get_ret_func_chunk} del modulo Gadgets. Più
interessante è il caso in cui il file oggetto eseguibile non utilizzi
questa funzione. 

Infatti in questo Dropper cerca di far costruire al modulo Dropper una
catena in grado di modificare il valore della Global Offset Table in
loco, in modo da effettuare una tecnica di tipo \emph{got patching}
(vedi \ref{sec:expl}). Il modulo Gadgets espone il metodo
\lstinline{build_memory_write} che si occupa dei dettagli
implementativi di questo tipo di catena. Viene chiamato dal modulo
Core con i seguenti parametri:

\begin{itemize}
  \item l'indirizzo di memoria dove eseguire la somma: l'indirizzo
    dell'elemento della GOT da ``patchare''. Quest'informazione viene
    ricavata durante la fase dell'analisi degli import
  \item il valore da aggiungere: l'offset relativo tra la funzione
    utilizzata come base e quella voluta
  \item il valore di base: l'offset che la funzione di base ha
    all'interno del file oggetto condiviso della libreria. Come
    vedremo questo rende possibile utilizzare la tecnica anche quando
    le tipologie di gadget disponibili non sono ideali al prezzo di
    sacrificare un'affidabilità del 100\% (vedi
    sez. \ref{sec:arithmeticstore})

\end{itemize}

La classe inoltre contiene alcuni metodi per il salvataggio dei dati
relativi ai gadgets su disco, questo perché l'analisi di eseguibili
molto grandi può durare diversi minuti. Al loro interno questi metodi
utilizzano la libreria \lstinline{pickle} per la serializzazione degli
oggetti. 

\section{Modulo Gadgets}

Questo modulo, come descritto nel capitolo dedicato all'architettura,
estrapola e cataloga i gadget presenti nel binario in esame ed espone
i metodi di creazione delle catene per l'esecuzione di diverse
funzioni. Le catene vengono costruite arrangiando i gadget disponibili
e collaborando con il modulo Payload che si occupa della generazione
del payload vero e proprio. Attualmente il modulo mette a disposizione
i seguenti metodi per la generazione delle catene:
\begin{itemize}
  \item \lstinline{get_stack_slide_chunk}: consente di aggiungere un
    valore arbitrario al registro dello stack, questo tipo di catene
    ci consente ad esempio di ``saltare'' gli argomenti di una
    funzione sullo stack e passare al gadget successivo
  \item \lstinline{get_ret_func_chunk}: restituisce una catena che
    richiami una funzione. Riceve come parametri di ingresso gli
    argomenti da passare alla funzione e l'indirizzo della funzione
    stesssa 
  \item \lstinline{get_memory_add_chunk}: restituisce una catena che
    esegue un operazione di addizione in una locazione di memoria
    arbitraria
  \item \lstinline{get_memory_store_chunk}: restituisce una catena che
    scriva in memoria dei valori arbitrari
  \item \lstinline{get_ccf_chunk}: restituisce un chunk che azzeri il
    carry flag
\end{itemize}

Le tipologie di gadget utilizzate per la costruzione di queste catene
sono descritte nella parte restante di questo capitolo. 

Fondamentale per poter utilizzare un gadget è poter controllare i
registri coinvolti nelle sue operazioni, per questo motivo la prima
analisi che viene condotta una volta ottenuta la lista dei gadget è la
selezione di quelli che ci consentono di controllare i registri. La
ricerca dei gadget viene effettuata in tutte le sezioni che vengono
caricate in memoria e marcate come eseguibili. Si è deciso di separare
il modulo in diverse sottoclassi, ognuna delle quali contiene la
logica per la selezione e l'utilizzo di un particolare tipo di gadget.

\subsection{Selezione dei gadget per l'impostazione dei registri}

La logica e le strutture dati utilizzate per la selezione e la
gestione dei gadget utilizzati per l'impostazione dei registri è
contenuta nella classe \lstinline{RegSet}. Il metodo
\lstinline{_analyze_gadget} riceve come argomento in ingresso un
gadget e, nel caso esso possa essere utilizzato per l'impostazione di
registri, viene aggiunto nelle strutture dati interne alla classe. La
classe espone il metodo \lstinline{get_chunk} che ritorna una catena
che imposta i registri in modo da riflettere le coppie registro/valore
del dizionario preso come argomento. Attualmente solo gadget che
contengono istruzioni di tipo \lstinline{pop reg} vengono considerati
come potenziali candidati per l'impostazione di registri.

L'analisi avviene tramite una simulazione istruzione per
istruzione. Se l'istruzione attuale è di tipo \lstinline{pop reg}
allora si memorizza il nome del registro e l'offset tra il valore
attuale dello stack e quello all'inizio della simulazione. Questo ci
consente di conoscere dove posizionare il valore da impostare nel
registro al momento della generazione del payload. Durante la
simulazione si controlla anche che le istruzioni che non sono del tipo
\lstinline{pop reg} non eseguano operazioni di lettura o scrittura
sulla memoria. Questo ci consente di scartare già in questa fase i
gadget che presentano effetti secondari. Dopo la fase di analisi
quindi avremo, per ogni gadget, la lista dei registri che possono
essere impostati insieme all'ordine e alla posizione con cui vengono
prelevati dallo stack. 


La funzione \lstinline{get_chunk} invece restituisce una catena che
consente di impostare l'insieme dei registri che riceve in
ingresso. Dato che un gadget potrebbe impostare più di un registro, la
scelta dei gadget che andranno a comporre questa catena è riducibile
ad un problema di \emph{set covering}. È preferibile che il payload
generato dalla catena sia il più piccolo possibile. La lunghezza della
catena è influenzata sia dal numero di gadget utilizzati, sia dal
padding che è necessario inserire nell'utilizzo di ogni gadget per via
delle operazioni che esso effettua sul registro dello
stack. Attualmente per trovare la catena viene utilizzato un algoritmo
greedy che scorre i registri da ``coprire'' e, se il registro non è
già coperto da un gadget inserito nella catena, sceglie il gadget che
lo copre e che necessita del minor padding possibile. Se l'aggiunta di
un gadget ne rende superfluo uno già presente nei candidati
quest'ultimo viene rimosso. La soluzione trovata con quest'algoritmo
non è ottima. Nella sezione \ref{sec:pddl} è descritto un primo
approccio per l'utilizzo di planner nella ricerca di una sequenza
ottima.

\subsection{Selezione dei gadget per la scrittura in memoria}
\label{sec:memorystore}

La logica e le strutture dati per gestire i gadget che ci consentono
di scrivere in memoria si trovano nella classe
\lstinline{MemoryStore}. 

Durante la prima classificazione la classe
\lstinline{GadgetClassifier} del framework BARF verifica se a seguito
di alcune simulazioni un valore presente in un registro (denominato
sorgente) viene scritto in una locazione di memoria il cui indirizzo
corrisponde al valore di un registro (denominato destinazione) sommato
ad un eventuale offset.

Ogni volta che questa condizione viene verificata il gadget viene
aggiunto alla categoria StoreMemory. È da notare che la condizione di
cui sopra potrebbe verificarsi per più coppie di registri nello stesso
gadget, per ognuna di queste viene creato un oggetto gadget di tipo
StoreMemory e aggiunto alla categoria. La classe parte da questi
gadget per procedere alla fase di verifica e validazione.

Innanzitutto se il registro di destinazione è il registro dello stack
il gadget viene scartato. Questo fa in modo che non vengano presi in
considerazione i gadget con istruzioni tipo \lstinline{push
  reg}. Questi gadget infatti, solitamente non consentono di scrivere
in una regione arbitraria della memoria, oltre a complicare la
gestione dello schema dello stack. Dopo di che vengono scartati anche
i gadget che hanno come registro di destinazione rip e eip. Infatti
essendo questo registro fisso rispetto all'istruzione che lo esegue
neanche questi gadget ci consentono di scrivere in una posizione
arbitraria di memoria.

Dopo di che si procede seguendo le linee guida descritte in
\ref{s:verifica_validazione}: il gadget viene scartato se non si
possono controllare i registri sorgente e destinazione, si verifica e
valida la generalità e ci si assicura che il gadget non presenti
effetti secondari. Nella verifica degli effetti secondari si controlla
che non ci siano scritture al di fuori della locazione scelta e dello
stack.

Il metodo \lstinline{get_chunk} invece prende come argomenti un
indirizzo in memoria e una stringa e restituisce una catena che scrive
la stringa in memoria partendo da quell'indirizzo. Il metodo quindi
espone un livello d'astrazione più alto al modulo che la utilizza e,
normalmente, vi è bisogno di più iterazioni dell'esecuzione del gadget
per ottenere la scrittura desiderata. Il metodo, infatti, oltre a
scegliere uno dei gadget disponibili calcola il numero di queste
iterazioni chen sono necessarie per la scrittura dei valori. Per ogni
iterazione poi utilizza una modellazione logica unita ad un risolutore
SMT per ottenere i valori dei registri sorgente e destinazione
corretti. Anche se si potrebbero utilizzare il valore di offset
ottenuto durante la fase di catalogazione del gadget utilizzare il
risolutore ci permette di utilizzare anche gadget i cui effetti
secondari modifichino la locazione di memoria in maniera indipendente
dagli altri registri. Sempre per ogni iterazione, si utilizza la
classe \lstinline{regset} per ottenere una catena che setti i registri
a quei valori, e la si concatena a sua volta al gadget che si sta
utilizzando per scrivere in memoria. Viene restituita la
concatenazione delle catene create per ogni iterazione, che
verosimilmente scriverà in memoria, partendo dall'indirizzo indicato i
valori in ingresso.

\subsection{Selezione dei gadget per la operazioni aritmetiche in
  memoria}
\label{sec:arithmeticstore}

La logica e le strutture dati per la gestione dei gadget che ci
consentono di scrivere in memoria risiedono nella classe
\lstinline{aritmeticmemorystore}.

Anche in questo caso si parte dai gadget classificati dal framework
BARF, ma appartenenti, per l'appunto, alla categoria Arithmetic Memory
Store. Un gadget viene inserito in questa categoria se, dopo alcune
simulazioni del gadget, una locazione di memoria viene modificata e
come valore finale riflette il risultato di un'operazione aritmetica
binaria tra il valore precedente di quella locazione e il valore
iniziale contenuto in un registro. Inoltre si individua un registro
che potrebbe essere usato dal gadget come indirizzo della locazione di
memoria (insieme ad un eventuale offset). Dato che la simulazione
viene ripetuta più volte i falsi positivi vengono ridotti. È possibile
che un gadget esibisca il comportamento di cui sopra più di una volta,
anche in questo caso vengono creati diversi oggetti di tipo Arithmetic
Memory Store. L'analisi di questi gadget procede, come anche nella categoria
precedente, sulla falsa riga della procedura indicata in
\ref{s:verifica_validazione}.

Tuttavia un'attenzione particolare va posta nell'analisi dei gadget
che utilizzano un'operazione di addizione che considerano il valore di
riporto (ad esempio adc su intel). Questi gadget possono rilevarsi
molto utili. Spesso infatti sono disponibili solo gadget aritmetici
che ci consentono di aggiungere un byte per volta. Effettuare
un'operazione di got patching (vedi sez. \ref{sec:got_patching}) un
byte per volta senza considerare i valori di riporto tra un byte e il
successivo può portare ad un risultato errato dell'operazione che si
sta effettuando. Per poter permettere la cataloghizzazione di questo
tipo di gadget, si è dovuto applicare una piccola patch al framework
per permetterci di poter controllare in qualche modo il contesto dei
registri che utilizza durante le simulazioni per la
classificazione. Questo ci ha permesso di impostare durante le
simulazioni il valore del carry flag a zero, facendo in modo che la
procedura di catalogazione non scarti i gadget di questo tipo. Anche
le verifiche tramite un risolutore SMT impongono che il carry flag
abbia valore zero. In più viene aggiunta una seconda coppia di
modellazioni SMT con le quali si verifica se il gadget in esame
utilizza il carry flag nell'operazione e se a seguito di un'operazione
che sicuramente generi un valore di riporto, il carry flag al termine
dell'esecuzione del gadget abbia valore uno. Questo ci dà la certezza
rispettivamente che il gadget va ad utilizzare il carry flag nelle
operazioni di addizione e che le operazioni successive all'interno del
gadget non vadano ad influenzare il carry flag.

Come detto sopra, per costruire una catena che ci consente di
aggiungere un valore arbitrario ad una locazione di memoria
arbitraria, bisogna avere a disposizione o gadget aritmetici la cui
dimensione dell'operazione sia di almeno un byte più grande
dell'addendo o un gadget che, per l'appunto, consenta di eseguire
operazioni utilizzando i valori di ritorno. In questo secondo caso
però bisogna anche concatenare, se è disponibile, un gadget che abbia
come effetto quello di azzerare il carry flag, da inserire prima
dell'operazione. La logica per la ricerca di questo tipo di gadget è
presente nella classe \lstinline{ClearCarryFlag}. È da notare che se
questo tipo di gadget non fosse presente la catena generata potrebbe
non avere l'esito sperato, tuttavia è presumibile che in molti tipi di
exploit il carry flag abbia sempre lo stesso valore all'inizio
dell'esecuzione della catena. La sottrazione quindi di un'unità al
primo byte dell'addendo dovrebbe essere sufficiente per generare una
catena corretta. 

Se non sono disponibili né un un gadget che consenta di aggiungere
abbastanza bytes per volta né un gadget che effettua operazioni
considerando il valore di riporto è tuttavia possibile generare una
catena con alte probabilità di successo. Infatti l'indirizzo di
memoria nel quale viene caricato il file oggetto condiviso della
libreria, nonostante sia casualizzato, dev'essere allineato in memoria
al valore specificato nel parametro \lstinline{p_align} all'interno
del program header. Questo comporta che un certo numero di bits meno
significativi dell'indirizzo di una funzione non cambino da
un'esecuzione all'altra. Solitamente sono i 21bits meno significativi,
e l'offset tra due funzioni nella libc solitamente è tre byte. Questo
fa si che si può calcolare il valore di riporto per i primi due
byte. Solo il valore di riporto dell'ultimo byte (di cui conosciamo
solo la parte meno significativa) risulterebbe casuale, facendo si che
si possano generare catene con probabilità di successo pari almeno al
50\%.


\section{Modulo Payload}

Questo modulo si occupa di generare a partire da una lista ordinata di
uno o più gadgets una stringa binaria di dati da iniettare nello stack
del file oggetto eseguibile esaminato, denominata Payload. La
preparazione del payload varia rispetto al gadget per il quale si sta
creando il chunk e può consistere nel solo indirizzo del gadget, o in
schemi più complicati come indirizzi di funzioni più relativi
argomenti o a valori da trasferire dalla stack ai registri. Ogni
classe che si occupa della generazione di un particolare tipo di chunk
ha le informazioni necessarie per permettere al chunk stesso di essere
concatenato ad altri chunk per generarne un altro, anch'esso a sua
volta concatenabile, che rappresenti i chunk di partenza. Per
permettere questa concatenazione ogni chunk deve presentare le
seguenti informazioni:

%3201899105

\begin{itemize}

\item L'indirizzo che deve trovarsi sullo stack al momento
  dell'esecuzione dell'istruzione ret del gadget precedente 

\item In quale posizione del payload inserire l'indirizzo del gadget
  successivo

\item La dimensione degli indirizzi nell'architettura considerata

\end{itemize}


Le classi che modellano i vari tipi di chunk hanno tutti un metodo,
\lstinline{get_payload}, che restituisce il payload del chunk di
riferimento (escluso l'indirizzo del gadget stesso). Quando due gadget
vengono concatenati per generare un chunk più generico (nel metodo
\lstinline{PayloadChunk.get_general_chunk}), l'indirizzo di ogni chunk
viene inserito nella posizione indicata dal chunk precedente. Nel
momento che si richiede il payload vero e proprio (attraverso il
metodo \lstinline{PayloadChunk.chain}) oltre all'operazione di
concatenazione viene anche inserito l'indirizzo del primo gadget
all'inizio del payload.




 %TODO spostare in implementazione attuale?
%% Questo avviene emulando tutti i gadgets, e nello stesso momento
%% parsando le istruzioni. Si tiene traccia, istruzione per istruzione,
%% di quanto si sia modificato il valore dello stack. Nel caso si
%% incontri un'istruzione del tipo pop reg, viene associato a quel
%% registro l'attuale differenza tra il valore dello stack relativamente
%% all'istruzione corrente e quello all'inizio della simulazione. A
%% questo punto abbiamo per ogni gadget una lista di registri e di offset
%% rispetto ai quali inserire il valore nello stack per inserire un
%% valore arbitrario nel relativo registro. Inoltre tramite la
%% simulazione si deduce la differenza tra il valore dello stack
%% all'inizio del gadget e quello alla fine. Questo valore dovrà essere
%% noto solitamente per tutti i gadget in quando ci permette di far
%% trovare al posto giusto l'indirizzo di ritorno del gadget lo che
%% succede nella catena. Nel caso il gadget non contenga nessuna
%% istruzione del tipo pop reg, questo viene scartato. Inoltre nella
%% simulazione viene tenuta traccia degli indirizzi di memoria letti e/o
%% scritti. Nel caso vi sia scrittura nella memoria o lettura da una zona
%% di memoria non riconducibile all'area relativa allo stack il gadget
%% viene scartato. 


%% La semantica potrebbe essere poi verificata trasformando le istruzioni
%% del gadget in formule smt, verifcando che la semantica e i valori
%% estratti non siano dipendendi dai valori di ingresso, e quindi
%% relativi solo alla simulazione effettuata. In pspecial modo è possibile
%% che il valore dello stack venga modificato utilizzando un valore di un
%% registro. TODO




%% La catena dei gadget generata da dROPper deve, una volta
%% iniettata nello stack del processo in esecuzione, poter eseguire un
%% comando arbitrario sulla macchina in cui è in corso l'esecuzione del
%% programma. La catena generata deve funzionare anche nel caso che gli
%% indirizzi in memoria delle librerie utilizzate da programma siano
%% stati casualizzati (ASRL) e che nè lo stack nè alcuna regione
%% scrivibile sia eseguibile (NX). La strategia da utilizzare dipende dal
%% binario utilizzato e da che tipo di controllo abbiamo sui dati di
%% ingresso del programma. In ogni caso lo schema principale, con alcune
%% varianti che analizzeremo in seguito, si può dividere in due fasi
%% principali: (a) preparare gli argomenti per eseguire una chiamata ad
%% una funzione di libreria che ci consenta di eseguire un comando
%% arbitrario e (b) eseguire quella funzione al fine si eseguire il
%% comando voluto. Le strategie per arrivare alla concretizzazione di
%% queste due fasi dipendono dal file oggetto eseguibile che si sta
%% analizzando che può presentare scenari relativamente semplici, in cui
%% ad esempio il programma utilizza già le chiamate a funzioni
%% necessarie, permettendoci di utilizzare \emph{return-to-plt} sia a
%% scenari più complessi, dove l'indirizzo delle funzioni necessarie
%% viene ottenuto durante l'esecuzione partendo dagli indirizzi presenti
%% nella Global Offset Table.

%% Dropper inizia ad analizzare il binario affidandosi al framework BARF
%% per l'estrazione e una prima cataloghizzazione dei gadget. Le
%% categorie in cui il framework suddivide i gadget sono quelle indicate
%% in[]:

%% \begin{itemize}


%% \end{itemize}

%% \section{Implementazione attuale}

%% \subsection{dropper}

%% Dropper legge dal file le sezioni
%% \lstinline{.rel.plt}, \lstinline{.dynsym},\lstinline{.got} e
%% \lstinline{.got.plt}.

%%  analizza
%% poi la sezione ``.data'', se non viene specificato diversamente
%% dall'utente verrà utilizzato l'indirizzo in memoria di questa sezione
%% come area di memoria dove poter scrivere dati. Questo ci permette di
%% avere una posizione nota (l'indirizzo in memoria di ``.data.'' non è
%% casualizzato) dove poter scvrivere gli argomenti delle funzioni che
%% andremo ad utilizzare



%% TODO :
%% # verificare i gadget di tipo regset
%% # trovare i tipi regset e modellare la memoria (utilizzando > e < per
%% segnare le aree di memoria scrivibili e le aree di memoria leggibili)


%% Una prima
%% possibiltà è quella di utilizzare delle sequenze di gadget che ci
%% consentato di modificare un valore in memoria aggiungendoci un
%% addendo. In questo modo è possibile addizionare l'offset noto nell got
%% table, e successivamente fare un ret-to-plt. Il valore presente nella
%% got table adesso punterà alla funzione voluta. Per utilizzare questa
%% tecnica devono essere soddisfatte le seguenti precondizioni:
