\documentclass[12pt,oneside,a4paper,titlepage]{book}
%% \oddsidemargin52pt      %commentare se si stampa solo fronte
%% \evensidemargin10pt     %commentare se si stampa solo fronte


\usepackage[italian]{babel}
%\usepackage[signatures]{frontespizio}
%packages
\usepackage[utf8x]{inputenc}
%\usepackage{graphicx}

\usepackage{listings}
%\usepackage{amsmath}
%\usepackage{amsfonts}
\usepackage{fancyhdr}

%listening
\lstset{
language=Python,                % choose the language of the code
numbers=left,                   % where to put the line-numbers
%stepnumber=2,                   % the step between two line-numbers. If it's 1 each line 
%numbersep=5pt,                  % how far the line-numbers are from the code
%frame=single,                % adds a frame around the code
tabsize=2,                % sets default tabsize to 2 spaces
captionpos=b,                   % sets the caption-position to bottom
breaklines=true,                % sets automatic line breaking
breakatwhitespace=false,        % sets if automatic breaks should only happen at whitespace
stringstyle=\footnotesize,
}

%-----------------------------------------------------------------
%Intestazione e piè di pagina
%-----------------------------------------------------------------

%\pagestyle{headings}

\pagestyle{fancy}
\renewcommand{\chaptermark}[1]{\markboth{#1}{}}
\renewcommand{\sectionmark}[1]{\markright{\thesection\ #1}}
\fancyhf{}

\fancyhead[LE,RO]{\bfseries\thepage}
\fancyhead[LO]{\bfseries\rightmark}
\fancyhead[RE]{\bfseries\leftmark}
\renewcommand{\headrulewidth}{0.3pt}
\renewcommand{\footrulewidth}{0pt}
\addtolength{\headheight}{0.3pt} % riserva spazio per la linea
\fancypagestyle{plain}{
    \fancyhead{}                        % ignora, nello stile plain, le intestazioni
    \renewcommand{\headrulewidth}{0pt}  % e la linea
}


%Interlinea
\linespread{1.1}

%Commands
\newcommand{\code}[1]{\lstinline$#1$}

\begin{document}

%Frontespizio
%Per avere il frontespizio compilare tesi.tex, poi tesi-frn.tex e dinuovo tesi.tex:

%pdflatex tesi
%pdflatex tesi-frn
%pdflatex tesi

%% \begin{frontespizio}
%% \Istituzione{Universit\`a della Calabria}
%% \Logo[3.5cm]{unical}
%% \Facolta{Scienze Matematiche Fisiche e Naturali}
%% \Corso[Laurea]{Informatica}
%% \Annoaccademico{2009--2010}
%% \Dipartimento{Matematica}
%% \Titoletto{Tesi di laurea}
%% \Titolo{dROPper: strumento semiautomatico per la generazione payload ROP}
%% \Candidato[117678]{Oliverio Marco}
%% \Relatore{Dott. Ianni Giovanmbattista}
%% \Margini{2.0cm}{1.5cm}{2.0cm}{1.5cm}
%% \end{frontespizio}

\lstset{language=Python}

%Per stampare solo fronte correggere i vari clearpage e newpage 
%neccessari per l'aggiunta di pagine bianche aggiuntive nel fronte-retro.

%% \clearpage{\pagestyle{empty}\cleardoublepage}
%% \clearpage{\pagestyle{empty}\cleardoublepage}
%% \frontmatter
\tableofcontents

\thispagestyle{empty}
\
\newpage
\clearpage{\pagestyle{empty}\cleardoublepage}
\mainmatter
\chapter{Introduzione}
%\addcontentsline{toc}{chapter}{Introduzione}


A seguito dell'introduzione di misure di sicurezza come l'isolamento
delle pagine di memoria, la casualizzazione dello spazio degli
indirizzi, e il controllo della consistenza dello stack le tecniche
per poter sfruttare gli errori presenti in un programma al fine di
ottenerne il controllo sono cambiate radicalmente.

La tecnica forse più abusata era lo \emph{stack smashing}
\cite{Phrack-96}, con la quale si compromisero centinaia di
sistemi. La tecnica consisteva nello sfruttare un errato controllo dei
limiti di un buffer memorizzato sullo stack per poter corrompere lo
stack stesso, errore identificato come \emph{stack buffer
  overflow}. In questo modo poteva essere ``iniettato'' nello stack
codice arbitrario, solitamente definito \emph{shellcode} per via del
suo obiettivo più comune: ottenere una shell. 

Nello stack sono presenti informazioni chiave per il funzionamento del
programma. Ad esempio, al momento di una chiamata a funzione,
l'indirizzo dell'istruzione successiva viene memorizzato sullo stack,
in questo modo, una volta terminato il lavoro della funzione, il
programma sa da dove riprendere l'esecuzione. L'attaccante, una volta
preso il controllo dello stack, può sovrascrivere il valore
dell'indirizzo di ritorno, controllando da dove il programma
continuerà. L'attaccante quindi può, compromettendo lo stack,
redirezionare il flusso del programma in modo che esegua il codice da
lui stesso iniettato.

Sono state introdotte diverse misure di protezione per mitigare questo
tipo di attacco. Con l'introduzione dell'isolamento delle pagine di
memoria è stato possibile adottare una politica dei permessi in
memoria con granularità una singola pagina. Il tentativo di eseguire
delle istruzioni presenti in una pagina non eseguibile, così come il
tentativo di scrivere su una pagina non scrivibile causa l'arresto del
programma. Il codice legittimo viene mappato in memoria in pagine
eseguibili ma non scrivibili, mentre stack e dati che il programma
deve manipolare vengono mappati su pagine scrivibili ma non
eseguibili. In questo modo non si può più semplicemente iniettare del
codice, in quanto le aree di memoria scrivibili (tra cui l'area di
memoria riservata allo stack) non potranno essere eseguite. Un'altra
misura di protezione per rendere più difficile questo tipo di attacco
è la casualizzazione dell'indirizzo di memoria nel quale risiede lo
stack. In questo modo, non sapendo dove il codice iniettato
effettivamente risiede, non si sa dove dover redirezionare il
controllo del flusso per eseguire il codice iniettato.

Tuttavia l'isolamento delle pagine di memoria e la casualizzazione
dello stack non evita la compromissione né del flusso del programma né
dello schema dello stack. Una delle tecniche utilizzata per ottenere
il controllo di un programma nonostante lo stack non sia marcato come
eseguibile si chiama \emph{return-to-libc}\cite{c0ntex}. In un
utilizzo tipico di questa tecnica il flusso del programma non viene
dirottato su un payload iniettato dall'attaccante ma su una funzione
di una qualche libreria utilizzata dal programma. Su un architettura a
32bit gli argomenti vengono passati alle funzioni posizionandoli
secondo un ordine preciso (definito dalla \emph{calling convention})
sullo stack. Dato che anche lo schema dello stack può essere
compromesso l'attaccante ha di fatto il controllo degli argomenti da
passare alla funzione. Potendo, ad esempio, richiamare la funzione
\lstinline{system} della libc con argomenti arbitrari è possibile
eseguire un qualsiasi comando sulla macchina in cui si sta eseguendo il
programma, oppure chiamando la funzione \lstinline{mprotect} (sempre
nella libc) con gli opportuni argomenti si possono modificare i
permessi della pagina di memoria dove risiede lo stack, rendendola
eseguibile. A quel punto redirezionare il flusso del programma sul
codice iniettato sullo stack non ne provocherà l'arresto. La tecnica
funziona perché il codice ad essere eseguito risiede in un area di
memoria eseguibile. Una tecnica per mitigare questo tipo di attacco è
la casualizzazione dello spazio degli indirizzi (\emph{Address Space
  Layout Randomization}). Le librerie vengono mappate ad indirizzi di
memoria casuali, che cambiano da esecuzione ad esecuzione del
programma. In questo modo non si conoscono gli indirizzi delle
funzioni e un attacco di tipo return-to-libc naive non risulta
possibile. Tuttavia su 32bit e sotto alcune condizioni questi
indirizzi possono essere facilmente ricavati con un attacco a forza
bruta\cite{Shacham:2004}. Anche se su un'architettura 64bit un attacco
a forza bruta non risulta fattibile è tuttavia possibile che si riesca
a sfruttare qualche altro errore di programmazione per ottenere
l'indirizzo della libreria, riuscendo così a montare comunque
l'attacco.

Una generalizzazione della return-to-libc è la tecnica che prende il
nome di \emph{Return Oriented
  Programming}\cite{Shacham-2007,Roemer-2012}. In questo caso il
flusso viene redirezionato non più all'inizio di una funzione, ma
all'inizio di una piccola serie di istruzioni, già presenti in un area
di memoria del programma marcata come eseguibile, seguita da
un'istruzione ret. L'istruzione ret preleva dalla memoria dello stack
un indirizzo e ridireziona il flusso del programma a quel dato
indirizzo. Viene utilizzata al termine dell'esecuzione di una funzione
per tornare alla porzione di codice che l'ha richiamata. Potendolo
controllare l'attaccante può sistemare lo schema dello stack in modo
che una volta che il programma arrivi all'istruzione ret, sullo stack
sia presente l'indirizzo di una seconda serie dalle caratteristiche
simili alla prima. Ripetendo lo stesso discorso più volte è possibile
concatenare un numero arbitrario di queste piccole serie di istruzioni
(che prendono il nome di \emph{gadget}). In \cite{roemer-12} è stato
dimostrato che, quando sono presenti opportune tipologie di gadget,
utilizzando il ROP si possono effettuare computazioni
turing-complete. Utilizzando opportune sequenze di gadget è possibile
applicare tecniche per eludere varie misure di protezione
\cite{roglia:2009}.

Un programma può essere compilato affinché il suo codice sia
indipendente dalla posizione in cui viene mappato in memoria
(\emph{Position Indipendent Code}), in questo modo non solo le
librerie, ma anche il programma stesso può essere caricato in un
indirizzo casuale di memoria, che cambia tra un'esecuzione e
l'altra. In questo caso non è più facilmente possibile montare un
attacco che sfrutti i gadget, in quanto la posizione degli stessi non
è nota. Altre tecniche sono stati sviluppate per mitigare attacchi di
tipi ROP, anche se non risultano efficaci o l'overhead è troppo grande
\cite{Davi-2014}.

La \emph{Satisfability Modulo Theories}\cite{Barrett-14} è
un'estensione della logica del primo ordine, che viene estesa con
teorie come quella dei bitvector e degli array, rendendolo un ottimo
strumento per la modellazione dell'architettura di un elaboratore
(specialmente dei registri e della memoria). Negli ultimi tempi è
stata dedicata molta attenzione all'utilizzo di risolutori di formule
SMT con applicazione nell'ambito della sicurezza\cite{}
informatica. Riuscendo infatti a modellare le istruzioni e il contesto
di esecuzione di un programma in formule SMT, il risolutore può essere
utilizzato, ad esempio, per verificare che un programma aderisca ad
una data semantica o per estrapolarne la semantica stessa.

Anche la simulazione simbolica viene utilizzata sempre di più
nell'ambito della sicurezza. Ad esempio può essere utilizzata per
mettere in relazione dati in ingresso e comportamento del
programma. Semplificando, una volta trovato un modo per poter simulare
il programma, ad un valore concreto di ingresso viene sostituito un
valore simbolico. Ogni volta che questo valore viene manipolato viene
tenuta traccia delle operazioni effettuate su di esso. Le strutture
dati del programma possono essere espresse in funzione di quel dato
valore simbolico. È possibile poi ricavare in modo automatico una
formula SMT che rappresenti queste relazioni, in modo da poter
interrogare un risolutore per chiedere, ad esempio, per quali valori
in ingresso il programma segua o meno una branch o per quali valori
una struttura dati contenga o no un dato valore.

Nell'ambito del ROP un risolutore SMT può essere utilizzato sia per
estrapolare in maniera automatica la semantica di un gadget, sia per
ottenere le condizioni di partenza che cosentino ad un gadget di
eseguire una determinata funzione (ad esempio i valori di registro per
i quali un dato gadget scrivi in una specifica locazione di memoria un
dato valore).

Lo scopo di questo lavoro di tesi è lo sviluppo di uno strumento che
possa aiutare nel compito di generare una catena di gadget che sia in
grado di prendere controllo di un programma, premesse alcune
condizioni. In particolare lo strumento riesce a generare catene di
gadgets per eludere misure di sicurezza come stack non eseguibile,
casualizzazione dello spazio degli indirizzi e casualizzazione dello
stack applicando, in maniera automatica, tecniche come \emph{got
  patching} e \emph{return to plt} \cite{roglia:2009}. Lo strumento
utilizza per l'analisi del binario il recente framework di sviluppo
BARF \cite{Heitman-14} e supporta l'architettura intel a 32bit e a
64bit.

Descrizioni dei capitoli...

%% NX, ASRL, ROP

%% SMT, AEG, 

%% STRUMENTO

%%% Local Variables: 
%%% mode: latex
%%% TeX-master: "tesi"
%%% End: 

\chapter{dROPper}

La prima parte di questo capitolo descrive l'architettura alla quale
si sta facendo riferimento per lo sviluppo di dropper mentre nella
seconda si farà una analisi dettagliata dell'attuale implementazione.

\section{Architettura principale}

L'obiettivo di dropper è quello di generare, analizzando un file
oggetto eseguibile, una catena di gadget che iniettata nello stack
dell'eseguibile consenta di prenderne il controllo, eseguendo un
comando arbitrario sulla macchina. Lo strumento deve poter applicare
tecniche note per eludere misure di protezioni esistenti come la
casualizzazione degli indirizzi (\lstline{ASRL}), l'isolamento delle
pagine (\lstline{NX}) o la read-only relocation. Dropper analizza il
binario utilizzando alcune delle funzionalità del framework per
l'analisi dei binari \lstline{BARF}. Il linguaggio utilizzato per la
scrittura di dropper è python. Il file oggetto eseguibili supportati
da dropper sono ELF32 e ELF64 ed è destinato ad analisi di binari su
piattaforme GNU/Linux. Dropper, nella sua fase di analisi, estrapola
dal file binario i gadgets disponibli, per poi estrapolarne la
semantica. Per questi compiti dropper si avvale sia dell'emulazione
dei gadget, sia della loro modellizzazione in formule SMT, come
descritto più in dettaglio nelle prossime sezioni. Una volta
estrapolata la semantica dei gadget è possibile costruire piccole
catene di uno o più gadget che consentano di eseguire operazioni
particolari (come modificare il valore dello stack, chiamare una
funzione, settare un registro, scrivere in memoria, ecc). Dropper
assembla queste piccole catene in catene sempre più lunghe che
eseguono operazioni più complesse e articolate. La generazione di
queste sequenze avviene, nell'implementazione attuale, in maniera
programmatica, cioè la procedura per la sua costruzione è codificata
in un algoritmo. È interessante esplorare metodi di generazione di
sequenze in maniera dichiarativa, ad esempio utilizzando linguaggi di
planning come \lstinline{PDDL}\cite{pddl-97}. Una possibile modellizzazione è
quella in cui per ogni sequenza ``elementare'' vengono codificate le
precondizioni e gli effetti. Viene modellato lo stato del sistema
(memoria e registri) che si desidera ottenere dopo l'esecuzione di una
sequenza di gadget. Una volta codificato il problema in questi termini
si può chiedere ad un planner, come se fosse una sorta di oracolo, in
che sequenza collegare le varie azioni (cioè le varie sequenze
elementari di gadget) per ottenere quello stato. La rappresentazione
delle informazioni sulla semantica dei gadget è espressa in questo
modo in forma esplicita, e non più codificata in un algoritmo.

Dropper è composto da tre moduli principali:

\begin{itemize}

  \item dropper: È il modulo che si occupa di (a) leggere il binario,
    (b) coordinare le operazioni dei restanti moduli, (c) interagire
    con l'utente per ottenere informazioni aggiuntive sul binario e/o
    sullo scenario nel quale dev'essere prodotta la catena e (d)
    scegliere le tecniche da utilizzare per riuscire a generare la
    catena

  \item gadgetstools: È il modulo che si occupa del'estrapolazione,
    dell'analisi, della scelta e dell'assemblaggio dei gadget da utilizzare nella
    composizione della catena e dei vari sui frammenti

  \item payloadchunk: È il modulo che si occupa di trasformare una
    sequenza ordinata di gadget in un payload vero e proprio da
    iniettare sullo stack, gestendo il posizionamento corretto degli
    indirizzi e dei valori utilizzati dai vari gadget

\end{itemize}

\subsection{Modulo dropper}

È il modulo che si occupa di leggere il file binario, interagire con
l'utente, scegliere le varie strategie da adottare e di coordinare le
azioni degli altri moduli. Il modulo legge il file e instanzia un
oggetto di tipo gadgets tools che si occuperà di estrapolare i gadgets
dal binario e analizzarli. Per il parsing delle strutture dati
presenti nell'elf dropper utilizza la libreria
\lstinline{pyelftools}\cite{pyelftools}. Combinando le informazioni
provenienti da queste sezioni dropper ricava la lista delle funzioni
che il file oggetto eseguibile utilizza e le relative entry nella
\emph{Procedure Linkage Table} (vedi \rif{}). La presenza di alcune
funzioni può semplificare molto la generazione della catena, di fatto
trasformandola in una serie di \emph{return-to-plt}. Dropper analizza
altre strutture chiave dell'ELF per individuare aree nella memoria
dell'immagine del processo dove è possibile scrivere. L'utente,
interagendo con il modulo, fornisce informazioni aggiuntive necessarie
alla generazione della catena e linee guida sulla strategia
utilizzare. Tra le informazioni che il modulo non può ricavare
automaticamente troviamo:

\begin{itemize}
  \item La possibilità da parte dell'utente di controllare un file
    descriptor

  \item I file oggetto condivisi utilizzati dall'eseguibile sulla
    macchina dove si intende far eseguire la catena (per la
    ricavazione degli offset da utilizzare nelle tecniche di
    sovrascrizione della GOT Table vedi \rif{})

  \item Eventuali caratteri che non possono comparire all'interno del
    payload

  \item Eventuali limiti sulla lunghezza dei payload

  \item Il comando da eseguire

\end{itemize}

Con tutte queste informazioni il modulo può quindi procedere nello
stabilire quale strategia utilizzare e richiedere agli altri moduli le
sequenze di gadget necessarie. Lo schema principale, con alcune varianti
che analizzeremo in seguito, si può dividere in due fasi: (a)
preparare gli argomenti per eseguire una chiamata ad una funzione di
libreria che ci consenta di eseguire un comando arbitrario e (b)
eseguire quella funzione.

Alcune delle strategie che si possono utilizzare per raggiungere il
primo scopo insieme alle precondizioni necessarie sono elencate di
seguito:

\begin{itemize}

\item  Utilizzo della funzione \lstinline{read}

  Condizioni necessarie: 
  \begin{itemize}
   \item Controllo di un file descriptor da parte dell'utente
   \item Utilizzo da parte del file eseguibile della funzione
     \lstinline{read} o possibilità di richiamarla attraverso tecniche
     basate sulla ricavazione degli indirizzi partendo da indirizzi
     nella \lstinline{GOT} e offset relativi
  \end{itemize}

  Viene richiamata la funzione \lstinline{read}, impostando gli
  argomenti in modo che i dati letti dal file descriptor vengano
  scritti nell'area di memoria voluta

\item Utilizzo di gadget per la scrittura della memoria

  Condizioni necessarie:
  \begin{itemize}

    \item All'interno del file oggetto eseguibile deve essere presente
      almeno un gadget in grado di scrivere in memoria un valore
      arbitrario

    \item I valori da scrivere nella memoria non possono presentare
      caratteri che non possono essere presenti nel payload, essendo
      essi stessi scritti nel payload della catena generata

  \end{itemize}
  Solitamente una sequenza del genere consente di scrivere al massimo
  8 byte per volta in architettura 64bit (4 byte per volta su
  un'architettura a 32bit) e sono composte da almeno due gadget
  (almeno un gadget serve per settare i registri utilizzati del gadget
  che esegue la scrittura vera e propria). Questa tecnica tende quindi
  a generare catene piuttosto lunghe se la dimensione dei valori da
  scrivere in memoria è grande

\end{itemize}

Queste due sono le uniche due tecniche attualmente implementate,
strategie alternative potrebbero comprendere:

\begin{itemize}

\item Utilizzo delle funzioni \lstinline{strcpy} o \lstinline{memcpy}
  per copiare byte presenti in posizioni note all'interno
  dell'eseguibile nella posizione desiderata
  \cite{roglia:2009}. Questa tecnica, copiando verosimilmente solo un
  byte per volta tende anch'essa a generare payload abbastanza lunghi

\item Nel caso un bug renda possibile la conoscenza dell'indirizzo
  dello stack gli argomenti possono essere scritti direttamente nello
  stack

\end{itemize}

Una volta scritti gli argomenti per la funzione da richiamare in
memoria è necessario chiamare la funzione stessa. La funzione deve
permetterci di eseguire un comando arbitrario. All'interno della
\lstinline{libc} sono presenti due funzioni con queste
caratteristiche: \lstinline{excve} e \lstinline{system}. Un ritorno
diretto alla funzione solitamente non può avvenire, in quanto il
valore del suo indirizzo in memoria è casualizzato. Tuttavia, se una
funzione è utilizzata all'interno del programma, basterà ritornare
alla sua entry nella Program Linkage Table, che ha una posizione nota
e fissa all'interno della memoria dell'immagine del processo. Nel caso
invece che la funzione non venga utilizzata un'alternativa è quella di
adoperare tecniche per ricavare l'indirizzo della funzione partendo
dall'indirizzo di un'altra funzione della stessa libreria che invece
viene utilizzata dal programma. Questo funziona perchè nonostante
l'indirizzo inziale al quale una libreria viene caricata sia
casualizzato, gli offset tra le funzioni non cambiano. Alcune
considerazioni sull'utilizzo di questa tecnica:

\begin{itemize}

  \item Devono essere noti gli offset delle funzioni all'interno del
    file oggetto condiviso della libreria utilizzato sulla macchina
    bersaglio (anche se, variando in un range relativamente piccolo,
    non dovrebbe essere impossibile, se le condizioni lo consentono,
    montare un attacco di tipo forza bruta per ricavarli)

  \item il binario deve presentare una sequenza di gadgets che ci
    consenta di eseguire un'operazione di somma (o un'operazione che
    ci consenta si simulare una somma, come una sottrazione) tra un
    offset di tre/quattro byte e un'area di memoria arbitraria (l'area
    dove è memorizzato l'indirizzo della funzione usata come
    riferimento, solitamente nella GOT). Molto spesso però abbiamo
    gadget che ci consentono di addizionare solo un byte per
    volta. Questo è un problema perchè in un'operazione di somma byte
    per byte dev'essere considerato un eventuale valore di riporto tra
    un byte e il successivo. Questo può essere risolto se il tipo di
    gadget trovato usa un'istruzione di tipo \lstinline{adc} (add with
    carry), controllando che tra gli effetti secondari degli altri
    gadgets non venga modificato il carry flag. In realtà anche non
    avendo a disposizione questo tipo di gadget, per via
    dell'allineamento in memoria, gli ultimi bit dell'indirizzo di una
    funzione non variano tra un'esecuzione e la successiva. Con
    questa parziale conoscenza del valore iniziale si può comunque
    creare una catena con un alta percentuale di successo come
    descritto in \rif{}

  \item Un tipo di attacco di questo tipo può essere mitigato se le
    rilocazioni vengono impostate come read-only. In questo caso
    infatti gli indirizzi delle funzioni vengono calcolati tutti al
    momento di caricamento del programma, e la GOT, una volta
    modificata opportunamente, viene resa non scrivibile. In questo
    caso è ancora possibile montare un attacco del genere, ma
    l'operazione aritmetica non può avvenire direttamente sulla got e
    non può essere utilizzato un ret-to-plt per saltare all'indirizzo
    calcolato. Bisogna cioè avere a disposizione i gadget necessari
    per leggere dalla memoria, effettuare l'operazione aritmetica e
    eseguire un jmp al nuovo indirizzo calcolato

\end{itemize}

È interessante notare come, anche se vi è una guida nelle operazioni
da seguire basata sulla strategia scelta, la seguenza di operazioni
(cioè di gadget) che conducono allo stato del sistema desiderato sono
molteplici. Ad esempio l'operazione aritmentica per il calcolo
dell'indirizzo della funzione bersaglio potrebbe avvenire in memoria,
oppure tra due registri e poi salvata in memoria. Qui si notano i
vantaggi che potrebbe avere un approccio che modelli la situazione in
un problema di planning rispetto ad un approccio di tipo
programmatico. Infatti, con un approccio programmatico, tutte le
combinazioni vanno codificate nell'algoritmo, una procedura la cui
completezza è più difficile da dimostrare e che è molto più soggetta
ad errori.

\subsection{Modulo gadgetstools}

Il modulo gadgets tools si occupa di ricavare i gadgets presenti nel
file oggetto eseguibile, di estrapolarne la semantica e della
creazione di sequenze di gadget che se eseguite assolvino funzioni
particolari (come scrivere in memoria, settare un registro,
ecc.). Questo modulo interagisce con payloadchunk per trasformare
seguenze ordinate di gadget in un payload vero e proprio da poter
iniettare sullo stack.

Per la ricerca dei gadget all'interno del file oggetto eseguibile
viene utilizzata la classe \lstinline{GadgetFinder} del framework
BARF, che implementa al suo interno l'algoritmo galileo\cite{roemer-12}. 

%TODO descrivere galileo ?
%TODO utilizzare il nostro ?

Una volta disponibile l'insieme dei gadget presenti nel file oggetto
eseguibile bisogna, per ognuno di essi, estrapolarne la semantica. Per
estrapolarne la semantica si intende estrapolare in maniera automatica
gli effetti che una serie di istruzioni hanno sul contesto il quale,
semplificando, è rappresentato dalla memoria e dai registri. Per
questo tipo di analisi si utilizzano due strumenti principali: la
simulazione e la traduzione delle istruzioni in formule
SMT. Un'analisi completa della semantica presenta alcuni limiti dovuti
alla difficoltà di una modellizzazione completa del sistema nonchè
alla conoscenza parziale del contesto (memoria e registri) reale in
cui si troveranno ad essere esguite le istruzioni. Ad esempio
utilizzando l'attuale traduzione delle istruzioni in formule SMT, è
difficile porre al risolutore domande che riguardino il numero di
letture effettuate, o domandare se un'area di memoria viene mai
letta. L'estrapolazione della semantica di un gadget avviene in tre
fasi: (a) emulazione e cataloghizzazione, (b) verifica e validazione e
(c) verifica degli effetti secondati.

\subsubsection{Emulazione e cataloghizzazione}
La classe \lstinline{GadgetClassifier} emula il gadget tenendo traccia
delle operazioni sui registri e sulla memoria. Con questi dati si
verifica se il gadget appartiene ad una o più categorie indicate in
\rif{}. %TODO vanno elencate le categorie?  Inoltre vengono
identificati quali registri vengono utilizzati come operandi della
funzione alla quale il gadget viene associato. Basandosi su una
simulazione e non su un effettiva analisi semantica questa
cataloghizzazione deve essere poi verificata. Per le simulazioni
vengono utilizzati dati concreti ma una simulazione simbolica,
funzionalità ancora non disponibile all'interno del framework BARF,
consentirebbe una cataloghizzazione più veloce e accurata esplicitando
le relazioni tra i valori che i registri hanno all'inizio e alla fine
della simulazione

\subsubsection{Verifica e validazione}
Si deve poi verificare che, anche se il gadget esegua la funzione
della categoria alla quale viene assegnato, possa essere utilizzato
per i valori di tutto il ``dominio'' di questa funzione, cioè che non
ci siano restrizioni sugli operandi del gadget. Per esempio, un gadget
catalogato come \lstinline{MemoryStore} ma che utilizzi lo stesso
registro come operando di destinazione e operando sorgente non può
essere utilizzato per scrivere in una locazione arbitraria di memoria
un valore arbitrario. Per questa verifica si scelgono casualmente gli
effetti che deve avere la funzione sul contesto. Ad esempio se stiamo
verificando un \lstinline{MemoryStore} si sceglie sia una locazione
casuale di memoria sia un valore casuale da scrivere in questa
locazione. Si traduce il gadget in formule SMT e si impone che il
contesto finale rifletta gli effetti, scelti casualmente, che la
nostra funzione deve avere. Nel nostro esempio si impone che la
locazione di memoria cambi al valore casuale scelto. Si chiede a
questo punto al risolutore se il sistema di formule ammette una
soluzione. Se il sistema non ammette soluzione il gadget può essere
scartato, in quanto non essendo usabile per quei valori perde la sua
generalità. Nel caso invece il sistema ammetta soluzione vengono
richiesti al risolutore i valori iniziali, all'interno del modello
trovato, dei registri che, secondo la cataloghizzazione al punto uno,
sono gli ``operandi'' della funzione. Si impongono poi al risolutore
nuovamente le formule SMT che rappresentano il gadget.  Questa volta
si impongono però come valori iniziali per i registri operandi i
valori ottenuti dalla prima verifica e, inoltre, che il valore finale
sia diverso da quello voluto. La non soddisfacibilità di questa
formula ci dà la certezza che in tutti i modelli che soddisfano la
semantica del gadget, che hanno come valori iniziali dei registri
``operandi'' quelli ottenuti dalla prima verifica, portano lo stato
del contesto ai valori voluti. Questa verifica esclude che ci siano
effetti secondari che possono interferire con gli effetti della nostra
funzione. È da notare che queste proprietà sono verificate solo per
dei valori casuali e dovrebbero essere verificate per tutti i
valori. Questo non viene fatto sia per questioni di efficenza sia per
non includere quantificatori universali nelle formule, complicando di
molto il lavoro del risolutore e rischiando di ottenere formule non
decidibili. Questo problema può essere risolto ripetendo queste
verifiche al momento dell'utilizzo reale del gadget con i valori
concretamente utilizzati.

\subsubsection{Verifica degli effetti secondari}  

Il punto precedente ci assicura che un gadget non abbia effetti
secondari che interferiscano con gli effetti della nostra
funzione. Tuttavia è possibile che il gadget in esame abbia altri
effetti secondari che non lo rendano utilizzabile. Ad esempio se il
gadget legge da una locazione di memoria non leggibile causerà
l'arresto del programma. La modellizzazione in SMT utilizzata
attualmente nel framework BARF non consente di porre al risolutore
domande sul numero di letture in memoria o sulla lettura o meno di
un area di memoria. Per questo motivo per escludere effetti
secondari di questo tipo il gadget viene emulato con i valori di
ingresso ricavati al punto 2, e ci si assicura che non vengano lette
e scritte aree di memoria che non siano quelle relative alla
funzione propria del gadget o dello stack.


La linea guida generale appena descritta va adattata alla tipologia di
gadget analizzata, per le implementazioni specifiche fare riferimento
alla seconda parte di questo capitolo. Inoltre per poter essere
utilizzato un gadget, i registri utilizzati come ``operandi'' devono
essere ``controllabili'', cioè deve esistere una serie di gadget che
consenta di poter caricare valori arbitrari in questi registri. Per
questo prima di procedere alla verifica dei gadget si cercano
all'interno dell'eseguibile tutte le sequenze di gadget che ci
permettono di controllare registri. Se un gadget utilizza come
operandi registri che non sono ``controllabili'' questo viene
scartato.

Inoltre è da notare come la verifica potrebbe avvenire in modo più
efficente, utilizzando delle simulazioni mirate a scartare i gadget
che sicuramente non ci consentono di utilizzare valori arbitrari. In
questa fase dello sviluppo del progetto però l'efficenza non viene
considerata come prioritaria.  Durante queste le fasi di
classificazione e verifica vengono anche raccolte informazioni
aggiuntive sui gadget, come di come un gadget modifica il valore del
registro dello stack e la posizione dello stack dove inserire i valori
utilizzati dal gadget. Queste informazioni vengono utilizzate dal
modulo \lstinline{PayloadChunk} per costruire il payload vero e proprio.


Per le simulazioni viene utilizzata la classe \lstinline{ReilEmulator}
e la traduzione in formule smt avviene attraverso la classe
\lstinline{CodeAnalyzer}, entrambe del framework BARF.


\subsection{Payload Chunk}

Questo modulo si occupa di generare il payload vero e proprio partendo
dai gadget analizzati dal modulo gadgetstools. Questa classe è
strutturata in modo che i diversi chunk prodotti siano concatenabili,
e che la loro concatenazione dia a sua una volta un frammento
contatenabile. Questo ci consente di associare ad una data funzione un
payload e utilizzarlo in diverse parti del programma senza dover più
conoscerne i dettagli. Ci sono diverse cose da tenere in conto quando
si costruisce il payload partendo dai gadgets, quasi tutte legate alla
manipolazione dello stack da parte del gadget stesso:

\begin{itemize}

  \item posizioni dei valori utilizzati dal gadget in relazione alla
    valore puntato dallo stack al momento della sua esecuzione

  \item valore al quale punterà lo stack al momento dell'esecuzione
    dell'istruzione ret del gadget stesso. In base alla tipologia di
    gadget questo valore potrebbe non essere alla fine del payload
    relativo al gadget stesso
    
  \item valore finale dello stack dopo l'esecuzione del gadget

\end{itemize}

Le informazioni relative alle tre problematiche descritte sopra
vengono estrapolate dal gadgetstools e sono disponibili al payload
chunk. La procedura del payload è specifica del tipo di gadget, i
metodi \lstinline{PayloadChunk.chain} e
\lstinline{PayloadChunk.get_general_chunk} sono invece generici,
prendono come argomento una lista di chunk e restiuiscono
rispettivamente il payload dei chunk concatenati o uno chunk che può
essere nuovamente concatenato e rappresenta la concatenazione dei
chunks in ingresso.


\section{Implementazione attuale}

%TODO spostare in implementazione attuale?
%% Questo avviene emulando tutti i gadgets, e nello stesso momento
%% parsando le istruzioni. Si tiene traccia, istruzione per istruzione,
%% di quanto si sia modificato il valore dello stack. Nel caso si
%% incontri un'istruzione del tipo pop reg, viene associato a quel
%% registro l'attuale differenza tra il valore dello stack relativamente
%% all'istruzione corrente e quello all'inizio della simulazione. A
%% questo punto abbiamo per ogni gadget una lista di registri e di offset
%% rispetto ai quali inserire il valore nello stack per inserire un
%% valore arbitrario nel relativo registro. Inoltre tramite la
%% simulazione si deduce la differenza tra il valore dello stack
%% all'inizio del gadget e quello alla fine. Questo valore dovrà essere
%% noto solitamente per tutti i gadget in quando ci permette di far
%% trovare al posto giusto l'indirizzo di ritorno del gadget lo che
%% succede nella catena. Nel caso il gadget non contenga nessuna
%% istruzione del tipo pop reg, questo viene scartato. Inoltre nella
%% simulazione viene tenuta traccia degli indirizzi di memoria letti e/o
%% scritti. Nel caso vi sia scrittura nella memoria o lettura da una zona
%% di memoria non riconducibile all'area relativa allo stack il gadget
%% viene scartato. 

%% La semantica potrebbe essere poi verificata trasformando le istruzioni
%% del gadget in formule smt, verifcando che la semantica e i valori
%% estratti non siano dipendendi dai valori di ingresso, e quindi
%% relativi solo alla simulazione effettuata. In pspecial modo è possibile
%% che il valore dello stack venga modificato utilizzando un valore di un
%% registro. TODO




%% La catena dei gadget generata da dROPper deve, una volta
%% iniettata nello stack del processo in esecuzione, poter eseguire un
%% comando arbitrario sulla macchina in cui è in corso l'esecuzione del
%% programma. La catena generata deve funzionare anche nel caso che gli
%% indirizzi in memoria delle librerie utilizzate da programma siano
%% stati casualizzati (ASRL) e che nè lo stack nè alcuna regione
%% scrivibile sia eseguibile (NX). La strategia da utilizzare dipende dal
%% binario utilizzato e da che tipo di controllo abbiamo sui dati di
%% ingresso del programma. In ogni caso lo schema principale, con alcune
%% varianti che analizzeremo in seguito, si può dividere in due fasi
%% principali: (a) preparare gli argomenti per eseguire una chiamata ad
%% una funzione di libreria che ci consenta di eseguire un comando
%% arbitrario e (b) eseguire quella funzione al fine si eseguire il
%% comando voluto. Le strategie per arrivare alla concretizzazione di
%% queste due fasi dipendono dal file oggetto eseguibile che si sta
%% analizzando che può presentare scenari relativamente semplici, in cui
%% ad esempio il programma utilizza già le chiamate a funzioni
%% necessarie, permettendoci di utilizzare \emph{return-to-plt} sia a
%% scenari più complessi, dove l'indirizzo delle funzioni necessarie
%% viene ottenuto durante l'esecuzione partendo dagli indirizzi presenti
%% nella Global Offset Table.

%% Dropper inizia ad analizzare il binario affidandosi al framework BARF
%% per l'estrazione e una prima cataloghizzazione dei gadget. Le
%% categorie in cui il framework suddivide i gadget sono quelle indicate
%% in[]:

%% \begin{itemize}


%% \end{itemize}

%% \section{Implementazione attuale}

%% \subsection{dropper}

%% Dropper legge dal file le sezioni
%% \lstinline{.rel.plt}, \lstinline{.dynsym},\lstinline{.got} e
%% \lstinline{.got.plt}.

%%  analizza
%% poi la sezione ``.data'', se non viene specificato diversamente
%% dall'utente verrà utilizzato l'indirizzo in memoria di questa sezione
%% come area di memoria dove poter scrivere dati. Questo ci permette di
%% avere una posizione nota (l'indirizzo in memoria di ``.data.'' non è
%% casualizzato) dove poter scvrivere gli argomenti delle funzioni che
%% andremo ad utilizzare



%% TODO :
%% # verificare i gadget di tipo regset
%% # trovare i tipi regset e modellare la memoria (utilizzando > e < per
%% segnare le aree di memoria scrivibili e le aree di memoria leggibili)


%% Una prima
%% possibiltà è quella di utilizzare delle sequenze di gadget che ci
%% consentato di modificare un valore in memoria aggiungendoci un
%% addendo. In questo modo è possibile addizionare l'offset noto nell got
%% table, e successivamente fare un ret-to-plt. Il valore presente nella
%% got table adesso punterà alla funzione voluta. Per utilizzare questa
%% tecnica devono essere soddisfatte le seguenti precondizioni:

\chapter{Panoramica tecnica e strategie di exploit}

L'utilizzo di varie tecniche di protezione hanno reso sempre più
difficile lo sfruttamento di errori nell'esecuzione di un programma al
fine di prenderne il controllo. Questo capitolo introduce brevemente
alcune di queste tecniche e analizza con quali metodi è possibile
eluderle. Anche se alcuni concetti sono indipendenti dall'architettura
e dal sistema operativo questo capitolo tratterà esclusivamente di come
queste tecniche sono implementate in ambiente GNU/Linux, su
architettura 8086 e amd64. Nella prima parte di questo capitolo si
parlerà brevemente dei principali meccanismi con i quali il sistema
operativo, partendo dalle informazioni presenti all'interno di un file
oggetto eseguibile, crea una rappresentazione dinamica del programma,
detta \emph{immagine del processo}. Una volta creata l'immagine del
processo il controllo viene ceduto al programma che inizia la sua
esecuzione. La prossima sezione si soffermerà solo sulle parti
rilevanti per il lavoro svolto.

\section{ELF, loading}

ELF (Executable and Linkable Format) viene utilizzato per 
rappresentare tre tipi di file oggetto:

\begin{itemize}
  \item \emph{file oggetto rilocabile}: contiene dati e codice che
    vengono collegati ad altri file oggetto al fine di ottenere un
    \emph{file oggetto eseguibile} o un \emph{file oggetto condiviso}

  \item \emph{file oggetto eseguibile}: è la rappresentazione statica
    di un programma che può essere eseguito, questo file ha le
    informazioni necessarie affinché il sistema operativo crei la sua
    immagine di processo

  \item \emph{file oggetto condiviso} che contiene codice e dati
    adatti per essere collegati (nel processo di linking) in due
    contesti differenti. Può essere infatti utilizzato sia in fase di
    building insieme ad altri file oggetto rilocabile o file oggetto
    condivisi (\emph{linking statico}) per creare altri file oggetto,
    oppure può essere collegato in fase di creazione dell'immagine del
    processo di un file oggetto eseguibile (\emph{linking dinamico}).

\end{itemize}

Il formato ELF può contenere sia le informazioni necessarie per
processi relativi alla fase di creazione di diversi file oggetto a
partire da file sorgente (in particolare informazioni necessarie per
la fase di linking), sia relativamente allo scopo della creazione
dell'immagine di un processo a partire da un file oggetto
eseguibile. Anche le strutture dati all'interno dell'ELF responsabili
di questi due ``punti di vista'' sono differenti, e, a seconda della
tipologia del file oggetto alcune delle strutture dati possono non
essere presenti. Essendo più rilevante ai fini del nostro lavoro ci
concentreremo principalmente sulle strutture dati rilevanti nelle
operazioni di creazione dell'immagine del processo e nelle procedure
di dynamic linking, ovvero nel processo con il quale si rendono
fruibili all'eseguibile i metodi esposti da una libreria condivisa a
tempo di esecuzione.

\subsection{Program header table}
\label{sec:plt}

L'ELF header (che è l'unica struttura dati che ha una posizione fissa
all'interno del file) funge da una sorta di mappa, e ci consente di
localizzare tutte le altre strutture dati presenti nel file.

Tra queste vi è la \emph{Program header table}. La program header
table è un array di strutture, ognuna delle quali contiene o
informazioni su come costruire un segmento della memoria dell'immagine
di un processo o informazioni per la preparazione dell'immagine
stessa. Una entry all'interno della program header table (su
un'architettura a 32bit) è rappresentata dalla seguente struttura C:

\begin{lstlisting}[caption=program header table entry]
typedef struct {
  uint32_t   p_type;
  Elf32_Off  p_offset;
  Elf32_Addr p_vaddr;
  Elf32_Addr p_paddr;
  uint32_t   p_filesz;
  uint32_t   p_memsz;
  uint32_t   p_flags;
  uint32_t   p_align;
} Elf32_Phdr;
\end{lstlisting}

La struttura relativa ad un'architettura a 64bit è pressoché identica,
l'unica differenza sta nella posizione degli attributi.  L'attributo
\lstinline{ptype} indica il tipo di entry. Per il nostro scopo ci
limiteremo a descrivere le entry di tipo \lstinline{PT_LOAD},
\lstinline{PT_INTERPETER} e \lstinline{PT_DYNAMIC}. Le entry di tipo
\lstinline{PT_LOAD} rappresentano informazioni su un segmento da
caricare in memoria. Il campo \lstinline{p_offset} e
\lstinline{p_filesz} indicano rispettivamente l'offset e la sua
grandezza all'interno del file. \lstinline{p_vaddr} e
\lstinline{p_memsz} invece indicano rispettivamente il base address
nel quale caricare in memoria il segmento e la dimensione che avrà in
memoria (che potrebbe essere più grande rispetto a quella sul
file). \lstinline{p_flags} è molto rilevante per il nostro lavoro e
indica con quali flags il segmento andrà ad essere caricato in memoria
ovvero se il segmento sarà o meno scrivibile, leggibile e/o
eseguibile. Tuttavia nel caso il segmento contenga codice compilato
affinché risulti indipendente dal indirizzo base nel quale risiede in
memoria, il campo \lstinline{p_vaddr} potrebbe essere nullo e la
posizione del segmento in memoria potrebbe essere casualizzata.

La voce all'interno della program header table con il tipo
\lstinline{PT_INTERP} invece contiene una stringa che rappresenta il
percorso di un file oggetto eseguibile o condiviso all'interno del
filesystem, detto interprete. Il sistema operativo crea l'immagine del
processo dell'interprete, dandogli il controllo. Sarà compito poi
dell'interprete creare l'immagine del processo necessaria
all'esecuzione dell'eseguibile. Affinché questo possa accadere
l'interprete avrà accesso alle informazioni presenti nel file
principale. Normalmente l'interprete è costituito da codice
indipendente dalla posizione nella quale viene caricato, che viene
casualizzata evitando conflitti tra gli spazi di memoria utilizzati
dall'eseguibile principale e quelli dell'interprete stesso.

Quando viene creato del codice oggetto che utilizza delle librerie
dinamiche, il linker aggiunge alla program header table un elemento di
tipo \lstinline{PT_INTERPRETER} con impostato come interprete il
\emph{dynamic linker}, che si occupa di trovare le librerie necessarie
all'eseguibile, caricarle in memoria, caricare in memoria i segmenti
dell'eseguibile, risolvere le relocation verso i simboli delle
librerie, e ridare poi controllo all'eseguibile stesso. Come vedremo
la risoluzione di un simbolo potrebbe essere rimandata fin quando non
sia realmente necessario. A supporto di questi processi troviamo
alcune strutture dati, anch'esse aggiunte al file oggetto eseguibile
durante la fase di linking. Queste strutture risiedono all'interno di
segmenti che vengono caricati in memoria, e sono quindi disponibili
durante l'esecuzione del programma. Informazioni su dove trovare
queste strutture dati possono essere ricavati attraverso un'altra
struttura dati del file ELF, la \emph{Section header table}, che è un
array di strutture che descrivono, per l'appunto, le parti che
compongono il file. Tuttavia, anche se anche nel nostro lavoro di tesi
le section vengono utilizzate per ottenere informazioni sul binario
esaminato, è da notare che esse non sono necessarie per un file
oggetto di tipo eseguibile. Un file oggetto infatti può essere
caricato esclusivamente con le informazioni presenti nella program
header table (lo strumento sstrip\cite{sstrip} elimina appunto le
sezioni da un file eseguibile). Tuttavia è conveniente per scopi
illustrativi riferirsi alle strutture dati relative al processo di
dynamic linking esaminandone le rispettive sezioni:

\begin{itemize}

  \item Una sezione \lstinline{.dynamic} che contiene gli indirizzi di
    altre strutture necessarie al processo di dynamic linking e la
    lista delle librerie necessarie all'esecuzione del file oggetto
    eseguibile

  \item Una sezione \lstinline{.hash} che contiene una tabella dei
    simboli

  \item Le sezioni \lstinline{.got} e \lstinline{.plt} contengono
    rispettivamente due tabelle: \emph{la global offset table} e
    \emph{la procedure linkage table}. Nelle sezioni successive
    vedremo come queste due strutture dati vengono utilizzate dal
    dynamic linker per risolvere i simboli e le chiamate a funzioni
    presenti nelle librerie dinamiche

\end{itemize}

\subsection{Global Offset Table e Procedure Linkage Table}
\label{sec:got}

Un programma che sia indipendente dalla posizione in cui viene
caricato in memoria non può contenere al suo interno indirizzi
assoluti. Le global offset table (\lstinline{GOT}) contengono
indirizzi assoluti in un'area di memoria privata destinata ai dati,
non compromettendo quindi il fatto che il codice sia indipendente
dalla posizione (e quindi che sia condivisibile da più immagini di
processo, come nel caso di una libreria dinamica). Il linker, una
volta creata l'immagine di un processo, processa tutte le rilocazioni
(strutture dati che contengono informazioni per la risoluzione dei
simboli) di tipo \lstinline{R_386_GLOB_DAT} e per ognuna di esse
calcola l'indirizzo assoluto. Il linker conoscendo l'indirizzo di
tutti i file oggetto caricati in memoria ha tutte le informazioni
necessarie per calcolare il valore di questi indirizzi. Una volta
calcolati il linker inserisce i valori assoluti nei rispettivi
elementi della \lstinline{GOT}, permettendo all'eseguibile di
accedervi attraverso posizioni relative.

A noi interessa principalmente come la \lstinline{GOT} è coinvolti nel
momento in cui un file oggetto eseguibile esegue una chiamata ad una
funzione di una libreria dinamica, processo in cui è coinvolta anche
una seconda tabella, la procedure linkage table (\lstinline{PLT}).

La procedure linkage table redireziona chiamate a funzioni che non
siano dipendenti dalla posizione a valori assoluti non conosciuti al
momento della fase di linking di un eseguibile. Infatti in fase di
building di un eseguibile il linker fa in modo che le chiamate a
queste funzioni vengano direzionate a elementi della
\lstinline{PLT}. Anche se la \lstinline{PLT} risiede nel segmento di
memoria destinato a contenere il codice di un eseguibile, utilizza
valori nella global offset table, non compromettendo così né
l'indipendenza dalla posizione ne la condivisibilità del codice. È
compito del dynamic linker calcolare i valori degli indirizzi della
funzione ei impostarli nei relativi elementi della
\lstinline{GOT}. Nella restante parte di questa sezione andremo a
descrivere come il dynamic linker utilizza queste due tabelle per
risolvere gli indirizzi. È da notare che faremo riferimento
all'implementazione su un'architettura a 32bit. Tuttavia il concetto
principale resta pressoché invariato anche per un'architettura a
64bit. La \lstinline{PLT} è strutturata come nel listato
\ref{cod:plt}.

\begin{lstlisting}[caption=Procedure linkage table, label=cod:plt]
.PLT0:
    pushl   got_plus_4
    jmp     *got_plus_8
    nop;    nop
    nop;    nop
.PLT1:
    jmp     *name1_in_GOT
    pushl   $offset
    jmp     .PLT0@PC
.PLT2:
    jmp     *name2_in_GOT
    pushl   $offset
    jmp     .PLT0@PC
    ...
    ...
\end{lstlisting}

\begin{itemize}
\item Quando viene creata l'immagine del processo i primi due valori
  nella GOT assumono valori particolari (spiegati di seguito)
\item Quando l'eseguibile esegue una chiamata di una funzione in una
  libreria dinamica il flusso del programma viene indirizzato nella
  \lstinline{PLT}. Nel nostro esempio viene chiamata la funzione
  \lstinline{name1}, che avrà come indirizzo di destinazione
  l'istruzione marcata dall'etichetta \lstinline{.PLT1}
\item Il programma a questo punto esegue un \lstinline{jmp}
  all'indirizzo contenuto nell'elemento nella GOT relativo a
  \lstinline{name1}. Al momento di caricamento del programma, a
  eccezione di casi particolari come descritto di seguito, questo
  elemento è impostato con il valore dell'istruzione successiva al
  \lstinline{jmp} stesso, cioè all'istruzione alla linea 7 del listato
  \ref{cod:plt}
\item Il codice a questo punto salva sullo stack l'offset all'interno
  della tabella delle rilocazioni che permette di individuare la
  rilocazione relativa al simbolo \lstinline{name1}. La rilocazione
  permette di ricavare sia l'elemento della \lstinline{GOT} relativo a
  quel simbolo che il nome del simbolo stesso, fornendo le
  informazioni necessarie al dynamic linker per capire quale simbolo è
  stato chiamato e qual'è l'elemento modificare nella \lstinline{GOT}.
\item A questo punto il programma salta al primo elemento della
  \lstinline{PLT} e, dopo aver salvato sullo stack il secondo elemento
  della \lstinline{GOT}, salta all'indirizzo contenuto nel terzo
  elemento della got, che dà il controllo al dynamic linker
\item Il dynamic linker esamina lo stack, controlla quale simbolo è
  stato chiamato, calcola l'indirizzo e imposta il valore
  nell'elemento relativo nella \lstinline{GOT}. In questo modo una
  seconda chiamata a \lstinline{name1} non causerà una seconda
  chiamata al linker ma salterà direttamente all'indirizzo corretto
\end{itemize}

Quest'approccio nel quale gli indirizzi delle funzioni vengono risolti
solo al momento in cui vengono chiamate si chiama \emph{lazy
  binding}. Se il programma viene eseguito con la variabile d'ambiente
\lstinline{LD_BIND_NOW} impostata gli indirizzi vengono tutti
calcolati e scritti nella \lstinline{GOT} al momento del caricamento
dell'eseguibile. Questo comportamento avviene anche quando si compila
un eseguibile con \lstinline{RELRO}. Inoltre, in questo caso, una
volta modificata la \lstinline{GOT} viene resa non più scrivibile, in
modo da mitigare attacchi di tipo got patching.


\section{Return oriented programming}

Per mitigare classici attacchi di tipo stack-smashing sono state
elaborate diverse tecniche per precludere la possibilità di eseguire
il codice iniettato da un attaccante. La prima di questa tecniche è
stata implementata in una patch di Solar Designer \cite{stackpatch},
che modifica lo schema della memoria dell'immagine di un processo al
fine di rendere le istruzioni presenti nello stack non eseguibili. Dato
che nella maggior parte dei casi l'attaccante utilizzava lo stack come
locazione del codice iniettato questa patch rendeva innocui molti di
questi attacchi. Una tecnica più completa, chiamata ``$W \oplus X$'',
assicura invece che non ci sia una pagina di memoria all'interno del
processo che sia scrivibile ed eseguibile nello stesso momento. Per
eludere questa tecnica un attaccante è costretto a non utilizzare più
codice da lui stesso iniettato ma codice già presente nell'immagine
del processo bersaglio (il primo a suggerire quest'approccio è stato
Solar Designer \cite{solar-return-to-libc}). Dato che la libreria C
standard è praticamente utilizzata in quasi tutti i programmi Unix e
contiene funzioni utili ad un attaccante (come \lstinline{system} o
\lstinline{execve}) il codice utilizzato solitamente è proprio quello
della libc, da cui il nome con cui vengono catalogati questo tipo di
attacchi: \emph{return-to-libc}. Tra le misure per mitigare attacchi
di questo tipo fu ipotizzato la rimozione di alcune funzioni dalle
libc al fine di rendere questo tipo di attacchi meno efficace. Shacham
in \cite{Shacham-2007} ha evidenziato come questo tipo di protezione
fosse in realtà inefficace introducendo per la prima volta la tecnica
da lui battezzata come \emph{Return Oriented Programming} (ROP), in
cui venivano usate piccole serie di istruzioni di codice già presente
nell'immagine del processo e non intere funzioni come in attacchi di
tipo return-to-libc. Anche se, in questa prima applicazione, la
tecnica estrapolava queste serie di istruzioni dalla libc, in
\cite{schwartz-2011} è stato evidenziato come questa tecnica restasse valida
nonostante le istruzioni venissero direttamente prelevate dal testo
dell'eseguibile e non dalle sue librerie. In questo modo si può
utilizzare la \lstinline{ROP} per eludere tutta un'altra serie di
protezioni che si basano sulla casualizzazione della posizione di aree
di memoria chiave all'interno dell'immagine del processo di un
eseguibile (in particolar modo delle aree che contengono le librerie
condivise e di quella riservata allo stack) detta appunto \emph{Addres
  Space Layout Randomization}(ASLR). Nell'applicazione della
\lstinline{ROP} vengono individuate piccole serie di istruzioni con
delle particolari caratteristiche, chiamate \emph{gadget}. Arrangiando
minuziosamente lo stack si possono eseguire gadget uno dietro l'altro,
cicli o computazioni arbitrarie. Un punto importante è che, almeno per
quanto riguarda l'architettura intel, possono essere individuate
all'interno di un file oggetto molte sequenze di istruzioni che non
sono ``intenzionali'', ovvero inserite dal compilatore come le
istruzioni relative alla traduzione di codice sorgente. Infatti,
essendo la codifica delle istruzioni in linguaggio macchina molto
densa e non allineata, se si iniziano ad interpretare le istruzioni
partendo dal centro di un'altra istruzione, vi è molta probabilità di
decodificare una sequenza di istruzioni alternativa e valida. Questo è
dovuto alla natura stessa della codifica, quello che Shacham chiama
geometria.

\subsection{Gadget}

In questa sezione analizzeremo più in dettaglio i gadget e le loro
caratteristiche. Bisogna innanzitutto considerare che essendo i gadget
composti da una piccola serie di istruzioni questo tipo di attacchi
agisce ad un livello più basso rispetto a quello degli attacchi di
tipo return to libc. I gadget possono essere visti come delle
istruzioni di basso livello di uno strano calcolatore. Queste
istruzioni non vengono concatenate in maniera standard ma arrangiando
minuziosamente i valori presenti nello stack. I gadget sono terminati
dall'istruzione \lstinline{ret} e nelle istruzioni che lo compongono
non sono contenuti salti e altre istruzioni che deviano in qualche
modo il flusso del programma (tecniche che utilizzano anche questo
tipo di gadget sono state sviluppate, come in
\cite{JOP}). Aggiuntando lo stack in modo che nel momento in cui il
flusso del programma arrivi all'istruzione \lstinline{ret} di un
gadget il registro dello stack punti alla zona di memoria che contiene
l'indirizzo del prossimo gadget si riesce a concatenare un gadget al
successivo. I gadget, soli o in combinazione, possono svolgere diverse
funzioni. Per quanto riguarda, ad esempio, le operazioni di lettura o
scrittura possiamo notare che:

\begin{itemize}
\item gadget del tipo \lstinline{pop REG;ret} ci permettono di
caricare una costante in un registro

\item gadget del tipo \lstinline{mov reg1 ,[reg2 + imm];ret} ci
  permettono di leggere un valore dalla memoria (il valore di
  \lstinilne{reg2} può essere impostato con un gadget del punto
  precedente) e scriverlo su un registro

\item gadget del tipo \lstinline{mov [reg1 + imm], reg2} ci permettono
  di scrivere in una posizione di memoria (i valori di
  \lstinline{reg1} e \lstinline{reg2} possono essere impostati con
  gadget che ci consentono di caricare una costante di un registro)
\end{itemize}

In base all'eseguibile che stiamo considerando però potrebbero non
essere disponibili gadget esattamente come questi.  In particolare
potrebbero presentarsi altre istruzioni tra quella che a noi
effettivamente interessa (perché svolge la funzione a noi ``utile'') e
l'istruzione \lstinline{ret}. A volte modificando opportunamente lo
stack si può fare in modo che gli effetti di queste istruzioni non
incidano sulla nostra computazione. Ovviamente altri gadget, o
sequenze di gadget, possono svolgere funzioni equivalenti e, il fatto
che una categoria di gadget non è disponibile non implica che la
stessa funzione non possa essere svolta utilizzando gadget e/o
strategie differenti. Ad esempio se non abbiamo a disposizione un
gadget che ci permetta di impostare il registro \lstinline{eax}
tramite \lstinline{pop eax;ret} un'opportuna combinazione di
\lstinline{mov eax, 0xFFFFFFFF;ret} e \lstinline{inc eax;ret} ci
permette (non considerando restrizioni sulla lunghezza del payload) di
impostare valori arbitrari in \lstinline{eax}. Estrapolare, in maniera
automatica, sequenze e combinazioni di gadget che ci permettano di
raggiungere condizioni desiderate o gestire gli effetti secondari
delle istruzioni ``superflue'' è uno dei maggiori problemi di questo
lavoro di tesi.

Le operazioni che l'architettura ci permette di eseguire non si
limitano ad operazioni di lettura e scrittura. Infatti possiamo
eseguire operazioni aritmetiche, operazioni logiche e utilizzare
sequenze di gadget che ci permettono di controllare il flusso del
programma. Come considerato nel paragrafo precedente spesso i metodi
per eseguire queste operazioni non sono diretti, ma è necessario
considerare gli effetti secondari e operare solo con i gadget a
disposizione. Ad esempio possiamo avere a disposizione solo operazioni
che operano su dimensioni di un byte, e per eseguire operazioni su
dimensioni maggiori è necessario concatenare più volte la stessa
operazione eseguendola un byte per volta, oppure un'operazione
potrebbe essere eseguita utilizzando una serie di più operazioni ma
che nel complesso sia equivalenti all'operazione voluta.

\subsection{Address Space Layout Randomization}

L'\emph{'Address Space Layout Randomization} (ASLR) è una tecnica di
protezione che consiste nel caricare in posizione casuale regioni
della memoria del programma. Questo è possibile perché un'istruzione
può riferirsi ad una locazione di memoria tramite la distanza tra il
proprio indirizzo (contenuta nel registro instruction pointer) e la
locazione. Non utilizzando quindi più indirizzi assoluti il codice può
risultare indipendente dalla posizione di memoria in cui è caricato a
patto che che gli offset relativi restino invariati (cioè che il
codice non venga ``mischiato''). Il dynamic linker e alcune strutture
dati vengono utilizzate per far funzionare porzioni di codice
indipendenti anche dagli offset relativi (come nel caso del testo
dell'eseguibile e le librerie dinamiche), utilizzando un livello di
indirezione (vedi sez. \ref{sec:plt} e \ref{sec:got}).

L'utilizzo dell'ASLR porta due complicazioni principali nell'utilizzo
di tecniche basate sulla ROP. Prima di tutto non conoscendo
l'indirizzo di una porzione di codice non si possono conoscere neanche
gli indirizzi dei gadget contenuti all'interno. Questo riduce
notevolmente la quantità di gadget a disposizione, e, a seconda di
quanto codice venga casualizzato può rendere impossibile un attacco
che utilizzi la ROP. Infatti anche se solitamente, per questioni di
efficienza, solo le librerie vengono casualizzate, è possibile
compilare un'eseguibile in modo che una volta eseguito tutte le aree
che contengono codice vengano casualizzate (\emph{Position Indipendent
  Executable} (PIE)). In quest'ultimo scenario, a meno di non ottenere
informazioni sulla posizione di memoria in altri modi (ad esempio
tramite altre vulnerabilità che espongano indirizzi di strutture
dell'eseguibile), l'applicazione della ROP non risulta possibile.

Inoltre a venire casualizzate non sono solo porzioni che contengono
codice, ma anche la porzione di memoria riservata allo stack. Non
conoscendo la posizione dello stack risulta molto più difficile creare
sequenze di gadget che possano eseguire operazioni di controllo del
flusso. Una possibile soluzione potrebbe essere quella di iniettare
gli indirizzi dei gadget in un'altra posizione (nota) e modificare il
valore del registro dello stack perché punti a quella zona di memoria,
ovvero creare un \emph{Fake stack frame}. In più non avendo più
accesso a librerie dinamiche come la libc, difficilmente si riesce a
trovare nella porzione di codice non casualizzato istruzioni che ci
consentano di lanciare una syscall (solitamente il programma fa
affidamento alla libreria C per eseguirle e quindi dovrebbe comparire
tra le sequenze di istruzioni non ``intenzionali''). Tuttavia in
\cite{schwartz-2011} è stato dimostrato come anche utilizzando gadget che si
incontrano in porzioni di codice relativamente piccole è possibile
eseguire computazioni arbitrarie, addirittura concatenando questi
gadget in maniera automatica. In più come vedremo nelle prossime
sottosezioni è possibile, sotto alcune condizioni, ricavare gli
indirizzi di funzioni presenti nella libc partendo dalle strutture
dati coinvolte nelle operazioni di dynamic linking. Se le giuste
tipologie di gadget sono disponibili è possibile quindi eseguire
comandi arbitrari (come lanciare una shell) anche in contesti in cui
stack e librerie sono casuallizati.


\subsection{Return-to-plt, GOT dereferencing e GOT patching}
\label{sec:expl}

In questa sottosezione presupponiamo che il codice del testo
dell'eseguibile non sia stato casualizzato. Anche se le librerie sono
casualizzate se un'eseguibile utilizza una funzione presente in una
libreria dinamica il linker, in fase di building, inserisce un
elemento della PLT relativo a quella funzione (vedi
sez. \ref{sec:plt}). Se la posizione in cui viene caricato il codice è
nota lo è anche quella della PLT e quindi l'indirizzo
dell'elemento. Questo ci permette di poter ``utilizzare'' nella nostra
catena le funzioni che utilizza l'eseguibile bersaglio, dirottando il
flusso all'indirizzo del relativo elemento nella plt, da qui il nome
\emph{return-to-plt}. Una volta utilizzata una funzione il valore
dell'indirizzo assoluto di quella funzione si verrà a trovare
nell'elemento relativo nella GOT (vedi sez. \ref{sec:got}). Come
descritto in \cite{roglia:2009} questo ci permette di utilizzare due
tecniche di attacco particolari: \emph{GOT derefercing} e \emph{GOT
  patching}. Tutte e due le tecniche utilizzano l'indirizzo assoluto
di una funzione contenuto nell'entry relativa nella GOT per calcolare
l'indirizzo assoluto di una funzione della libc desiderata, di fatto
eludendo l'ASRL. Nella prima tecnica il valore viene letto, modificato
tramite un'operazione aritmetica e viene poi eseguito un
\lstinline{jmp} a questo indirizzo. Un esempio di gadget che
consentono questo tipo di attacco sono: \lstinline{add eax, [ebx +
    off]; ret} e \lstinline{jmp [eax]}. La seconda tecnica invece
modifica l'elemento della GOT relativo in loco, utilizzando poi un
return-to-plt per richiamare la funzione voluta. Un esempio di gadget
che può essere utilizzato per modificare la funzione in loco è:

\lstinline{adc byte ptr [esi + 0x5f], bl ; pop ebp ; ret}

Infatti concatenando quattro istruzioni del genere insieme ad un
opportuno gadget che permette di impostare il registro esi è possibile
aggiungere un offset arbitrario in una locazione di memoria
arbitraria. È da notare che questa tecnica non funziona nel caso la
GOT sia in un'area di memoria non scrivibile (ad esempio se
l'eseguibile è stato compilato con l'opzione RELRO) ma i gadget
utilizzati in questa seconda tecnica sono di un tipo più comune
rispetto a quelli utilizzati nella GOT dereferencing.


%%% Local Variables: 
%%% mode: latex
%%% TeX-master: "tesi"
%%% End: 

\chapter{Conclusioni}

Lo sviluppo di uno strumento come Dropper mi ha permesso di studiare
ed esplorare problematiche interessanti e complesse. Uno strumento che
riesce infatti ad analizzare un file binario e da esso estrarne in
maniera automatica le informazioni necessarie per poter costruire un
payload funzionante implica la risoluzione di diverse problematiche
che abbracciano più aree dell'informatica. In particolare in questo
lavoro alcuni strumenti propri dell'intelligenza artificiale hanno
giocato un ruolo cruciale rendendo evidente l'apporto che questa
disciplina può dare all'area della sicurezza informatica in
generale. Tra le problematiche più rilevanti nello sviluppo di uno
strumento come Dropper elenchiamo:

\begin{enumerate}
\item L'estrapolazione della semantica dei singoli gadget
\item L'individuazione di una sequenza di gadget che ci consente di
  eseguire una data operazione. Questo punto è complicato dal fatto che gli
  effetti di un gadget possono interferire con quelli di un altro
  nonché con l'esecuzione stessa del programma in esame
\item Scegliere, in base alle operazioni disponibili, la strategia da
  utilizzare per eseguire l'exploit
\item Individuare una sequenza di operazioni che eseguono con successo
  una data strategia
\end{enumerate}

Il primo punto è stato affrontato, come descritto nei capitoli
\ref{cap:architettura} e \ref{cap:implementazione}, attraverso sia
tecniche di emulazione che tecniche di verifica basate sulla
rappresentazione in formule logiche delle istruzioni dei
gadget. Dropper può estrapolare la semantica di diverse tipologie di
gadget che ci permettono di eseguire operazioni, come il caricamento di
costanti nei registri o l'effettuare operazioni aritmetiche in
locazioni di memoria arbitrarie, e ed è facilmente estendibile per
incorporare ulteriori analisi semantiche che consentano di eseguire
operazioni sempre più complesse.

La scelta di una sequenza di gadget che ci permetta di eseguire
un'operazione articolata è un'altra problematica
interessante. Infatti, oltre ad essere sicuri che la catena abbia gli
effetti desiderati, è interessante cercare, tra le sequenze di gadget
che ci permettano una particolare operazione, quella che produca un
payload di minore lunghezza. Infatti, anche se questo dipende molto
dalla tipologia di errore che si sta sfruttando, spesso si è limitati
nella quantità di dati da poter iniettare.

Le catene che attualmente dropper riesce a generare ci consentono di
avere una visione ad un livello di astrazione più alto delle
operazioni che possiamo eseguire con i singoli gadget. Per generare le
catene di queste operazioni, come descritto nei capitoli precedenti,
si utilizza un approccio per lo più programmatico, usando quando
possibile algoritmi greedy per la generazione di soluzioni di una
certa qualità. Tuttavia il problema di generazione di una catena,
definito come l'ordine nel quale effettuare alcune azioni (i gadget)
per raggiungere un dato stato del sistema, si presta naturalmente ad
una traduzione in un problema di planning.

Alcune considerazioni su questo approccio, insieme ad una prima
possibile modellazione che utilizzi il linguaggio PDDL 2.1
\cite{pddl-97,fox-03} sono riportate nella sez. \ref{sec:pddl}.

Le ultime problematiche che sono state affrontate, nella nostra
panoramica dal basso verso l'alto, sono la scelta di una tecnica di
exploiting e di una sequenza di operazioni per applicare tale
tecniche. Tra le strategie applicate con successo nei primi testcase
troviamo
\begin{inparaenum}[1)]
\item l'utilizzo della funzione ``read'' per la scrittura in memoria
\item utilizzo di sequenza di gadget sempre per la  scrittura in memoria
\item got patching e
\item return-to-plt
\end{inparaenum}

Trovare le sequenze per portare a termine queste tecniche è un
problema per certi versi simile a trovare le catene di gadget per
eseguire le operazioni ``elementari'', ma da un punto di astrazione
più alto.  Vi è qui, tra le difficoltà più rilevanti, quella di
esaminare tutte le possibili sequenze di operazioni per applicare
quella tecnica. In questa fase i vantaggi dell'utilizzo di un planner
sembrano ancora più rilevanti.

I test durante lo sviluppo sono stati effettuati prendendo in esame
alcuni livelli di exploit-exercise\cite{exploit-exercise} e alcuni
eseguibili contenuti nelle bin-utils (ls, echo, mv).

\section{Sviluppi futuri}

Dropper ha prospettive di sviluppo su diversi fronti. 

Sicuramente l'utilizzo di un'esecuzione simbolica semplificherebbe di
molto l'analisi semantica, potendo verificare direttamente le
relazioni tra valori in ingresso ed effetti dei gadget.

Dai primi risultati ottenuti utilizzando un planner,  ci si è resi
conto che quest'area potrebbe portare notevoli vantaggi ma che
richiede ulteriori indagini.

Un'altra idea interessante è cercare di estrapolare non solo gadget
``classici'' ma anche altre tipologie. Ad esempio un'estensione della
ROP consiste nell'utilizzare sequenze di istruzioni che terminano con
un'istruzione di tipo \lstinline{jmp} invece che con una di tipo
\lstinline{ret} (infatti questa tecnica prende il nome di \emph{Jump
  Oriented Programming} \cite{Checkoway-10}).

Un altro esempio è quello di ricercare gadget di lunghezza maggiore ma
con effetti secondari ``controllabili'', ad esempio obbligando il
gadget a seguire un flusso di esecuzione piuttosto che
un altro. Quest'ultima tipologia di gadget è necessaria per eludere
alcune tecniche di mitigazione che si basano sul monitoraggio di un
numero limitato di istruzioni.

%%% Local Variables: 
%%% mode: latex
%%% TeX-master: "tesi"
%%% End: 


%% \clearpage{\pagestyle{empty}\cleardoublepage}
%% \listoffigures \addcontentsline{toc}{chapter}{Indice delle figure}
%% \listoftables \addcontentsline{toc}{chapter}{Indice delle tabelle}
%% \clearpage{\pagestyle{empty}\cleardoublepage}

\bibliographystyle{plain}
\bibliography{tesi}

%% \include{latex}
\end{document}
