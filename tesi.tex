\documentclass[12pt,oneside,a4paper,titlepage]{book}
%% \oddsidemargin52pt      %commentare se si stampa solo fronte
%% \evensidemargin10pt     %commentare se si stampa solo fronte


\usepackage[italian]{babel}
%\usepackage[signatures]{frontespizio}
%packages
\usepackage[utf8x]{inputenc}
%\usepackage{graphicx}

\usepackage{listings}
%\usepackage{amsmath}
%\usepackage{amsfonts}
\usepackage{fancyhdr}

%listening
\lstset{
language=Python,                % choose the language of the code
numbers=left,                   % where to put the line-numbers
%stepnumber=2,                   % the step between two line-numbers. If it's 1 each line 
%numbersep=5pt,                  % how far the line-numbers are from the code
%frame=single,                % adds a frame around the code
tabsize=2,                % sets default tabsize to 2 spaces
captionpos=b,                   % sets the caption-position to bottom
breaklines=true,                % sets automatic line breaking
breakatwhitespace=false,        % sets if automatic breaks should only happen at whitespace
stringstyle=\footnotesize,
}

%-----------------------------------------------------------------
%Intestazione e piè di pagina
%-----------------------------------------------------------------

%\pagestyle{headings}

\pagestyle{fancy}
\renewcommand{\chaptermark}[1]{\markboth{#1}{}}
\renewcommand{\sectionmark}[1]{\markright{\thesection\ #1}}
\fancyhf{}

\fancyhead[LE,RO]{\bfseries\thepage}
\fancyhead[LO]{\bfseries\rightmark}
\fancyhead[RE]{\bfseries\leftmark}
\renewcommand{\headrulewidth}{0.3pt}
\renewcommand{\footrulewidth}{0pt}
\addtolength{\headheight}{0.3pt} % riserva spazio per la linea
\fancypagestyle{plain}{
    \fancyhead{}                        % ignora, nello stile plain, le intestazioni
    \renewcommand{\headrulewidth}{0pt}  % e la linea
}


%Interlinea
\linespread{1.1}

%Commands
\newcommand{\code}[1]{\lstinline$#1$}

\begin{document}

%Frontespizio
%Per avere il frontespizio compilare tesi.tex, poi tesi-frn.tex e dinuovo tesi.tex:

%pdflatex tesi
%pdflatex tesi-frn
%pdflatex tesi

%% \begin{frontespizio}
%% \Istituzione{Universit\`a della Calabria}
%% \Logo[3.5cm]{unical}
%% \Facolta{Scienze Matematiche Fisiche e Naturali}
%% \Corso[Laurea]{Informatica}
%% \Annoaccademico{2009--2010}
%% \Dipartimento{Matematica}
%% \Titoletto{Tesi di laurea}
%% \Titolo{dROPper: strumento semiautomatico per la generazione payload ROP}
%% \Candidato[117678]{Oliverio Marco}
%% \Relatore{Dott. Ianni Giovanmbattista}
%% \Margini{2.0cm}{1.5cm}{2.0cm}{1.5cm}
%% \end{frontespizio}

\lstset{language=Python}

%Per stampare solo fronte correggere i vari clearpage e newpage 
%neccessari per l'aggiunta di pagine bianche aggiuntive nel fronte-retro.

%% \clearpage{\pagestyle{empty}\cleardoublepage}
%% \clearpage{\pagestyle{empty}\cleardoublepage}
%% \frontmatter
\tableofcontents

\thispagestyle{empty}
\
\newpage
\clearpage{\pagestyle{empty}\cleardoublepage}
\mainmatter
\chapter{Introduzione}
%\addcontentsline{toc}{chapter}{Introduzione}


A seguito dell'introduzione di misure di sicurezza come l'isolamento
delle pagine di memoria, la casualizzazione dello spazio degli
indirizzi, e il controllo della consistenza dello stack le tecniche
per poter sfruttare gli errori presenti in un programma al fine di
ottenerne il controllo sono cambiate radicalmente.

La tecnica forse più abusata era lo \emph{stack smashing}
\cite{Phrack-96}, con la quale si compromisero centinaia di
sistemi. La tecnica consisteva nello sfruttare un errato controllo dei
limiti di un buffer memorizzato sullo stack per poter corrompere lo
stack stesso, errore identificato come \emph{stack buffer
  overflow}. In questo modo poteva essere ``iniettato'' nello stack
codice arbitrario, solitamente definito \emph{shellcode} per via del
suo obiettivo più comune: ottenere una shell. 

Nello stack sono presenti informazioni chiave per il funzionamento del
programma. Ad esempio, al momento di una chiamata a funzione,
l'indirizzo dell'istruzione successiva viene memorizzato sullo stack,
in questo modo, una volta terminato il lavoro della funzione, il
programma sa da dove riprendere l'esecuzione. L'attaccante, una volta
preso il controllo dello stack, può sovrascrivere il valore
dell'indirizzo di ritorno, controllando da dove il programma
continuerà. L'attaccante quindi può, compromettendo lo stack,
redirezionare il flusso del programma in modo che esegua il codice da
lui stesso iniettato.

Sono state introdotte diverse misure di protezione per mitigare questo
tipo di attacco. Con l'introduzione dell'isolamento delle pagine di
memoria è stato possibile adottare una politica dei permessi in
memoria con granularità una singola pagina. Il tentativo di eseguire
delle istruzioni presenti in una pagina non eseguibile, così come il
tentativo di scrivere su una pagina non scrivibile causa l'arresto del
programma. Il codice legittimo viene mappato in memoria in pagine
eseguibili ma non scrivibili, mentre stack e dati che il programma
deve manipolare vengono mappati su pagine scrivibili ma non
eseguibili. In questo modo non si può più semplicemente iniettare del
codice, in quanto le aree di memoria scrivibili (tra cui l'area di
memoria riservata allo stack) non potranno essere eseguite. Un'altra
misura di protezione per rendere più difficile questo tipo di attacco
è la casualizzazione dell'indirizzo di memoria nel quale risiede lo
stack. In questo modo, non sapendo dove il codice iniettato
effettivamente risiede, non si sa dove dover redirezionare il
controllo del flusso per eseguire il codice iniettato.

Tuttavia l'isolamento delle pagine di memoria e la casualizzazione
dello stack non evita la compromissione né del flusso del programma né
dello schema dello stack. Una delle tecniche utilizzata per ottenere
il controllo di un programma nonostante lo stack non sia marcato come
eseguibile si chiama \emph{return-to-libc}\cite{c0ntex}. In un
utilizzo tipico di questa tecnica il flusso del programma non viene
dirottato su un payload iniettato dall'attaccante ma su una funzione
di una qualche libreria utilizzata dal programma. Su un architettura a
32bit gli argomenti vengono passati alle funzioni posizionandoli
secondo un ordine preciso (definito dalla \emph{calling convention})
sullo stack. Dato che anche lo schema dello stack può essere
compromesso l'attaccante ha di fatto il controllo degli argomenti da
passare alla funzione. Potendo, ad esempio, richiamare la funzione
\lstinline{system} della libc con argomenti arbitrari è possibile
eseguire un qualsiasi comando sulla macchina in cui si sta eseguendo il
programma, oppure chiamando la funzione \lstinline{mprotect} (sempre
nella libc) con gli opportuni argomenti si possono modificare i
permessi della pagina di memoria dove risiede lo stack, rendendola
eseguibile. A quel punto redirezionare il flusso del programma sul
codice iniettato sullo stack non ne provocherà l'arresto. La tecnica
funziona perché il codice ad essere eseguito risiede in un area di
memoria eseguibile. Una tecnica per mitigare questo tipo di attacco è
la casualizzazione dello spazio degli indirizzi (\emph{Address Space
  Layout Randomization}). Le librerie vengono mappate ad indirizzi di
memoria casuali, che cambiano da esecuzione ad esecuzione del
programma. In questo modo non si conoscono gli indirizzi delle
funzioni e un attacco di tipo return-to-libc naive non risulta
possibile. Tuttavia su 32bit e sotto alcune condizioni questi
indirizzi possono essere facilmente ricavati con un attacco a forza
bruta\cite{Shacham:2004}. Anche se su un'architettura 64bit un attacco
a forza bruta non risulta fattibile è tuttavia possibile che si riesca
a sfruttare qualche altro errore di programmazione per ottenere
l'indirizzo della libreria, riuscendo così a montare comunque
l'attacco.

Una generalizzazione della return-to-libc è la tecnica che prende il
nome di \emph{Return Oriented
  Programming}\cite{Shacham-2007,Roemer-2012}. In questo caso il
flusso viene redirezionato non più all'inizio di una funzione, ma
all'inizio di una piccola serie di istruzioni, già presenti in un area
di memoria del programma marcata come eseguibile, seguita da
un'istruzione ret. L'istruzione ret preleva dalla memoria dello stack
un indirizzo e ridireziona il flusso del programma a quel dato
indirizzo. Viene utilizzata al termine dell'esecuzione di una funzione
per tornare alla porzione di codice che l'ha richiamata. Potendolo
controllare l'attaccante può sistemare lo schema dello stack in modo
che una volta che il programma arrivi all'istruzione ret, sullo stack
sia presente l'indirizzo di una seconda serie dalle caratteristiche
simili alla prima. Ripetendo lo stesso discorso più volte è possibile
concatenare un numero arbitrario di queste piccole serie di istruzioni
(che prendono il nome di \emph{gadget}). In \cite{roemer-12} è stato
dimostrato che, quando sono presenti opportune tipologie di gadget,
utilizzando il ROP si possono effettuare computazioni
turing-complete. Utilizzando opportune sequenze di gadget è possibile
applicare tecniche per eludere varie misure di protezione
\cite{roglia:2009}.

Un programma può essere compilato affinché il suo codice sia
indipendente dalla posizione in cui viene mappato in memoria
(\emph{Position Indipendent Code}), in questo modo non solo le
librerie, ma anche il programma stesso può essere caricato in un
indirizzo casuale di memoria, che cambia tra un'esecuzione e
l'altra. In questo caso non è più facilmente possibile montare un
attacco che sfrutti i gadget, in quanto la posizione degli stessi non
è nota. Altre tecniche sono stati sviluppate per mitigare attacchi di
tipi ROP, anche se non risultano efficaci o l'overhead è troppo grande
\cite{Davi-2014}.

La \emph{Satisfability Modulo Theories}\cite{Barrett-14} è
un'estensione della logica del primo ordine, che viene estesa con
teorie come quella dei bitvector e degli array, rendendolo un ottimo
strumento per la modellazione dell'architettura di un elaboratore
(specialmente dei registri e della memoria). Negli ultimi tempi è
stata dedicata molta attenzione all'utilizzo di risolutori di formule
SMT con applicazione nell'ambito della sicurezza\cite{}
informatica. Riuscendo infatti a modellare le istruzioni e il contesto
di esecuzione di un programma in formule SMT, il risolutore può essere
utilizzato, ad esempio, per verificare che un programma aderisca ad
una data semantica o per estrapolarne la semantica stessa.

Anche la simulazione simbolica viene utilizzata sempre di più
nell'ambito della sicurezza. Ad esempio può essere utilizzata per
mettere in relazione dati in ingresso e comportamento del
programma. Semplificando, una volta trovato un modo per poter simulare
il programma, ad un valore concreto di ingresso viene sostituito un
valore simbolico. Ogni volta che questo valore viene manipolato viene
tenuta traccia delle operazioni effettuate su di esso. Le strutture
dati del programma possono essere espresse in funzione di quel dato
valore simbolico. È possibile poi ricavare in modo automatico una
formula SMT che rappresenti queste relazioni, in modo da poter
interrogare un risolutore per chiedere, ad esempio, per quali valori
in ingresso il programma segua o meno una branch o per quali valori
una struttura dati contenga o no un dato valore.

Nell'ambito del ROP un risolutore SMT può essere utilizzato sia per
estrapolare in maniera automatica la semantica di un gadget, sia per
ottenere le condizioni di partenza che cosentino ad un gadget di
eseguire una determinata funzione (ad esempio i valori di registro per
i quali un dato gadget scrivi in una specifica locazione di memoria un
dato valore).

Lo scopo di questo lavoro di tesi è lo sviluppo di uno strumento che
possa aiutare nel compito di generare una catena di gadget che sia in
grado di prendere controllo di un programma, premesse alcune
condizioni. In particolare lo strumento riesce a generare catene di
gadgets per eludere misure di sicurezza come stack non eseguibile,
casualizzazione dello spazio degli indirizzi e casualizzazione dello
stack applicando, in maniera automatica, tecniche come \emph{got
  patching} e \emph{return to plt} \cite{roglia:2009}. Lo strumento
utilizza per l'analisi del binario il recente framework di sviluppo
BARF \cite{Heitman-14} e supporta l'architettura intel a 32bit e a
64bit.

Descrizioni dei capitoli...

%% NX, ASRL, ROP

%% SMT, AEG, 

%% STRUMENTO

%%% Local Variables: 
%%% mode: latex
%%% TeX-master: "tesi"
%%% End: 

\chapter{Panoramica tecnica e strategie di exploit}

L'utilizzo di varie tecniche di protezione hanno reso sempre più
difficile lo sfruttamento di errori nell'esecuzione di un programma al
fine di prenderne il controllo. Questo capitolo introduce brevemente
alcune di queste tecniche e analizza con quali metodi è possibile
eluderle. Anche se alcuni concetti sono indipendenti dall'architettura
e dal sistema operativo questo capitolo tratterà esclusivamente di come
queste tecniche sono implementate in ambiente GNU/Linux, su
architettura 8086 e amd64. Nella prima parte di questo capitolo si
parlerà brevemente dei principali meccanismi con i quali il sistema
operativo, partendo dalle informazioni presenti all'interno di un file
oggetto eseguibile, crea una rappresentazione dinamica del programma,
detta \emph{immagine del processo}. Una volta creata l'immagine del
processo il controllo viene ceduto al programma che inizia la sua
esecuzione. La prossima sezione si soffermerà solo sulle parti
rilevanti per il lavoro svolto.

\section{ELF, loading}

ELF (Executable and Linkable Format) viene utilizzato per 
rappresentare tre tipi di file oggetto:

\begin{itemize}
  \item \emph{file oggetto rilocabile}: contiene dati e codice che
    vengono collegati ad altri file oggetto al fine di ottenere un
    \emph{file oggetto eseguibile} o un \emph{file oggetto condiviso}

  \item \emph{file oggetto eseguibile}: è la rappresentazione statica
    di un programma che può essere eseguito, questo file ha le
    informazioni necessarie affinché il sistema operativo crei la sua
    immagine di processo

  \item \emph{file oggetto condiviso} che contiene codice e dati
    adatti per essere collegati (nel processo di linking) in due
    contesti differenti. Può essere infatti utilizzato sia in fase di
    building insieme ad altri file oggetto rilocabile o file oggetto
    condivisi (\emph{linking statico}) per creare altri file oggetto,
    oppure può essere collegato in fase di creazione dell'immagine del
    processo di un file oggetto eseguibile (\emph{linking dinamico}).

\end{itemize}

Il formato ELF può contenere sia le informazioni necessarie per
processi relativi alla fase di creazione di diversi file oggetto a
partire da file sorgente (in particolare informazioni necessarie per
la fase di linking), sia relativamente allo scopo della creazione
dell'immagine di un processo a partire da un file oggetto
eseguibile. Anche le strutture dati all'interno dell'ELF responsabili
di questi due ``punti di vista'' sono differenti, e, a seconda della
tipologia del file oggetto alcune delle strutture dati possono non
essere presenti. Essendo più rilevante ai fini del nostro lavoro ci
concentreremo principalmente sulle strutture dati rilevanti nelle
operazioni di creazione dell'immagine del processo e nelle procedure
di dynamic linking, ovvero nel processo con il quale si rendono
fruibili all'eseguibile i metodi esposti da una libreria condivisa a
tempo di esecuzione.

\subsection{Program header table}
\label{sec:plt}

L'ELF header (che è l'unica struttura dati che ha una posizione fissa
all'interno del file) funge da una sorta di mappa, e ci consente di
localizzare tutte le altre strutture dati presenti nel file.

Tra queste vi è la \emph{Program header table}. La program header
table è un array di strutture, ognuna delle quali contiene o
informazioni su come costruire un segmento della memoria dell'immagine
di un processo o informazioni per la preparazione dell'immagine
stessa. Una entry all'interno della program header table (su
un'architettura a 32bit) è rappresentata dalla seguente struttura C:

\begin{lstlisting}[caption=program header table entry]
typedef struct {
  uint32_t   p_type;
  Elf32_Off  p_offset;
  Elf32_Addr p_vaddr;
  Elf32_Addr p_paddr;
  uint32_t   p_filesz;
  uint32_t   p_memsz;
  uint32_t   p_flags;
  uint32_t   p_align;
} Elf32_Phdr;
\end{lstlisting}

La struttura relativa ad un'architettura a 64bit è pressoché identica,
l'unica differenza sta nella posizione degli attributi.  L'attributo
\lstinline{ptype} indica il tipo di entry. Per il nostro scopo ci
limiteremo a descrivere le entry di tipo \lstinline{PT_LOAD},
\lstinline{PT_INTERPETER} e \lstinline{PT_DYNAMIC}. Le entry di tipo
\lstinline{PT_LOAD} rappresentano informazioni su un segmento da
caricare in memoria. Il campo \lstinline{p_offset} e
\lstinline{p_filesz} indicano rispettivamente l'offset e la sua
grandezza all'interno del file. \lstinline{p_vaddr} e
\lstinline{p_memsz} invece indicano rispettivamente il base address
nel quale caricare in memoria il segmento e la dimensione che avrà in
memoria (che potrebbe essere più grande rispetto a quella sul
file). \lstinline{p_flags} è molto rilevante per il nostro lavoro e
indica con quali flags il segmento andrà ad essere caricato in memoria
ovvero se il segmento sarà o meno scrivibile, leggibile e/o
eseguibile. Tuttavia nel caso il segmento contenga codice compilato
affinché risulti indipendente dal indirizzo base nel quale risiede in
memoria, il campo \lstinline{p_vaddr} potrebbe essere nullo e la
posizione del segmento in memoria potrebbe essere casualizzata.

La voce all'interno della program header table con il tipo
\lstinline{PT_INTERP} invece contiene una stringa che rappresenta il
percorso di un file oggetto eseguibile o condiviso all'interno del
filesystem, detto interprete. Il sistema operativo crea l'immagine del
processo dell'interprete, dandogli il controllo. Sarà compito poi
dell'interprete creare l'immagine del processo necessaria
all'esecuzione dell'eseguibile. Affinché questo possa accadere
l'interprete avrà accesso alle informazioni presenti nel file
principale. Normalmente l'interprete è costituito da codice
indipendente dalla posizione nella quale viene caricato, che viene
casualizzata evitando conflitti tra gli spazi di memoria utilizzati
dall'eseguibile principale e quelli dell'interprete stesso.

Quando viene creato del codice oggetto che utilizza delle librerie
dinamiche, il linker aggiunge alla program header table un elemento di
tipo \lstinline{PT_INTERPRETER} con impostato come interprete il
\emph{dynamic linker}, che si occupa di trovare le librerie necessarie
all'eseguibile, caricarle in memoria, caricare in memoria i segmenti
dell'eseguibile, risolvere le relocation verso i simboli delle
librerie, e ridare poi controllo all'eseguibile stesso. Come vedremo
la risoluzione di un simbolo potrebbe essere rimandata fin quando non
sia realmente necessario. A supporto di questi processi troviamo
alcune strutture dati, anch'esse aggiunte al file oggetto eseguibile
durante la fase di linking. Queste strutture risiedono all'interno di
segmenti che vengono caricati in memoria, e sono quindi disponibili
durante l'esecuzione del programma. Informazioni su dove trovare
queste strutture dati possono essere ricavati attraverso un'altra
struttura dati del file ELF, la \emph{Section header table}, che è un
array di strutture che descrivono, per l'appunto, le parti che
compongono il file. Tuttavia, anche se anche nel nostro lavoro di tesi
le section vengono utilizzate per ottenere informazioni sul binario
esaminato, è da notare che esse non sono necessarie per un file
oggetto di tipo eseguibile. Un file oggetto infatti può essere
caricato esclusivamente con le informazioni presenti nella program
header table (lo strumento sstrip\cite{sstrip} elimina appunto le
sezioni da un file eseguibile). Tuttavia è conveniente per scopi
illustrativi riferirsi alle strutture dati relative al processo di
dynamic linking esaminandone le rispettive sezioni:

\begin{itemize}

  \item Una sezione \lstinline{.dynamic} che contiene gli indirizzi di
    altre strutture necessarie al processo di dynamic linking e la
    lista delle librerie necessarie all'esecuzione del file oggetto
    eseguibile

  \item Una sezione \lstinline{.hash} che contiene una tabella dei
    simboli

  \item Le sezioni \lstinline{.got} e \lstinline{.plt} contengono
    rispettivamente due tabelle: \emph{la global offset table} e
    \emph{la procedure linkage table}. Nelle sezioni successive
    vedremo come queste due strutture dati vengono utilizzate dal
    dynamic linker per risolvere i simboli e le chiamate a funzioni
    presenti nelle librerie dinamiche

\end{itemize}

\subsection{Global Offset Table e Procedure Linkage Table}
\label{sec:got}

Un programma che sia indipendente dalla posizione in cui viene
caricato in memoria non può contenere al suo interno indirizzi
assoluti. Le global offset table (\lstinline{GOT}) contengono
indirizzi assoluti in un'area di memoria privata destinata ai dati,
non compromettendo quindi il fatto che il codice sia indipendente
dalla posizione (e quindi che sia condivisibile da più immagini di
processo, come nel caso di una libreria dinamica). Il linker, una
volta creata l'immagine di un processo, processa tutte le rilocazioni
(strutture dati che contengono informazioni per la risoluzione dei
simboli) di tipo \lstinline{R_386_GLOB_DAT} e per ognuna di esse
calcola l'indirizzo assoluto. Il linker conoscendo l'indirizzo di
tutti i file oggetto caricati in memoria ha tutte le informazioni
necessarie per calcolare il valore di questi indirizzi. Una volta
calcolati il linker inserisce i valori assoluti nei rispettivi
elementi della \lstinline{GOT}, permettendo all'eseguibile di
accedervi attraverso posizioni relative.

A noi interessa principalmente come la \lstinline{GOT} è coinvolti nel
momento in cui un file oggetto eseguibile esegue una chiamata ad una
funzione di una libreria dinamica, processo in cui è coinvolta anche
una seconda tabella, la procedure linkage table (\lstinline{PLT}).

La procedure linkage table redireziona chiamate a funzioni che non
siano dipendenti dalla posizione a valori assoluti non conosciuti al
momento della fase di linking di un eseguibile. Infatti in fase di
building di un eseguibile il linker fa in modo che le chiamate a
queste funzioni vengano direzionate a elementi della
\lstinline{PLT}. Anche se la \lstinline{PLT} risiede nel segmento di
memoria destinato a contenere il codice di un eseguibile, utilizza
valori nella global offset table, non compromettendo così né
l'indipendenza dalla posizione ne la condivisibilità del codice. È
compito del dynamic linker calcolare i valori degli indirizzi della
funzione ei impostarli nei relativi elementi della
\lstinline{GOT}. Nella restante parte di questa sezione andremo a
descrivere come il dynamic linker utilizza queste due tabelle per
risolvere gli indirizzi. È da notare che faremo riferimento
all'implementazione su un'architettura a 32bit. Tuttavia il concetto
principale resta pressoché invariato anche per un'architettura a
64bit. La \lstinline{PLT} è strutturata come nel listato
\ref{cod:plt}.

\begin{lstlisting}[caption=Procedure linkage table, label=cod:plt]
.PLT0:
    pushl   got_plus_4
    jmp     *got_plus_8
    nop;    nop
    nop;    nop
.PLT1:
    jmp     *name1_in_GOT
    pushl   $offset
    jmp     .PLT0@PC
.PLT2:
    jmp     *name2_in_GOT
    pushl   $offset
    jmp     .PLT0@PC
    ...
    ...
\end{lstlisting}

\begin{itemize}
\item Quando viene creata l'immagine del processo i primi due valori
  nella GOT assumono valori particolari (spiegati di seguito)
\item Quando l'eseguibile esegue una chiamata di una funzione in una
  libreria dinamica il flusso del programma viene indirizzato nella
  \lstinline{PLT}. Nel nostro esempio viene chiamata la funzione
  \lstinline{name1}, che avrà come indirizzo di destinazione
  l'istruzione marcata dall'etichetta \lstinline{.PLT1}
\item Il programma a questo punto esegue un \lstinline{jmp}
  all'indirizzo contenuto nell'elemento nella GOT relativo a
  \lstinline{name1}. Al momento di caricamento del programma, a
  eccezione di casi particolari come descritto di seguito, questo
  elemento è impostato con il valore dell'istruzione successiva al
  \lstinline{jmp} stesso, cioè all'istruzione alla linea 7 del listato
  \ref{cod:plt}
\item Il codice a questo punto salva sullo stack l'offset all'interno
  della tabella delle rilocazioni che permette di individuare la
  rilocazione relativa al simbolo \lstinline{name1}. La rilocazione
  permette di ricavare sia l'elemento della \lstinline{GOT} relativo a
  quel simbolo che il nome del simbolo stesso, fornendo le
  informazioni necessarie al dynamic linker per capire quale simbolo è
  stato chiamato e qual'è l'elemento modificare nella \lstinline{GOT}.
\item A questo punto il programma salta al primo elemento della
  \lstinline{PLT} e, dopo aver salvato sullo stack il secondo elemento
  della \lstinline{GOT}, salta all'indirizzo contenuto nel terzo
  elemento della got, che dà il controllo al dynamic linker
\item Il dynamic linker esamina lo stack, controlla quale simbolo è
  stato chiamato, calcola l'indirizzo e imposta il valore
  nell'elemento relativo nella \lstinline{GOT}. In questo modo una
  seconda chiamata a \lstinline{name1} non causerà una seconda
  chiamata al linker ma salterà direttamente all'indirizzo corretto
\end{itemize}

Quest'approccio nel quale gli indirizzi delle funzioni vengono risolti
solo al momento in cui vengono chiamate si chiama \emph{lazy
  binding}. Se il programma viene eseguito con la variabile d'ambiente
\lstinline{LD_BIND_NOW} impostata gli indirizzi vengono tutti
calcolati e scritti nella \lstinline{GOT} al momento del caricamento
dell'eseguibile. Questo comportamento avviene anche quando si compila
un eseguibile con \lstinline{RELRO}. Inoltre, in questo caso, una
volta modificata la \lstinline{GOT} viene resa non più scrivibile, in
modo da mitigare attacchi di tipo got patching.


\section{Return oriented programming}

Per mitigare classici attacchi di tipo stack-smashing sono state
elaborate diverse tecniche per precludere la possibilità di eseguire
il codice iniettato da un attaccante. La prima di questa tecniche è
stata implementata in una patch di Solar Designer \cite{stackpatch},
che modifica lo schema della memoria dell'immagine di un processo al
fine di rendere le istruzioni presenti nello stack non eseguibili. Dato
che nella maggior parte dei casi l'attaccante utilizzava lo stack come
locazione del codice iniettato questa patch rendeva innocui molti di
questi attacchi. Una tecnica più completa, chiamata ``$W \oplus X$'',
assicura invece che non ci sia una pagina di memoria all'interno del
processo che sia scrivibile ed eseguibile nello stesso momento. Per
eludere questa tecnica un attaccante è costretto a non utilizzare più
codice da lui stesso iniettato ma codice già presente nell'immagine
del processo bersaglio (il primo a suggerire quest'approccio è stato
Solar Designer \cite{solar-return-to-libc}). Dato che la libreria C
standard è praticamente utilizzata in quasi tutti i programmi Unix e
contiene funzioni utili ad un attaccante (come \lstinline{system} o
\lstinline{execve}) il codice utilizzato solitamente è proprio quello
della libc, da cui il nome con cui vengono catalogati questo tipo di
attacchi: \emph{return-to-libc}. Tra le misure per mitigare attacchi
di questo tipo fu ipotizzato la rimozione di alcune funzioni dalle
libc al fine di rendere questo tipo di attacchi meno efficace. Shacham
in \cite{Shacham-2007} ha evidenziato come questo tipo di protezione
fosse in realtà inefficace introducendo per la prima volta la tecnica
da lui battezzata come \emph{Return Oriented Programming} (ROP), in
cui venivano usate piccole serie di istruzioni di codice già presente
nell'immagine del processo e non intere funzioni come in attacchi di
tipo return-to-libc. Anche se, in questa prima applicazione, la
tecnica estrapolava queste serie di istruzioni dalla libc, in
\cite{schwartz-2011} è stato evidenziato come questa tecnica restasse valida
nonostante le istruzioni venissero direttamente prelevate dal testo
dell'eseguibile e non dalle sue librerie. In questo modo si può
utilizzare la \lstinline{ROP} per eludere tutta un'altra serie di
protezioni che si basano sulla casualizzazione della posizione di aree
di memoria chiave all'interno dell'immagine del processo di un
eseguibile (in particolar modo delle aree che contengono le librerie
condivise e di quella riservata allo stack) detta appunto \emph{Addres
  Space Layout Randomization}(ASLR). Nell'applicazione della
\lstinline{ROP} vengono individuate piccole serie di istruzioni con
delle particolari caratteristiche, chiamate \emph{gadget}. Arrangiando
minuziosamente lo stack si possono eseguire gadget uno dietro l'altro,
cicli o computazioni arbitrarie. Un punto importante è che, almeno per
quanto riguarda l'architettura intel, possono essere individuate
all'interno di un file oggetto molte sequenze di istruzioni che non
sono ``intenzionali'', ovvero inserite dal compilatore come le
istruzioni relative alla traduzione di codice sorgente. Infatti,
essendo la codifica delle istruzioni in linguaggio macchina molto
densa e non allineata, se si iniziano ad interpretare le istruzioni
partendo dal centro di un'altra istruzione, vi è molta probabilità di
decodificare una sequenza di istruzioni alternativa e valida. Questo è
dovuto alla natura stessa della codifica, quello che Shacham chiama
geometria.

\subsection{Gadget}

In questa sezione analizzeremo più in dettaglio i gadget e le loro
caratteristiche. Bisogna innanzitutto considerare che essendo i gadget
composti da una piccola serie di istruzioni questo tipo di attacchi
agisce ad un livello più basso rispetto a quello degli attacchi di
tipo return to libc. I gadget possono essere visti come delle
istruzioni di basso livello di uno strano calcolatore. Queste
istruzioni non vengono concatenate in maniera standard ma arrangiando
minuziosamente i valori presenti nello stack. I gadget sono terminati
dall'istruzione \lstinline{ret} e nelle istruzioni che lo compongono
non sono contenuti salti e altre istruzioni che deviano in qualche
modo il flusso del programma (tecniche che utilizzano anche questo
tipo di gadget sono state sviluppate, come in
\cite{JOP}). Aggiuntando lo stack in modo che nel momento in cui il
flusso del programma arrivi all'istruzione \lstinline{ret} di un
gadget il registro dello stack punti alla zona di memoria che contiene
l'indirizzo del prossimo gadget si riesce a concatenare un gadget al
successivo. I gadget, soli o in combinazione, possono svolgere diverse
funzioni. Per quanto riguarda, ad esempio, le operazioni di lettura o
scrittura possiamo notare che:

\begin{itemize}
\item gadget del tipo \lstinline{pop REG;ret} ci permettono di
caricare una costante in un registro

\item gadget del tipo \lstinline{mov reg1 ,[reg2 + imm];ret} ci
  permettono di leggere un valore dalla memoria (il valore di
  \lstinilne{reg2} può essere impostato con un gadget del punto
  precedente) e scriverlo su un registro

\item gadget del tipo \lstinline{mov [reg1 + imm], reg2} ci permettono
  di scrivere in una posizione di memoria (i valori di
  \lstinline{reg1} e \lstinline{reg2} possono essere impostati con
  gadget che ci consentono di caricare una costante di un registro)
\end{itemize}

In base all'eseguibile che stiamo considerando però potrebbero non
essere disponibili gadget esattamente come questi.  In particolare
potrebbero presentarsi altre istruzioni tra quella che a noi
effettivamente interessa (perché svolge la funzione a noi ``utile'') e
l'istruzione \lstinline{ret}. A volte modificando opportunamente lo
stack si può fare in modo che gli effetti di queste istruzioni non
incidano sulla nostra computazione. Ovviamente altri gadget, o
sequenze di gadget, possono svolgere funzioni equivalenti e, il fatto
che una categoria di gadget non è disponibile non implica che la
stessa funzione non possa essere svolta utilizzando gadget e/o
strategie differenti. Ad esempio se non abbiamo a disposizione un
gadget che ci permetta di impostare il registro \lstinline{eax}
tramite \lstinline{pop eax;ret} un'opportuna combinazione di
\lstinline{mov eax, 0xFFFFFFFF;ret} e \lstinline{inc eax;ret} ci
permette (non considerando restrizioni sulla lunghezza del payload) di
impostare valori arbitrari in \lstinline{eax}. Estrapolare, in maniera
automatica, sequenze e combinazioni di gadget che ci permettano di
raggiungere condizioni desiderate o gestire gli effetti secondari
delle istruzioni ``superflue'' è uno dei maggiori problemi di questo
lavoro di tesi.

Le operazioni che l'architettura ci permette di eseguire non si
limitano ad operazioni di lettura e scrittura. Infatti possiamo
eseguire operazioni aritmetiche, operazioni logiche e utilizzare
sequenze di gadget che ci permettono di controllare il flusso del
programma. Come considerato nel paragrafo precedente spesso i metodi
per eseguire queste operazioni non sono diretti, ma è necessario
considerare gli effetti secondari e operare solo con i gadget a
disposizione. Ad esempio possiamo avere a disposizione solo operazioni
che operano su dimensioni di un byte, e per eseguire operazioni su
dimensioni maggiori è necessario concatenare più volte la stessa
operazione eseguendola un byte per volta, oppure un'operazione
potrebbe essere eseguita utilizzando una serie di più operazioni ma
che nel complesso sia equivalenti all'operazione voluta.

\subsection{Address Space Layout Randomization}

L'\emph{'Address Space Layout Randomization} (ASLR) è una tecnica di
protezione che consiste nel caricare in posizione casuale regioni
della memoria del programma. Questo è possibile perché un'istruzione
può riferirsi ad una locazione di memoria tramite la distanza tra il
proprio indirizzo (contenuta nel registro instruction pointer) e la
locazione. Non utilizzando quindi più indirizzi assoluti il codice può
risultare indipendente dalla posizione di memoria in cui è caricato a
patto che che gli offset relativi restino invariati (cioè che il
codice non venga ``mischiato''). Il dynamic linker e alcune strutture
dati vengono utilizzate per far funzionare porzioni di codice
indipendenti anche dagli offset relativi (come nel caso del testo
dell'eseguibile e le librerie dinamiche), utilizzando un livello di
indirezione (vedi sez. \ref{sec:plt} e \ref{sec:got}).

L'utilizzo dell'ASLR porta due complicazioni principali nell'utilizzo
di tecniche basate sulla ROP. Prima di tutto non conoscendo
l'indirizzo di una porzione di codice non si possono conoscere neanche
gli indirizzi dei gadget contenuti all'interno. Questo riduce
notevolmente la quantità di gadget a disposizione, e, a seconda di
quanto codice venga casualizzato può rendere impossibile un attacco
che utilizzi la ROP. Infatti anche se solitamente, per questioni di
efficienza, solo le librerie vengono casualizzate, è possibile
compilare un'eseguibile in modo che una volta eseguito tutte le aree
che contengono codice vengano casualizzate (\emph{Position Indipendent
  Executable} (PIE)). In quest'ultimo scenario, a meno di non ottenere
informazioni sulla posizione di memoria in altri modi (ad esempio
tramite altre vulnerabilità che espongano indirizzi di strutture
dell'eseguibile), l'applicazione della ROP non risulta possibile.

Inoltre a venire casualizzate non sono solo porzioni che contengono
codice, ma anche la porzione di memoria riservata allo stack. Non
conoscendo la posizione dello stack risulta molto più difficile creare
sequenze di gadget che possano eseguire operazioni di controllo del
flusso. Una possibile soluzione potrebbe essere quella di iniettare
gli indirizzi dei gadget in un'altra posizione (nota) e modificare il
valore del registro dello stack perché punti a quella zona di memoria,
ovvero creare un \emph{Fake stack frame}. In più non avendo più
accesso a librerie dinamiche come la libc, difficilmente si riesce a
trovare nella porzione di codice non casualizzato istruzioni che ci
consentano di lanciare una syscall (solitamente il programma fa
affidamento alla libreria C per eseguirle e quindi dovrebbe comparire
tra le sequenze di istruzioni non ``intenzionali''). Tuttavia in
\cite{schwartz-2011} è stato dimostrato come anche utilizzando gadget che si
incontrano in porzioni di codice relativamente piccole è possibile
eseguire computazioni arbitrarie, addirittura concatenando questi
gadget in maniera automatica. In più come vedremo nelle prossime
sottosezioni è possibile, sotto alcune condizioni, ricavare gli
indirizzi di funzioni presenti nella libc partendo dalle strutture
dati coinvolte nelle operazioni di dynamic linking. Se le giuste
tipologie di gadget sono disponibili è possibile quindi eseguire
comandi arbitrari (come lanciare una shell) anche in contesti in cui
stack e librerie sono casuallizati.


\subsection{Return-to-plt, GOT dereferencing e GOT patching}
\label{sec:expl}

In questa sottosezione presupponiamo che il codice del testo
dell'eseguibile non sia stato casualizzato. Anche se le librerie sono
casualizzate se un'eseguibile utilizza una funzione presente in una
libreria dinamica il linker, in fase di building, inserisce un
elemento della PLT relativo a quella funzione (vedi
sez. \ref{sec:plt}). Se la posizione in cui viene caricato il codice è
nota lo è anche quella della PLT e quindi l'indirizzo
dell'elemento. Questo ci permette di poter ``utilizzare'' nella nostra
catena le funzioni che utilizza l'eseguibile bersaglio, dirottando il
flusso all'indirizzo del relativo elemento nella plt, da qui il nome
\emph{return-to-plt}. Una volta utilizzata una funzione il valore
dell'indirizzo assoluto di quella funzione si verrà a trovare
nell'elemento relativo nella GOT (vedi sez. \ref{sec:got}). Come
descritto in \cite{roglia:2009} questo ci permette di utilizzare due
tecniche di attacco particolari: \emph{GOT derefercing} e \emph{GOT
  patching}. Tutte e due le tecniche utilizzano l'indirizzo assoluto
di una funzione contenuto nell'entry relativa nella GOT per calcolare
l'indirizzo assoluto di una funzione della libc desiderata, di fatto
eludendo l'ASRL. Nella prima tecnica il valore viene letto, modificato
tramite un'operazione aritmetica e viene poi eseguito un
\lstinline{jmp} a questo indirizzo. Un esempio di gadget che
consentono questo tipo di attacco sono: \lstinline{add eax, [ebx +
    off]; ret} e \lstinline{jmp [eax]}. La seconda tecnica invece
modifica l'elemento della GOT relativo in loco, utilizzando poi un
return-to-plt per richiamare la funzione voluta. Un esempio di gadget
che può essere utilizzato per modificare la funzione in loco è:

\lstinline{adc byte ptr [esi + 0x5f], bl ; pop ebp ; ret}

Infatti concatenando quattro istruzioni del genere insieme ad un
opportuno gadget che permette di impostare il registro esi è possibile
aggiungere un offset arbitrario in una locazione di memoria
arbitraria. È da notare che questa tecnica non funziona nel caso la
GOT sia in un'area di memoria non scrivibile (ad esempio se
l'eseguibile è stato compilato con l'opzione RELRO) ma i gadget
utilizzati in questa seconda tecnica sono di un tipo più comune
rispetto a quelli utilizzati nella GOT dereferencing.


%%% Local Variables: 
%%% mode: latex
%%% TeX-master: "tesi"
%%% End: 

\chapter{Architettura di Dropper}
\label{cap:architettura}

In questo capitolo si fa una descrizione dettagliata dei requisiti e
delle funzionalità presenti nell'architettura designata per lo
sviluppo di Dropper.

L'obiettivo di Dropper è quello di generare, analizzando un file
oggetto eseguibile, il payload corrispondente ad una catena di gadget
che se iniettata nello stack dell'immagine del processo associato a
quell'eseguibile, consente di prenderne il controllo, eseguendo un
comando arbitrario sulla macchina. Lo strumento applica tecniche note
per eludere misure di protezioni esistenti come la casualizzazione
degli indirizzi (\lstinline{ASRL}), stack e aree dati non eseguibili
(``W $\oplus$ X'') o la read-only relocation, automatizzando il più
possibile la scelta della strategia appropriata. 

Dropper analizza il binario utilizzando alcune delle funzionalità del
recente framework per l'analisi dei binari \lstinline{BARF}
\cite{Heitman-14}. Il linguaggio utilizzato per la scrittura di
Dropper è python e, nella sua prima versione, supporta le architetture
\lstinline{x86} e \lstinline{x86-64} su piattaforme GNU/Linux (il
formato degli eseguibili supportato è ELF). Tuttavia come tratteremo
meglio nel corso di questo capitolo, queste restrizioni possono essere
facilmente superate nelle future versioni grazie all'astrazione data
dall'utilizzio di un linguaggio intermedio per rappresentare le
istruzioni.

Una tipica sessione di Dropper può essere grossolanamente divisa in
quattro fasi:

\begin{enumerate}
\item Analisi del binario 
\item Estrapolazione e analisi dei gadgets
\item Scelta della strategia da utilizzare
\item Creazione del payload
\end{enumerate}

Nella prima fase, quella in cui si analizzano le informazioni presenti
nel file oggetto eseguibile esaminato, vengono ricavate le
informazioni sulle misure di protezione attualmente in uso sul file
oggetto eseguibile, le funzioni utilizzate e le librerie dinamiche
importate. Queste informazioni verranno utilizzate specialmente nella
terza fase.

Dropper, nella seconda fase, ricerca all'interno del file binario i
gadgets disponibili. I gadget trovati vegono tradotti, dal framework
BARF, in un linguaggio intermedio: REIL
\cite{reil-dullien(csw09)}. Effettuare le analisi su un linguaggio
intermedio anziché sulle istruzioni dell'architettura vera e propria
porta una serie di vantaggi:

\begin{itemize}
 \item le operazioni di analisi vengono estratte e rese
   indipendenti dalla specifica architettura. In questo modo è
   possibile aggiungere il supporto di altre architetture allo
   strumento in modo molto più semplice, operazione che si riduce a
   sviluppare un traduttore da questa archiettura al linguaggio
   intermedio

 \item REIL ha un insieme di istruzioni molto ridotto (solo
   quattordici istruzioni), questo facilità notevolmente lo sviluppo
   di strumenti di analisi

  \item le istruzioni di REIL non presentano nessun effetto
  secondario implicito, come invece avviene per la maggior parte delle
  istruzioni delle architetture reali

\end{itemize}

Una volta trovati e tradotti in un linguaggio intermedio è necessario
estrapolare la semantica dei gadget. Questo permetterà nella fase
successiva l'assemblaggio di piccole catene che consentano di eseguire
operazioni particolari (come modificare il valore dello stack,
chiamare una funzione, settare un registro, scrivere in memoria,
operazioni aritmetiche, leggere dalla memoria, ecc). Dropper
utilizzerà nella fase di creazione del payload queste piccole catene
per assemblare catene via via più lunghe che eseguono operazioni più
complesse e articolate. L'architettura prevede che la generazione di
queste sequenze avvenga sia in maniera programmatica, in cui la
procedura per la sua costruzione è codificata in un algoritmo, sia in
maniera dichiarativa, ad esempio utilizzando linguaggi di planning
come \lstinline{PDDL}\cite{pddl-97}. Un analisi sui vantaggi e sui
problemi che si incontrano in una codifica del genere è riportata
nella sezione \ref{sec:pddl}.

Nella terza fase, per scegliere la tecnica di exploit appropriata,
Dropper si basa sia su alcune informazioni che fornisce l'utente in
merito all'errore che si sta sfruttando (ad esempio se può controllare
un file descriptor, se ci sono limiti nella dimensione del payload o
nei caratteri ammessi) sia su quelle ricavate dalla prima fase di
analisi dell'eseguibile. Dev'essere semplice poter aggiungere nuove
strategie.

L'ultima fase si occupa di assemblare gli indirizzi dei gadget e
valori concreti per la generazione di un payload che, una volta
iniettato sullo stack, comporti l'esecuzione delle varie catene
assemblate nelle fasi precedenti.

%% Una volta codificato il
%% problema in questi termini si può chiedere ad un planner, come se
%% fosse una sorta di oracolo, in che sequenza collegare le varie azioni
%% (cioè le varie sequenze elementari di gadget) per ottenere quello
%% stato. La rappresentazione delle informazioni sulla semantica dei
%% gadget è espressa in questo modo in forma esplicita, e non più
%% codificata in un algoritmo.

L'architettura di Dropper si compone di tre moduli logici:

\begin{itemize}

  \item Core: È il modulo che si occupa di 
    \begin{inparaenum}[a)] 
      \item leggere il binario,
      \item coordinare le operazioni dei restanti moduli,
      \item interagire con l'utente per ottenere informazioni
        aggiuntive sul binario e/o sullo scenario nel quale dev'essere
        prodotta la catena e 
      \item scegliere la strategia di exploit più appropriata
    \end{inparaenum}
  \item Gadgets: È il modulo che si occupa dell'estrapolazione,
    dell'analisi, della scelta e dell'assemblaggio dei gadget da utilizzare nella
    composizione della catena e dei vari sui frammenti

  \item Payload: È il modulo che si occupa di trasformare una
    sequenza ordinata di gadget in un payload vero e proprio da
    iniettare sullo stack, gestendo il posizionamento corretto degli
    indirizzi e dei valori utilizzati dai vari gadget

\end{itemize}

\section{Modulo Core}

È il modulo che si occupa di leggere il file binario, interagire con
l'utente, scegliere le varie strategie da adottare e di coordinare le
azioni degli altri moduli. 

Dropper parsa l'ELF e ricava la lista delle funzioni che il file
oggetto eseguibile utilizza, i relativi elementi nella \emph{Procedure
  Linkage Table} (vedi \ref{sec:plt}), le aree di memoria all'interno
dell'immagine del processo dove è possibile scrivere e da dove è
possibile leggere e le misure di protezione utilizzate dal file
binario. Per il parsing dell'ELF Dropper utilizza una libreria
apposita, pyelftools (\cite{pyelftools}).

L'utente poi, interagendo con il modulo, fornisce informazioni
aggiuntive necessarie alla generazione della catena e linee guida
sulla strategia da utilizzare. Tra le informazioni che può fornire
l'utente troviamo:

\begin{itemize}
  \item La possibilità da parte dell'utente di controllare un file
    descriptor
  \item I file oggetto condivisi utilizzati dall'eseguibile sulla
    macchina dove si intende far eseguire la catena (per la
    ricavazione degli offset da utilizzare nelle tecniche di
    sovrascrizione della GOT vedi sez. \ref{sec:expl})
  \item Eventuali caratteri che non possono comparire all'interno del
    payload
  \item Eventuali limiti sulla lunghezza del payload
  \item Il comando da eseguire
\end{itemize}

Con tutte queste informazioni il modulo può quindi procedere nello
stabilire quale strategia utilizzare e richiedere al modulo
gadgetstools le sequenze di gadget necessarie. Lo schema principale
delle varie strategie, si può dividere in due fasi: 

\begin{inparaenum}[1)]
\item preparare gli argomenti per eseguire una chiamata ad una
  funzione di libreria che ci consenta di eseguire un comando
  arbitrario (\lstinline{execve} o \lstinline{system}) e
\item eseguire quella
funzione.
\end{inparaenum}

Per scrivere in memoria è possibile utilizzare diverse strategie,
Dropper deve scegliere la più adatta o quella specificata
dall'utente. Il modulo deve essere facilmente estendibile per
aggiungere nuove strategie. Alcune delle strategie più efficenti per
scrivere in memoria sono:

\begin{itemize}

\item Nel caso sia possibile inserire arbitrariamente dei valori in un
  file descriptor (ad esempio se l'utente può inserire uno stream di
  dati nell'input del programma) allora è comodo utilizzare la
  funzione \lstinline{read} della \lstinline{libc} per copiare questi
  dati nell'area di memoria di destinazione. Dropper fornisce i valori
  da ``inserire'' nello stream dati del file descriptor. L'attaccante
  deve fare in modo che quando la funzione \lstinline{read} venga
  richiamata siano letti dal file descriptor proprio quei dati.

\item Utilizzo di catene di gadget che consentano di scrivere valori
  costanti in memoria. Per poter utilizzare questa tecnica devono
  essere disponibili i gadget delle categorie necessarie, in più i
  valori da caricare in memoria non devono presentare caratteri che
  non possono essere presenti nel payload. In questa tecnica i dati
  non vengono scritti tutti in una volta ma a chunk di alcuni bytes
  per volta (a seconda dell'architettura o del tipo di gadget). A
  seconda del tipo di gadget per ognuna di queste iterazioni è
  necessario inserire nel payload un certo numero di bytes di
  ``padding''. Per questi motivi l'utilizzo di questa tecnica tende a
  generare payload piuttosto lunghi.

\item Utilizzo delle funzioni \lstinline{strcpy} o \lstinline{memcpy}
  per copiare byte presenti in posizioni note all'interno
  dell'eseguibile nella posizione desiderata \cite{longld:2010}.
  Questa tecnica, copiando verosimilmente solo un byte per volta tende
  anch'essa a generare payload abbastanza lunghi ma ha il grande
  vantaggio di poter copiare anche valori che non possono essere
  presenti nel payload. È necessario che i valori dei byte che voglio
  copiare siano già presenti nell'eseguibile, condizione che, a meno
  di file oggetto eseguibili molto piccoli, risulta abbastanza comune

\item Nel caso si riesca ad ottenere in qualche modo (ad esempio
  attraverso una seconda vulnerabilità) la conoscenza dell'indirizzo
  dello stack i valori possono essere inseriti direttamente insieme al
  payload

\end{itemize}

Una volta scritti gli argomenti per la funzione da richiamare in
memoria è necessario chiamare la funzione stessa. La funzione deve
permetterci di eseguire un comando arbitrario. All'interno della
\lstinline{libc} sono presenti due funzioni con queste
caratteristiche: \lstinline{execve} e \lstinline{system}. Un ritorno
diretto alla funzione solitamente non può avvenire, in quanto il
valore del suo indirizzo in memoria è casualizzato. Tuttavia, se una
di queste due funzioni è utilizzata all'interno del programma, basterà
ritornare al suo elemento nella Program Linkage Table, che ha una
posizione nota e fissa all'interno della memoria dell'immagine del
processo. Se così non fosse un'alternativa è quella di adoperare
tecniche per ricavare l'indirizzo della funzione partendo
dall'indirizzo di un'altra funzione della stessa libreria che invece
viene utilnizzata dal programma. Alcune considerazioni su queste
tecniche, descritte nella sezione \ref{sec:expl}, che riguardano
l'architettura di Dropper sono:

\begin{itemize}
  \item Devono essere noti gli offset tra le funzioni all'interno del
    file oggetto condiviso della libreria utilizzato sulla macchina
    bersaglio. L'utente può fornire a Dropper un file di questo tipo
    dal quale verranno ricavati i relativi offset. È da notare che,
    variando in un range relativamente piccolo, non dovrebbe essere
    impossibile, se le condizioni lo consentono, montare un attacco di
    tipo forza bruta per ricavare questi offset

  \item Come visto in \ref{sec:expl} per poter utilizzare la tecnica
    di \lstinline{got patching} il binario deve presentare una
    sequenza di gadgets che ci consenta di eseguire un'operazione di
    somma (o un'operazione che ci consenta di simulare una somma, come
    una sottrazione) tra un offset di tre/quattro byte e un'area di
    memoria arbitraria (l'area dove è memorizzato l'indirizzo della
    funzione usata come riferimento, solitamente nella GOT). Molto
    spesso però abbiamo gadget che ci consentono di addizionare solo
    un byte per volta. Questo è un problema perché in un'operazione di
    somma byte per byte dev'essere considerato un eventuale valore di
    riporto tra un byte e il successivo. Un modo per risolvere questo
    problema è utilizzare gadget che utilizzino un'istruzione di tipo
    \lstinline{adc} (add with carry), facendo attenzione che gli
    effetti secondari degli altri gadget non vadano a modificare il
    carry flag. In realtà anche non avendo a disposizione questo tipo
    di gadget, per via dell'allineamento in memoria, gli ultimi bit
    dell'indirizzo di una funzione non variano tra un'esecuzione e la
    successiva. Con questa parziale conoscenza del valore iniziale si
    può comunque creare una catena con un alta percentuale di successo
    come descritto in \ref{sec:arithmeticstore}
\end{itemize}

È interessante notare come, anche se vi è una guida nelle operazioni
da seguire basata sulla strategia scelta, la sequenza di operazioni
che conducono allo stato del sistema desiderato sono molteplici. Ad
esempio l'operazione aritmetica per il calcolo dell'indirizzo della
funzione bersaglio potrebbe avvenire in memoria, oppure tra due
registri e poi salvata in memoria. Qui si notano i vantaggi che
potrebbe avere un approccio che modelli la situazione in un problema
di planning rispetto ad un approccio di tipo programmatico. Infatti,
con un approccio programmatico, tutte le combinazioni vanno codificate
nell'algoritmo, una procedura la cui completezza è più difficile da
dimostrare e che è molto più soggetta ad errori.

\section{Modulo Gadgets}

Il modulo Gadgets si occupa di ricavare i gadgets presenti nel file
oggetto eseguibile, di estrapolarne la semantica e della creazione di
sequenze di gadget che se eseguite assolvono funzioni particolari
(come scrivere in memoria, settare un registro, ecc.). Questo modulo
interagisce con il modulo Payload per trasformare sequenze ordinate di
gadget in un payload vero e proprio da poter iniettare sullo stack ed
espone i metodi per la creazione delle catene che vengono usati dal
modulo Core.

Per la ricerca dei gadget all'interno del file oggetto eseguibile
viene utilizzata la classe \lstinline{GadgetFinder} del framework
BARF, che implementa al suo interno l'algoritmo
galileo\cite{Shacham-2007}.

%TODO descrivere galileo ?
%TODO utilizzare il nostro ?

Una volta disponibile l'insieme dei gadget presenti nel file oggetto
eseguibile bisogna, per ognuno di essi, estrapolarne la semantica. Per
estrapolarne la semantica si intende estrapolare in maniera automatica
gli effetti che una serie di istruzioni hanno sul contesto il quale,
semplificando, è rappresentato dalla memoria e dai registri della
macchina sulla quale saranno eseguite le istruzioni. Per questo tipo
di analisi si utilizzano due strumenti principali: la simulazione e la
traduzione delle istruzioni in formule logiche. Sono state considerate
due alternative principali per la modellazione delle istruzioni in
formule logiche: \emph{Satisfaible Modulo
  Theories}(SMT)\cite{Barrett-14} e \emph{Answer Setp
  Programming}(ASP)\cite{Ianni-09}. Per questa prima versione è stata
scelta la modellazione in formule SMT, un razionale per questa scelta
viene fornito in \ref{sec:smt_vs_asp}.

 Un'analisi completa della semantica presenta alcuni limiti dovuti
alla difficoltà di una modellizzazione completa del sistema nonché
alla conoscenza parziale del contesto (memoria e registri) reale in
cui si troveranno ad essere eseguite le istruzioni. Ad esempio
utilizzando l'attuale traduzione delle istruzioni in formule SMT, è
difficile porre al risolutore domande che riguardino il numero di
letture effettuate, o domandare se un'area di memoria viene mai
letta. Per quanto detto l'estrapolazione della semantica di un gadget
avviene in tre fasi descritte nelle prossime sottosezioni:
\begin{inparaenum}[a)]
\item emulazione e catalogazione, 
\item verifica e validazione e 
\item verifica degli effetti secondari.
\end{inparaenum}

\subsection{Emulazione e catalogazione}
La classe \lstinline{GadgetClassifier} del framework BARF emula il
gadget tenendo traccia delle operazioni sui registri e sulla
memoria. Con questi dati si verifica se il gadget appartiene ad una o
più categorie indicate in \cite{schwartz-2011}.
%TODO vanno elencate le categorie?  
Inoltre vengono identificati quali registri vengono utilizzati come
``operandi'' della funzione alla quale il gadget viene
associato. Basandosi su una simulazione, e quindi sull'analisi degli
effetti di solo un sottoinsieme dei possibili valori in ingresso,
questa catalogazione deve essere poi verificata con un'analisi
semantica vera e propria.

Le simulazioni vengono effettuate utilizzando dati concreti ma una
simulazione simbolica, funzionalità ancora non disponibile all'interno
del framework BARF, consentirebbe una catalogazione più accurata
esplicitando le relazioni tra i valori che i registri hanno all'inizio
e alla fine della simulazione o tra i valori iniziali dei registri e
gli accessi alla memoria.

\subsection{Verifica e validazione}
\label{s:verifica_validazione}

Si deve poi verificare che, anche se il gadget esegua la funzione
della categoria alla quale viene assegnato, possa essere utilizzato
per i valori di tutto il ``dominio'' di questa funzione, cioè che non
ci siano restrizioni sugli operandi del gadget. Per esempio, un gadget
catalogato come \lstinline{MemoryStore} ma che utilizzi lo stesso
registro come operando di destinazione e operando sorgente non può
essere utilizzato per scrivere in una locazione arbitraria di memoria
un valore arbitrario in quanto il valore da scrivere e l'indirizzo di
memoria sono dipendenti tra di loro.

Bisogna verificare altresì che l'assegnamento dei valori appropriati
ai registri ``operandi'' sia condizione sufficiente perchè il gadget
si comporti aderendo alla semantica della categoria in cui è stato
catalogato. Sempre continuando con l'esempio della scrittura in
memoria, questo vuol dire essere sicuri che, se i registri utilizzati
come operandi sono impostati con i valori appropriati, la scrittura in
memoria del valore voluto sia garantita per qualsiasi altra
combinazione iniziale dei restanti registri.


Le verifiche che ci consentono di verificare le due proprietà appena
discusse sono dipendenti dalla tipologia di gadget, ma possono essere
tutte ricondotte alla seguente strategia generale: 
\begin{enumerate}
\item si scelgono casualmente gli effetti che deve avere la funzione
  sul contesto
\item si traduce il gadget in formule logiche e si impone che il
  contesto finale rifletta gli effetti scelti, nel caso l'insieme
  delle formule ottenuto non sia soddisfacibile il gadget può essere
  scartato
\item si richiede al risolutore logico il modello che rende vera la
  formula al punto precedente e da questo modello si ricavano i valori
  iniziali dei registri operandi
\item si ricrea la rappresentazione del gadget in formule logiche ma
  forzando i valori dei registri iniziali ai valori ottenuti al punto
  precedente e imponendo come condizione finale il complementare degli
  effetti voluti
\item si verica che il modello del punto precedente non sia
  soddisfacibile
\end{enumerate}

Ad esempio se stiamo verificando un \lstinline{MemoryStore} si sceglie
sia una locazione casuale di memoria sia un valore casuale da scrivere
in questa locazione. Si traduce il gadget in formule logiche e si
impone che nel contesto finale che la locazione di memoria scelta
cambi al valore casuale scelto. Si chiede a questo punto al risolutore
se il sistema di formule ammette una soluzione. Se il sistema non
ammette soluzione il gadget può essere scartato, in quanto non essendo
usabile per i valori perde la sua generalità. Nel caso invece il
sistema ammetta soluzione vengono richiesti al risolutore i valori
iniziali, all'interno del modello trovato, dei registri ``operandi''
della funzione, quelli utilizzati per impostare indirizzo di
destinazione e valore da scrivere. Si impongono poi al risolutore
nuovamente le formule logiche che rappresentano il gadget ma vengono
``forzati'' come valori iniziali per i registri operandi i valori
ottenuti dalla prima verifica e, inoltre, si impone che il valore
finale sia diverso da quello aspettato. La non soddisfacibilità di
questa formula ci dà la certezza che per i valori iniziali dei
registri operandi ottenuti dalla prima verifica, lo stato finale del
contesto rispecchi i valori attesi. Questa verifica esclude che ci
siano effetti secondari che possono interferire con gli effetti della
nostra funzione.

È da notare che queste proprietà sono verificate solo per dei valori
casuali e dovrebbero essere verificate per tutti i valori. Questo non
viene fatto sia per motivi di efficienza sia per non includere
quantificatori universali nelle formule, complicando di molto il
lavoro del risolutore. Questo problema può essere risolto sia
ripetendo queste verifiche al momento dell'utilizzo reale del gadget
con i valori concretamente utilizzati sia esprimendo gli effetti in
relazione ai valori iniziali, ad esempio utilizzando alla simulazione
simbolica.

\subsection{Verifica degli effetti secondari}  

Il punto precedente ci assicura che un gadget non abbia effetti
secondari che interferiscano con gli effetti della nostra
funzione. Tuttavia è possibile che il gadget in esame abbia altri
effetti secondari che non lo rendano utilizzabile. Ad esempio se il
gadget legge da una locazione di memoria non leggibile causerà
l'arresto del programma. La modellazione in formule logiche utilizzata
attualmente nel framework BARF non consente di porre al risolutore
domande sul numero di letture in memoria o sulla lettura o meno di
un'area di memoria. Per questo motivo per escludere effetti secondari
di questo tipo il gadget viene emulato con i valori di ingresso
ricavati nella fase di verifica e validazione, e ci si assicura che
non vengano lette e scritte aree di memoria che non siano quelle
relative alla funzione propria del gadget o all'area dello stack.


La linea guida generale appena descritta va adattata alla tipologia di
gadget analizzata, per le implementazioni specifiche fare riferimento
al capitolo \ref{cap:implementazione}. Inoltre per poter utilizzare un
gadget, i registri utilizzati come ``operandi'' devono essere
``controllabili'', cioè deve esistere una serie di gadget che consenta
di poter caricare valori arbitrari in questi registri. Per questo
prima di procedere alla verifica dei gadget si cercano all'interno
dell'eseguibile tutte le sequenze di gadget che ci permettono di
controllare registri. Se un gadget utilizza come operandi registri che
non sono ``controllabili'' questo viene scartato.

Durante queste le fasi di classificazione e verifica vengono anche
raccolte informazioni aggiuntive sui gadget, come di come un gadget
modifica il valore del registro dello stack e la posizione dello stack
dove inserire i valori utilizzati dal gadget. Queste informazioni
vengono utilizzate dal modulo \lstinline{Payload} per costruire il
payload vero e proprio.


Per le simulazioni viene utilizzata la classe \lstinline{ReilEmulator}
e la traduzione in formule SMT avviene attraverso la classe
\lstinline{CodeAnalyzer}, entrambe del framework BARF.


\section{Modulo Payload}

Questo modulo si occupa di generare il payload vero e proprio partendo
dai gadget analizzati dal modulo gadgetstools. Questa classe è
strutturata in modo che i diversi chunk prodotti siano concatenabili,
e che la loro concatenazione dia a sua una volta un frammento
concatenabile. Questo ci consente di associare ad una data funzione un
payload e utilizzarlo in diverse parti del programma senza dover più
conoscerne i dettagli. Ci sono diverse cose da tenere in conto quando
si costruisce il payload partendo dai gadgets, quasi tutte legate alla
manipolazione dello stack da parte del gadget stesso:

\begin{itemize}

  \item posizioni dei valori utilizzati dal gadget in relazione alla
    valore puntato dallo stack al momento della sua esecuzione

  \item valore al quale punterà lo stack al momento dell'esecuzione
    dell'istruzione ret del gadget stesso. In base alla tipologia di
    gadget questo valore potrebbe non essere alla fine del payload
    relativo al gadget stesso
    
  \item valore finale dello stack dopo l'esecuzione del gadget

\end{itemize}

Le informazioni relative alle tre problematiche descritte sopra
vengono estrapolate dal gadgetstools. La procedura di generazione del
payload è specifica del tipo di gadget, i metodi
\lstinline{PayloadChunk.chain} e
\lstinline{PayloadChunk.get_general_chunk} sono invece generici,
prendono come argomento una lista di chunk e restituiscono
rispettivamente il payload dei chunk concatenati o un chunk che può
essere nuovamente concatenato e rappresenta la concatenazione dei
chunks in ingresso.



\chapter{Implementazione attuale}
\label{cap:implementazione}
In questo capito si farà un'analisi dell'attuale stato di sviluppo di dropper.

\section{Dropper}

\subsection{Interazione con l'utente}

Attualmente dropper si interfaccia all'utente tramite alcune funzioni
della classe \lstinline{dropper}. Il costruttore di questa classe
riceve come argomento il percorso sul file system dove risiede il file
oggetto eseguibile da analizzare. Le operazioni di analisi vengono
avviate tramite la funzione \lstinline{analyze_all}. Una volta
analizzato si può richiamare la funzione
\lstinline{build_spawn_shell_payload} che si occupera di generare la
catena vera e propria. Anche se queste sono le uniche chiamate
essenziali l'utente ha a disposizione ulteriori funzioni:
\begin{itemize}

\item \lstinline{set_can_control_fd} setta il file descriptor
  controllato dall'utente (-1 se nessun file descriptor può essere
  controllato)

\item \lstinline{set_function_for_address_resolving} indica la
  funzione di riferimento da utilizzare nell'applicazione di tecniche
  che ricavino l'indirizzo di funzioni casualizzate in memoria
  partendo da valori nella GOT table. La funzione indicata dev'essere
  già stata richiamata almeno una volta al momento dell'esecuzione
  della catena

\item \lstinline{set_writeable_area} setta l'indirizzo di inizio
  dell'area di memoria dove effettuare operazioni di scrittura. Di
  default questo valore è settato all'indirizzo di caricamento in
  memoria della sezione \lstinline{.data}

\item \lstinline{set_cmd} setta il comando da eseguire sulla macchina
  bersaglio una volta che l'exploit ha avuto successo

\end{itemize}

Quando il file eseguibile è stato analizzato e le informazioni
necessarie impostate il metodo \lstinline{build_spawn_shell_payload}
cerca di generare una catena che avvii il comando impostato.

\subsection{Analisi degli import}

Se una funzione esterna viene utilizzata dal programma, anche se il
suo indirizzo reale viene casualizzato, sarà presente un elemento
relativo a quella funzione all'interno della procedure linkage
table. Questo elemento, ammesso che il binario non sia stato compilato
come indipendente dalla posizione in memoria (PIE), ha una posizione
nota. Possiamo quindi ritornare al suo indirizzo per emulare la
chiamata a quella funzione (return-to-plt). Nella GOT invece ogni
elemento inizialmente, e quindi anche staticamente nel file, contiene
l'indirizzo dell'istruzione successiva nell'elemento della PLT
relativo alla stessa funzione. Per ogni funzione la rilocazione
associata, nella sezione \lstinline{.rel.plt}, contiene nel campo
\lstinline{r_offset} l'indirizzo dell'elemento nella GOT table. A
partire da quest'ultimo, per le considerazioni appena fatte, può
essere ricavato l'indirizzo della entry nella PLT. Gli indirizzi del
relativo elemento della GOT e di quello nella PLT vengono ricavati per
ogni funzione utilizzata dal file oggetto eseguibile nel metodo
\lstinline{get_imports}.

\subsection{Costruzione della catena}

Attualmente il modulo dropper per prima cosa cerca di scrivere in
un'area di memoria scrivibile i dati ai quali punteranno gli argomenti
dell'invocazione della funzione \lstinline{execve}. Nel caso
quest'area di memoria non viene specificata dall'utente viene
utilizzato l'indirizzo di caricamento in memoria della sezione
\lstinline{.data}. Gli argomenti e i dati veri e propri vengono creati
dal modulo partendo dal comando che si è deciso di eseguire sulla
macchina (nella funzione \lstinline{payload_execve_args}). Se si può
controllare un file descriptor e \lstinline{read} è presente
nell'insieme delle funzioni utilizzate dal programma allora dropper
crea una catena che ritorni all'elemento all'interno della PLT
relativo a \lstinline{read} e fa in modo che vengano letti, dal file
descriptor indicato, una quantità di dati pari alla dimensione dei
dati da scrivere e che essi vengano scritti nell'area di memoria
scelta come area scrivibile. Per far questo viene utilizzato il metodo
\lstinline{get_ret_func_chunk} del modulo gadgets tool il quale ci
consente di creare una catena che richiami una funzione di cui si
conosce l'indirizzo. È compito poi dell'utente fare in modo che al
momento della chiamata di questa funzione il file descriptor indicato
si trovi nello stato giusto a restituire i valori calcolati dal
modulo. I valori da fornire nel file descriptor vengono forniti
all'utente nel parametro \lstinline{fd_payload} di dropper.
%%TODO
Esempio?

Attualmente è implementata una seconda strategia nel caso non sia
presente la funzione \lstinline{read} e/o non si controlli nessun file
descriptor. Essa consiste nell'utilizzare una serie di gadget che
possano spostare valori dallo stack in una locazione di memoria
arbitraria.  Per costruire una catena del genere si chiama la funzione
\lstinline{build_memory_write} del modulo gadgets tools che prende
come argomenti l'indirizzo della memoria dove scrivere e i dati da
scrivere. Solitamente, come vedremo in \ref{sec:memorystore}, le
catene generate in questo modo tendono ad essere piuttosto lunghe.

%%TODO inserire strategie non implementate?

Per quanto riguarda la seconda fase, la chiamata della funzione, sono
state implementate solo due strategie. La prima nel caso
\lstinline{execve} sia presente all'interno delle funzioni utilizzate
dal programma. In questo caso la generazione della parte restante
della catena è banale e, come visto nel paragrafo precedente, viene
costruita dalla funzione \lstinline{get_ret_func_chunk} del modulo
gagdettools. Più interessante è il caso in cui il file oggetto
eseguibile non utilizzi questa funzione. Una strategia che è possibile
adottare in questo caso si basa sul calcolo dell'indirizzo della
funzione voluta partendo da una funzione utilizzata dal programma, il
cui indirizzo viene ricavato dal relativo elemento nella GOT table. La
funzione dev'essere già stata chiamata almeno una volta, questo fa si
che nella GOT table sia presente l'indirizzo reale e non l'indirizzo
del relativo elemento all'interno della PLT. L'indirizzo può essere
sia calcolato in un registro o in una zona di memoria per poi
sfruttare una tipologia di gadget che ci permetta di ``saltare''
all'indirizzo calcolato (\emph{got dereferencing}). È anche possibile,
se l'eseguibile non è compilato con RORELOC, modificare il valore
della GOT in loco, aggiungendo l'offset necessario (\emph{got
  patching}). Attualmente è implementata solo la seconda strategia. Il
modulo gadgets tools espone il metodo \lstinline{build_memory_write} che
si occupa dei dettagli implementativi di questo tipo di catena. Viene
chiamato da dropper con i seguenti parametri:

\begin{itemize}
  \item l'indirizzo di memoria dove eseguire la somma: l'indirizzo
    dell'elemento della GOT da ``patchare''

  \item il valore da aggiungere: l'offset relativo tra la funzione
    utilizzata come base e quella voluta

  \item il valore di base: l'offset che la funzione di base ha
    all'interno del file oggetto condiviso della libreria. Come
    vedremo questo rende possibile utilizzare la tecnica anche quando
    le tipologie di gadget disponibili non sono ideali al prezzo di
    sacrificare un'affidabilità del 100\% (vedi
    sez. \ref{sec:arithmeticstore})

\end{itemize}

La classe inoltre contiene alcuni metodi per il salvataggio dei dati
relativi ai gadgets su disco, questo perché l'analisi di eseguibili
molto grandi può durare diversi minuti. Al loro interno questi metodi
utilizzano la libreria \lstinline{pickle} per la serializzazione degli
oggetti.

\section{GadgetsTools}

Questo modulo estrapola e cataloga i gadget presenti nel binario in
esame ed espone i metodi di creazione delle catene per l'esecuzione di
diverse funzioni. Le catene vengono costruite arrangiando i gadget
disponibili e collaborando con il modulo payloadchunk che si occupa
della generazione del payload vero e proprio. Fondamentale per poter
utilizzare un gadget è poter controllare i registri coinvolti nelle
sue operazioni, per questo motivo la prima analisi che viene condotta
una volta ottenuta la lista dei gadget è la selezione di quelli che ci
consentono di controllare i registri. La ricerca dei gadget viene
effettuata in tutte le sezioni che vengono caricate in memoria
(discriminate dal tipo \lstinline{SHT_PROGBITS}) e marcate come
eseguibili.

\subsection{Selezione dei gadget per l'impostazione dei registri}

La logica e le strutture dati utilizzate per la selezione e la
gestione dei gadget utilizzati per l'impostazione dei registri è
contenuta nella classe \lstinline{RegSet}. Il metodo
\lstinline{_analyze_gadget} riceve come argomento in ingresso un
gadget e, nel caso esso possa essere utilizzato per l'impostazione di
registri, viene aggiunto nelle strutture dati interne alla classe. La
classe espone il metodo \lstinline{get_chunk} che ritorna una catena
che imposta i registri in modo da riflettere le coppie registro/valore
del dizionario preso come argomento. Attualmente solo gadget che
contengono istruzioni di tipo \lstinline{pop reg} vengono considerati
come potenziali candidati per l'impostazione di registri. L'analisi
avviene tramite la simulazione istruzione per istruzione. Se
l'istruzione attuale è di tipo \lstinline{pop reg} allora si memorizza
il nome del registro e l'offset tra il valore attuale dello stack e
quello all'inizio della simulazione. Questo ci consente di conoscere
dove posizionare il valore da impostare nel registro al momento della
generazione del payload. Durante la simulazione si controlla anche che
le istruzioni che non sono del tipo \lstinline{pop reg} non eseguano
operazioni di lettura o scrittura sulla memoria. Questo ci consente di
scartare già in questa fase i gadget che presentano effetti
secondari. Alla fine della funzione abbiamo la lista dei registri che
il gadget può impostare con i relativi offset dello stack. La funzione
\lstinline{get_chunk} invece restituisce una catena che consente di
impostare l'insieme dei registri che riceve in ingresso. Dato che un
gadget potrebbe impostare più di un registro, la scelta dei gadget che
andranno a comporre questa catena è riducibile ad un problema di
\emph{set covering}. È preferibile che il payload generato dalla
catena sia il più piccolo possibile. La lunghezza della catena è
influenzata sia dal numero di gadget utilizzati, sia dal padding che è
necessario inserire nell'utilizzo di ogni gadget per via delle
operazioni che esso effettua sul registro dello stack. Attualmente per
trovare la catena viene utilizzato un algoritmo greedy che scorre i
registri da ``coprire'' e, se il registro non è già coperto da un
gadget inserito nella catena, sceglie il gadget che lo copre e che
necessita del minor padding possibile. Se l'aggiunta di un gadget ne
rende superfluo uno già presente nei candidati quest'ultimo viene
rimosso. La soluzione trovata con quest'algoritmo non è ottima.

\subsection{Selezione dei gadget per la scrittura in memoria}
\label{sec:memorystore}

La logica e le strutture dati per gestire i gadget che ci consentono
di scrivere in memoria si trovano nella classe
\lstinline{MemoryStore}.  Durante la prima classificazione la classe
\lstinline{GadgetClassifier} del framework BARF verifica se a seguito
di alcune simulazioni un valore presente in un registro (denominato
sorgente) viene scritto in una locazione di memoria il cui indirizzo
corrisponde al valore di un registro (denominato destinazione) sommato
ad un eventuale offset. Ogni volta che questa condizione viene
verificata il gadget viene aggiunto alla categoria StoreMemory. È da
notare che la condizione di cui sopra potrebbe verificarsi per più
coppie di registri nello stesso gadget, per ognuna di queste viene
creato un oggetto gadget di tipo StoreMemory e aggiunto alla
categoria. La classe parte da questi gadget per procedere alla fase di
verifica e validazione. Innanzitutto se il registro di destinazione è
il registro dello stack il gadget viene scartato. Questo fa in modo
che non vengano presi in considerazione i gadget con istruzioni tipo
\lstinline{push reg}. Questi gadget infatti, solitamente non
consentono di scrivere in una regione arbitraria della memoria, oltre
a complicare la gestione dello schema dello stack. Dopo di che vengono
scartati anche i gadget che hanno come registro di destinazione rip e
eip. Infatti essendo questo registro fisso rispetto all'istruzione che
lo esegue neanche questi gadget ci consentono di scrivere in una
posizione arbitraria di memoria. Dopo di che si procede seguendo le
linee guida descritte in \ref{s:verifica_validazione}: il gadget
viene scartato se non si possono controllare i registri sorgente e
destinazione, si verifica e valida la generalità e ci si assicura che
il gadget non presenti effetti secondari. I valori scelti casualmente
per la verifica e la validazione sono la locazione di memoria e il
valore da scrivere. Nella verifica degli effetti secondari si
controlla che non ci siano scritture al di fuori della locazione
scelta e dello stack. Il metodo \lstinline{get_chunk} invece prende
come argomenti un indirizzo in memoria e dei valori e restituisce una
catena che scrive i valori in memoria partendo da quell'indirizzo. Il
metodo sceglie uno dei gadget disponibili e calcola quante iterazioni
sono necessarie per la scrittura dei valori. Per ogni iterazione poi
utilizza il risolutore SMT per ottenere i valori dei registri sorgente
e destinazione corretti. Sempre per ogni iterazione, si utilizza la
classe \lstinline{regset} per ottenere una catena che setti i registri
a quei valori, e la si concatena a sua volta al gadget che si sta
utilizzando per scrivere in memoria. Viene restituita la
concatenazione delle catene create per ogni iterazione, che
verosimilmente scriverà in memoria, partendo dall'indirizzo indicato
i valori in ingresso.

\subsection{Selezione dei gadget per la operazioni aritmetiche in
  memoria}
\label{sec:arithmeticstore}

La logica e le strutture dati per la gestione dei gadget che ci
consentono di scrivere in memoria risiedono nella classe
\lstinline{aritmeticmemorystore}. Anche in questo caso si parte dai
gadget classificati dal framework BARF, ma appartenenti, per
l'appunto, alla categoria Arithmetic Memory Store. Un gadget viene
inserito in questa categoria se, dopo alcune simulazioni del gadget,
una locazione di memoria viene modificata e come valore finale
riflette il risultato di un'operazione aritmetica binaria tra il
valore precedente di quella locazione e il valore iniziale contenuto
in un registro. Inoltre si individua un registro che potrebbe essere
usato dal gadget come indirizzo della locazione di memoria (insieme ad
un eventuale offset). Dato che la simulazione viene ripetuta più volte
i falsi positivi vengono ridotti. È possibile che un gadget esibisca
il comportamento di cui sopra più di una volta, anche in questo caso
vengono creati diversi oggetti di tipo Arithmetic Memory
Store. L'analisi di questi gadget procede, come anche nella categoria
precedente, sulla falsa riga della procedura indicata in
\ref{s:verifica_validazione}. Tuttavia un'attenzione particolare va
posta nell'analisi dei gadget che utilizzano un'operazione di
addizione che considerano il valore di riporto (ad esempio adc su
intel). Questi gadget possono rilevarsi molto utili. Spesso infatti
sono disponibili solo gadget aritmetici che ci consentono di
aggiungere un byte per volta. Effettuare un'operazione di got patching
(vedi sez. \ref{sec:got_patching}) un byte per volta senza considerare
i valori di riporto tra un byte e il successivo può portare ad un
risultato errato dell'operazione che si sta effettuando. Per poter
permettere la cataloghizzazione di questo tipo di gadget, si è dovuto
applicare una piccola patch al framework per permetterci di poter
controllare in qualche modo il contesto dei registri che utilizza
durante le simulazioni per la classificazione. Infatti il
classificatore, nella traduzione in reil, considera il flag carry del
registro dei flag come un registro a se stante di 16bit che viene
aggiunto al risultato dell'operazione. Dato che questo valore è
impostato casualmente i risultati delle simulazioni vengono
falsati. La nostra modifica ci permette di settare un valore
arbitrario per alcuni registri nelle simulazione, quindi
inizializzeremo il ``registro'' con il quale è rappresentato il carry
flag a zero. Questo viene anche imposto nelle verifiche che utilizzano
un risolutore SMT. In più viene aggiunta una seconda coppia di
modellazioni SMT con le quali si verifica se il gadget in esame
utilizza il carry flag nell'operazione e se a seguito di un'operazione
che sicuramente generi un valore di riporto, il carry flag al termine
del gadget abbia valore uno. Questo ci dà la certezza rispettivamente
che il gadget va ad utilizzare il carry flag nelle operazioni di
addizione e che le operazioni successive all'interno del gadget non
vadano ad influenzare il carry flag. Come detto sopra, per costruire
una catena che ci consenta di aggiungere un valore arbitrario ad una
locazione di memoria arbitraria, bisogna avere a disposizione o gadget
aritmetici la cui dimensione dell'operazione sia di almeno un byte più
grande dell'addendo o un gadget che, per l'appunto, consenta di
eseguire operazioni utilizzando i valori di ritorno. In questo secondo
caso però bisogna anche concatenare, se è disponibile, un gadget che
abbia come effetto quello di azzerare il carry flag, da inserire prima
dell'operazione. La logica per la ricerca di questo tipo di gadget è
presente nella classe \lstinline{ClearCarryFlag}. È da notare che se
questo tipo di gadget non fosse presente la catena generata potrebbe
non avere l'esito sperato, tuttavia è presumibile che in molti tipi di
exploit il carry flag abbia sempre lo stesso valore all'inizio
dell'esecuzione della catena. La sottrazione quindi di un'unità al
primo byte dell'offset dovrebbe essere sufficiente per generare una
catena corretta. Se non sono disponibili né un un gadget che consenta
di aggiungere abbastanza bytes per volta né un gadget che effettua
operazioni considerando il valore di riporto è tuttavia possibile
generare una catena con alte probabilità di successo. Infatti
l'indirizzo di memoria nel quale viene caricato il file oggetto
condiviso della libreria, nonostante sia casualizzato, dev'essere
allineato in memoria al valore specificato nel parametro
\lstinline{p_align} all'interno del program header. Questo comporta
che un certo numero di bits meno significativi dell'indirizzo di una
funzione non cambino da un'esecuzione all'altra. Solitamente sono i
21bits meno significativi, e l'offset tra due funzioni nella libc
solitamente è tre byte. Questo fa si che si può calcolare il valore di
riporto per i primi due byte. Solo il valore di riporto dell'ultimo
byte (di cui conosciamo solo la parte meno significativa) risulterebbe
casuale, facendo si che si possano generare catene con probabilità di
successo pari almeno al 50\%.


È da notare che nell'implementazione attuale manca ancora
l'implementazione della logica per la ricerca e la verifica di
importanti categorie di gadget (come Jump, MemoryLoad, MoveRegister o
Arithmetic). L'implementazione di queste categorie dovrebbe consentire
la generazione di catene in condizioni più articolate o quando le
categorie di gadget attualmente implementate non sono disponibili.

\section{PayloadChunks}

Questo modulo si occupa di generare a partire da una lista ordinata di
uno o più gadgets un array binario di dati da iniettare nello stack
del file oggetto eseguibile esaminato. La preparazione del payload
varia rispetto al gadget per il quale si sta creando il chunk e può
consistere nel solo indirizzo del gadget, o in schemi più complicati
come indirizzi di funzioni più relativi argomenti o a valori da
trasferire dalla stack ai registri. Ogni classe che si occupa della
generazione di un particolare tipo di chunk ha le informazioni
necessarie per permettere al chunk stesso di essere concatenato ad
altri chunk per generarne un altro, anch'esso a sua volta
concatenabile, che rappresenti i chunk di partenza. Per permettere
questa concatenazione ogni chunk deve presentare le seguenti
informazioni:

%3201899105

\begin{itemize}

\item L'indirizzo che deve trovarsi sullo stack al momento
  dell'esecuzione dell'istruzione ret del gadget precedente 

\item In quale posizione del payload inserire l'indirizzo del gadget
  successivo

\item La dimensione degli indirizzi nell'architettura considerata

\end{itemize}


Le classi che modellano i vari tipi di chunk hanno tutti un metodo,
\lstinline{get_payload}, che restituisce il payload del chunk di
riferimento (escluso l'indirizzo del gadget stesso). Quando due gadget
vengono concatenati per generare un chunk più generico (nel metodo
\lstinline{PayloadChunk.get_general_chunk}), l'indirizzo di ogni chunk
viene inserito nella posizione indicata dal chunk precedente. Nel
momento che si richiede il payload vero e proprio (attraverso il
metodo \lstinline{PayloadChunk.chain}) oltre all'operazione di
concatenazione viene anche inserito l'indirizzo del primo gadget
all'inizio del payload.

\section{Utilizzo di pianificatori per la costruzione della
  catena}
\label{sec:pddl}

Il problema di generare una catena (quindi una sequenza) di gadget
(che possono essere visti come azioni) per arrivare ad uno stato del
sistema desiderato si traduce naturalmente in un problema di
planning. Una volta tradotto si può utilizzare un planner per ottenere
una soluzione valida e/o (dipende dal planner che si utilizza)
ottima. PDDL è un linguaggio che ha come obiettivo quello di dare uno
strumento universale per codificare problemi di planning.  In questa
sezione esamineremo un primo approccio che può essere utilizzato per
tradurre il nostro problema nel linguaggio PDDL. Il linguaggio divide
il problema in due descrizioni differenti, una che si occupa di
definire il dominio del problema, l'altra di definirne una singola
istanza. Nella definizione del dominio vengono descritti i tipi di
oggetto che avremo nel nostro sistema, gli oggetti presenti in tutte
le instanze (\emph{constants}), i \emph{fluents}, e le azioni, insieme
agli effetti che queste hanno sugli oggetti del dominio. Una singola
istanza del problema invece definisce gli oggetti che le azioni
possono manipolare, lo stato iniziale nel quale si trova il sistema,
lo stato che vogliamo raggiungere e la metrica con la quale valutare
la bontà di un piano. Un \emph{fluents} è un mapping tra una tupla di
oggetti e un valore numerico che può cambiare durante un piano. La
loro sintassi e semantica è stata stabilizzata nel linguaggio in
\cite{fox-03}. Come evidenziato in \cite{fox-03} i fluents non possono
essere utilizzati come parametri di azioni perchè queste agiscono su
oggetti ``fisici'' e quindi i parametri devono appartenere ad insiemi
finiti (e non ad insiemi infiniti come i numeri). Per ovviare a questa
limitazione utilizzeremo come stratagemma quello di creare degli
oggetti di tipo \lstinline{imm} e associando tramite il fluent
\lstinline{val (?x - imm)} un valore a quel dato oggetto. Questo fà si
che le azioni possono utilizzare parametri di tipo imm.  

I tipi di dato utilizzati nella modellazione sono \lstinline{imm}, con
il significato appena spiegato, e \lstinline{register} che rappresenta
per l'appunto un registro. I fluent utilizzati invece sono:

\begin{itemize}
\item (val ?x - imm) come appena definito
\item (reg ?r - register) che rappresenta il valore attuale del
  registro ?r
\item (mem ?a - imm) che rappresenta il valore attuale della locazione
  di memoria ?a
\end{itemize}

Vengono create tante costanti quante i registri.  Attualmente si è
proceduto a modellare solo gadget per l'impostazione di registri o la
scrittura di memoria. È da notare che per lo stratagemma usato tutti i
valori (indirizzi e valori concreti) che le nostre azioni vanno ad
utilizzare devono essere esplicitati. Un gadget per l'impostazione di
un registro viene tradotto in questo modo (a meno dei puntini
utilizzati per generalizzare la lista dei registri che è possibile
impostare attraverso il gadget):

\begin{lstlisting}
  (:action regset_gadget
	 :parameters (?vreg0 ?vreg1 ... ?vregn - imm)
	 :precondition (and)
	 :effect (and
	 	 (assign (reg reg0) (val ?vreg0))
	 	 (assign (reg reg1) (val ?vreg1))
                 ...
	 	 (assign (reg regn) (val ?vregn))
		 (increase (stack-length) g.stack_offset)))
\end{lstlisting}

Come si può notare la traduzione avviene ad un livello d'astrazione
più alto delle singole istruzioni del gadget, dettagli come la
posizione dei valori da impostare nello stack, gli effetti secondari,
vengono gestiti sempre da dropper. La traduzione invece di un gadget
per la scrittura in memoria è leggermente più complicata:

\begin{lstlisting}
(:action memstr_gadget
   :parameters ()
   :precondition (and)
   :effect (and 
            (forall (?a ?v - imm)
             (when (and (= (val ?a) (reg gadget.destination)) 
             (= (val ?v) (reg gadget.source)))
              (assign (mem ?a) (val ?v))))
            (increase (stack-length) gadget.stack_offset)))

\end{lstlisting}

Questa modellazione poco elegante è dovuta al non poter utilizzare i
fluents come argomenti per altri fluents.  Una volta tradotto il
dominio, si crea un'istanza del problema. Si specificano gli oggetti
che faranno parte dell'istanza del problema, cioè i valori che andremo
ad utilizzare. Come stato iniziale si impostano i fluents dei registri
a zero, si assegnano i rispettivi valori ai fluents di tipo (val ?v -
imm) e si setta anche il fluent \lstinline{stack-length} a zero,
quest'ultimo viene utilizzato come metrica per valutare la bontà del
piano trovato. Come stato desiderato si imposta o, nel caso si voglia
trovare un piano per settare un insieme di registri, che i fluent
rilevanti di tipo (reg ?r) abbiano i valori desiderati o, nel caso
invece si voglia trovare un piano per impostare alcune specifiche
locazioni di memoria, che i fluents del tipo (mem ?a) abbiano il
valore corrispondente. Un esempio, con la lista dei gadgets accorciata
per brevità è riportato nelle figure \ref{pddl:domain} e
\ref{pddl:facts}. L'esempio è generato tramite la classe
\lstinline{planner}.


\begin{lstlisting}

;gadgets: 
;0x412050 : pop rsi; pop r15; ret
;0x412052 : pop rdi; ret
;0x40ea2c : mov dword ptr [rdi], esi; ret

(define (domain dropper )
(:requirements :typing :fluents :conditional-effects)
(:types imm register)
(:constants rax rbx rcx rdx rdi rsi rbp rsp rip r8 r9 r10 r11 r12 r13 r14 r15 - register)

(:functions 
  (val ?x -imm)
  (reg ?r - register)
  (stack-length)
  (mem ?addres - imm))

(:action regset_412052
   :parameters (?vrdi - imm)
   :precondition (and)
   :effect (and         (assign (reg rdi) (val ?vrdi))
     (increase (stack-length) 8)))

(:action regset_412050
   :parameters (?vr15 ?vrsi - imm)
   :precondition (and)
   :effect (and         (assign (reg r15) (val ?vr15))
         (assign (reg rsi) (val ?vrsi))
     (increase (stack-length) 16)))

(:action memstr_40ea2c
  :parameters ()
  :precondition (and)
  :effect (and (forall (?a ?v -imm) 
      (when (and (= (val ?a) (reg rdi))
      (= (val ?v) (reg rsi))) (assign (mem ?a) (val ?v))))
     (increase (stack-length) 0)))
)


\end{lstlisting}
\label{pddl:domain}


\begin{lstlisting}
(define (problem memset)
(:domain dropper)
(:objects v_32896 v_2 v_33153 v_3 - imm)
(:init
(= (val v_32896) 32896)
(= (val v_33153) 33153)
(= (val v_2) 2)
(= (val v_3) 3)
(= (reg rax) 0)(= (reg rbx) 0)(= (reg rcx) 0)(= (reg rdx) 0)(= (reg rdi) 0)
(= (reg rsi) 0)(= (reg rbp) 0)(= (reg rsp) 0)(= (reg rip) 0)(= (reg r8) 0)
(= (reg r9) 0)(= (reg r10) 0)(= (reg r11) 0)(= (reg r12) 0)(= (reg r13) 0)
(= (reg r14) 0)(= (reg r15) 0)
(= (mem v_32896) 0)
(= (mem v_33153) 0)
(= (stack-length) 0))
(:goal 
(and (= (mem v_32896) 2)
  (= (mem v_33153) 3)))
(:metric minimize (stack-length))
)
\end{lstlisting}
\label{pddl:facts}



 %TODO spostare in implementazione attuale?
%% Questo avviene emulando tutti i gadgets, e nello stesso momento
%% parsando le istruzioni. Si tiene traccia, istruzione per istruzione,
%% di quanto si sia modificato il valore dello stack. Nel caso si
%% incontri un'istruzione del tipo pop reg, viene associato a quel
%% registro l'attuale differenza tra il valore dello stack relativamente
%% all'istruzione corrente e quello all'inizio della simulazione. A
%% questo punto abbiamo per ogni gadget una lista di registri e di offset
%% rispetto ai quali inserire il valore nello stack per inserire un
%% valore arbitrario nel relativo registro. Inoltre tramite la
%% simulazione si deduce la differenza tra il valore dello stack
%% all'inizio del gadget e quello alla fine. Questo valore dovrà essere
%% noto solitamente per tutti i gadget in quando ci permette di far
%% trovare al posto giusto l'indirizzo di ritorno del gadget lo che
%% succede nella catena. Nel caso il gadget non contenga nessuna
%% istruzione del tipo pop reg, questo viene scartato. Inoltre nella
%% simulazione viene tenuta traccia degli indirizzi di memoria letti e/o
%% scritti. Nel caso vi sia scrittura nella memoria o lettura da una zona
%% di memoria non riconducibile all'area relativa allo stack il gadget
%% viene scartato. 


%% La semantica potrebbe essere poi verificata trasformando le istruzioni
%% del gadget in formule smt, verifcando che la semantica e i valori
%% estratti non siano dipendendi dai valori di ingresso, e quindi
%% relativi solo alla simulazione effettuata. In pspecial modo è possibile
%% che il valore dello stack venga modificato utilizzando un valore di un
%% registro. TODO




%% La catena dei gadget generata da dROPper deve, una volta
%% iniettata nello stack del processo in esecuzione, poter eseguire un
%% comando arbitrario sulla macchina in cui è in corso l'esecuzione del
%% programma. La catena generata deve funzionare anche nel caso che gli
%% indirizzi in memoria delle librerie utilizzate da programma siano
%% stati casualizzati (ASRL) e che nè lo stack nè alcuna regione
%% scrivibile sia eseguibile (NX). La strategia da utilizzare dipende dal
%% binario utilizzato e da che tipo di controllo abbiamo sui dati di
%% ingresso del programma. In ogni caso lo schema principale, con alcune
%% varianti che analizzeremo in seguito, si può dividere in due fasi
%% principali: (a) preparare gli argomenti per eseguire una chiamata ad
%% una funzione di libreria che ci consenta di eseguire un comando
%% arbitrario e (b) eseguire quella funzione al fine si eseguire il
%% comando voluto. Le strategie per arrivare alla concretizzazione di
%% queste due fasi dipendono dal file oggetto eseguibile che si sta
%% analizzando che può presentare scenari relativamente semplici, in cui
%% ad esempio il programma utilizza già le chiamate a funzioni
%% necessarie, permettendoci di utilizzare \emph{return-to-plt} sia a
%% scenari più complessi, dove l'indirizzo delle funzioni necessarie
%% viene ottenuto durante l'esecuzione partendo dagli indirizzi presenti
%% nella Global Offset Table.

%% Dropper inizia ad analizzare il binario affidandosi al framework BARF
%% per l'estrazione e una prima cataloghizzazione dei gadget. Le
%% categorie in cui il framework suddivide i gadget sono quelle indicate
%% in[]:

%% \begin{itemize}


%% \end{itemize}

%% \section{Implementazione attuale}

%% \subsection{dropper}

%% Dropper legge dal file le sezioni
%% \lstinline{.rel.plt}, \lstinline{.dynsym},\lstinline{.got} e
%% \lstinline{.got.plt}.

%%  analizza
%% poi la sezione ``.data'', se non viene specificato diversamente
%% dall'utente verrà utilizzato l'indirizzo in memoria di questa sezione
%% come area di memoria dove poter scrivere dati. Questo ci permette di
%% avere una posizione nota (l'indirizzo in memoria di ``.data.'' non è
%% casualizzato) dove poter scvrivere gli argomenti delle funzioni che
%% andremo ad utilizzare



%% TODO :
%% # verificare i gadget di tipo regset
%% # trovare i tipi regset e modellare la memoria (utilizzando > e < per
%% segnare le aree di memoria scrivibili e le aree di memoria leggibili)


%% Una prima
%% possibiltà è quella di utilizzare delle sequenze di gadget che ci
%% consentato di modificare un valore in memoria aggiungendoci un
%% addendo. In questo modo è possibile addizionare l'offset noto nell got
%% table, e successivamente fare un ret-to-plt. Il valore presente nella
%% got table adesso punterà alla funzione voluta. Per utilizzare questa
%% tecnica devono essere soddisfatte le seguenti precondizioni:

\chapter{Conclusioni}

Lo sviluppo di uno strumento come Dropper mi ha permesso di studiare
ed esplorare problematiche interessanti e complesse. Uno strumento che
riesce infatti ad analizzare un file binario e da esso estrarne in
maniera automatica le informazioni necessarie per poter costruire un
payload funzionante implica la risoluzione di diverse problematiche
che abbracciano più aree dell'informatica. In particolare in questo
lavoro alcuni strumenti propri dell'intelligenza artificiale hanno
giocato un ruolo cruciale rendendo evidente l'apporto che questa
disciplina può dare all'area della sicurezza informatica in
generale. Tra le problematiche più rilevanti nello sviluppo di uno
strumento come Dropper elenchiamo:

\begin{enumerate}
\item L'estrapolazione della semantica dei singoli gadget
\item L'individuazione di una sequenza di gadget che ci consente di
  eseguire una data operazione. Questo punto è complicato dal fatto che gli
  effetti di un gadget possono interferire con quelli di un altro
  nonché con l'esecuzione stessa del programma in esame
\item Scegliere, in base alle operazioni disponibili, la strategia da
  utilizzare per eseguire l'exploit
\item Individuare una sequenza di operazioni che eseguono con successo
  una data strategia
\end{enumerate}

Il primo punto è stato affrontato, come descritto nei capitoli
\ref{cap:architettura} e \ref{cap:implementazione}, attraverso sia
tecniche di emulazione che tecniche di verifica basate sulla
rappresentazione in formule logiche delle istruzioni dei
gadget. Dropper può estrapolare la semantica di diverse tipologie di
gadget che ci permettono di eseguire operazioni, come il caricamento di
costanti nei registri o l'effettuare operazioni aritmetiche in
locazioni di memoria arbitrarie, e ed è facilmente estendibile per
incorporare ulteriori analisi semantiche che consentano di eseguire
operazioni sempre più complesse.

La scelta di una sequenza di gadget che ci permetta di eseguire
un'operazione articolata è un'altra problematica
interessante. Infatti, oltre ad essere sicuri che la catena abbia gli
effetti desiderati, è interessante cercare, tra le sequenze di gadget
che ci permettano una particolare operazione, quella che produca un
payload di minore lunghezza. Infatti, anche se questo dipende molto
dalla tipologia di errore che si sta sfruttando, spesso si è limitati
nella quantità di dati da poter iniettare.

Le catene che attualmente dropper riesce a generare ci consentono di
avere una visione ad un livello di astrazione più alto delle
operazioni che possiamo eseguire con i singoli gadget. Per generare le
catene di queste operazioni, come descritto nei capitoli precedenti,
si utilizza un approccio per lo più programmatico, usando quando
possibile algoritmi greedy per la generazione di soluzioni di una
certa qualità. Tuttavia il problema di generazione di una catena,
definito come l'ordine nel quale effettuare alcune azioni (i gadget)
per raggiungere un dato stato del sistema, si presta naturalmente ad
una traduzione in un problema di planning.

Alcune considerazioni su questo approccio, insieme ad una prima
possibile modellazione che utilizzi il linguaggio PDDL 2.1
\cite{pddl-97,fox-03} sono riportate nella sez. \ref{sec:pddl}.

Le ultime problematiche che sono state affrontate, nella nostra
panoramica dal basso verso l'alto, sono la scelta di una tecnica di
exploiting e di una sequenza di operazioni per applicare tale
tecniche. Tra le strategie applicate con successo nei primi testcase
troviamo
\begin{inparaenum}[1)]
\item l'utilizzo della funzione ``read'' per la scrittura in memoria
\item utilizzo di sequenza di gadget sempre per la  scrittura in memoria
\item got patching e
\item return-to-plt
\end{inparaenum}

Trovare le sequenze per portare a termine queste tecniche è un
problema per certi versi simile a trovare le catene di gadget per
eseguire le operazioni ``elementari'', ma da un punto di astrazione
più alto.  Vi è qui, tra le difficoltà più rilevanti, quella di
esaminare tutte le possibili sequenze di operazioni per applicare
quella tecnica. In questa fase i vantaggi dell'utilizzo di un planner
sembrano ancora più rilevanti.

I test durante lo sviluppo sono stati effettuati prendendo in esame
alcuni livelli di exploit-exercise\cite{exploit-exercise} e alcuni
eseguibili contenuti nelle bin-utils (ls, echo, mv).

\section{Sviluppi futuri}

Dropper ha prospettive di sviluppo su diversi fronti. 

Sicuramente l'utilizzo di un'esecuzione simbolica semplificherebbe di
molto l'analisi semantica, potendo verificare direttamente le
relazioni tra valori in ingresso ed effetti dei gadget.

Dai primi risultati ottenuti utilizzando un planner,  ci si è resi
conto che quest'area potrebbe portare notevoli vantaggi ma che
richiede ulteriori indagini.

Un'altra idea interessante è cercare di estrapolare non solo gadget
``classici'' ma anche altre tipologie. Ad esempio un'estensione della
ROP consiste nell'utilizzare sequenze di istruzioni che terminano con
un'istruzione di tipo \lstinline{jmp} invece che con una di tipo
\lstinline{ret} (infatti questa tecnica prende il nome di \emph{Jump
  Oriented Programming} \cite{Checkoway-10}).

Un altro esempio è quello di ricercare gadget di lunghezza maggiore ma
con effetti secondari ``controllabili'', ad esempio obbligando il
gadget a seguire un flusso di esecuzione piuttosto che
un altro. Quest'ultima tipologia di gadget è necessaria per eludere
alcune tecniche di mitigazione che si basano sul monitoraggio di un
numero limitato di istruzioni.

%%% Local Variables: 
%%% mode: latex
%%% TeX-master: "tesi"
%%% End: 


%% \clearpage{\pagestyle{empty}\cleardoublepage}
%% \listoffigures \addcontentsline{toc}{chapter}{Indice delle figure}
%% \listoftables \addcontentsline{toc}{chapter}{Indice delle tabelle}
%% \clearpage{\pagestyle{empty}\cleardoublepage}

\bibliographystyle{plain}
\bibliography{tesi}

%% \include{latex}
\end{document}
