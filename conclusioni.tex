\chapter{Conclusioni}

Lo sviluppo di uno strumento come dropper mi ha permesso di studiare
ed esplorare problematiche interessanti e complesse. Uno strumento che
riesce infatti ad analizzare un file binario e da esso estrarne in
maniera automatica le informazioni necessarie per poter costruire un
payload funzionante implica la risoluzione di diverse problematiche
che abbracciano più aree dell'informatica. In particolare in questo
lavoro alcuni strumenti propri dell'intelligenza artificiale hanno
giocato un ruolo cruciale rendendo evidente l'apporto che questa
disciplina può dare all'area della sicurezza informatica in
generale. Tra le problematiche più rilevanti nello sviluppo di uno
strumento come dropper elenchiamo:

\begin{enumerate}
\item L'estrapolazione della semantica dei singoli gadget
\item L'individuazione di una sequenza di gadget che ci consente di
  eseguire un'operazione articolata. Questo punto è complicato dal
  fatto che gli effetti di un gadget possono interferire con quelli di
  un altro nonché con l'esecuzione stessa del programma in esame
\item Scegliere, in base alle operazioni disponibili, la strategia da
  utilizzare per eseguire l'exploit
\item Individuare una sequenza di operazioni che portino a successo una
  data strategia
\end{enumerate}

Il primo punto è stato affrontato, come descritto nei capitoli
\ref{cap:architettura} e \ref{cap:implementazione}, attraverso sia
tecniche di emulazione che tecniche di verifica basate sulla
rappresentazione in formule logiche delle istruzioni dei
gadget. Dropper attualmente può estrapolare la semantica di tipologie
che ci permettono di:

\begin{itemize}
\item Impostare i valori di registri
\item Effettuare operazioni aritmetiche in locazioni di memoria
\item Scrivere su locazioni di memoria
\item Azzerare il carry flag
\end{itemize}

Vengono inoltre inferite le informazioni necessarie per l'utilizzo dei
gadget e gli effetti secondari con altri gadget (clobbering dei
registri, scritture in aree di memoria).

La scelta di una sequenza di gadget che ci permetta di eseguire
un'operazione articolata è un'altra problematica
interessante. Infatti, oltre ad essere sicuri che la catena abbia gli
effetti desiderati, è interessante cercare, tra le sequenze di gadget
che ci permettano una particolare operazione, quella che produca un
payload di minore lunghezza. Le catene che attualmente dropper riesce
a generare combinano le tipologie di gadget viste sopra e ci
consentono una visione ad un livello di astrazione più alto delle
operazioni che possiamo eseguire con i gadget disponibili. Per
generare le catene di queste operazioni, come descritto nei capitoli
precedenti, si utilizza un approccio per lo più programmatico, usando
quando possibile algoritmi greedy per la generazione di soluzioni di
una certa qualità, cioè il cui payload abbia una lunghezza
contenuta. Tuttavia il problema di generazione di una catena, definito
come l'ordine nel quale effettuare alcune azioni (i gadget) per
raggiungere un dato stato del sistema, si presta naturalmente ad una
traduzione in un problema di planning . È stata esplorata questa
strada, traducendo il problema, una volta estratta la semantica dei
gadget, nel linguaggio PDDL 2.1 \cite{pddl-97,fox-03}. La strategia
di modellizzazione utilizzata in questo primo approccio è riportata in
\ref{sec:pddl}. Le ultime problematiche da affrontare, nella nostra
panoramica dal basso verso l'alto, sono la scelta di una tecnica di
exploiting e trovare una sequenza di operazioni per applicare tale
tecniche.  Attualmente sono implementate le seguenti strategie:

\begin{itemize}
\item utilizzo della funzione ``read'' per la scrittura in memoria
\item utilizzo di sequenza di gadget per scritture dai in memoria
\item got patching
\item return-to-plt
\end{itemize}

Trovare questa sequenza è un problema per certi versi simile a trovare
le catene di gadget per eseguire le operazioni ``elementari'', ma da un
punto di astrazione più alto.  Vi è quindi, tra le difficoltà più
rilevanti, quella di esaminare tutte le possibili sequenze di
operazioni per applicare quella tecnica. In questa fase i vantaggi,
sopratutto pensando a tecniche non ancora implementate come \emph{got
  dereferencing}, dell'utilizzo di un planner sembrano ancora più
rilevanti. Lo strumento è stato testato su alcuni livelli di
exploit-exercise\cite{exploit-exercise} e su alcune delle bin-utils
(ls, echo, mv), riuscendo a ritornare payload funzionanti una volta
iniettati nello stack.

\section{Sviluppi futuri}


Dropper ha prospettive di sviluppo su diversi fronti. 

L'aggiunta dell'estrapolazione semantica per le categorie di gadget
mancanti consentirebbe l'applicazione di ulteriori tecniche per
l'exploitation.

Per quanto riguarda la generazione della catena invece si è iniziato
ad utilizzare un planner, ma essendo ancora ad una fase iniziale vi è
molto spazio per ulteriori sviluppi e indagini.

Un'altra idea interessante è cercare di estrapolare non solo gadget
``classici'' ma anche altre tipologie, come ad esempio gadget che
termino con un \lstinline{jmp} invece che con un ret, utilizzati nella
\emph{Jump Oriented Programming}, oppure gadget di lunghezza maggiore
ma con effetti secondari ``controllabili'' (ad esempio gadget lunghi
ma dove si può forzare un flusso di esecuzione). Quest'ultima
tipologia di gadget è necessaria per eludere alcune tecniche di
mitigazione che si basano sul monitoraggio di un numero limitato di
istruzioni.

Un vantaggio notevole, si avrebbe nell'effettuare anche simulazioni
simboliche per l'estrapolazione e la verifica semantica dei gadget.


%%% Local Variables: 
%%% mode: latex
%%% TeX-master: "tesi"
%%% End: 
