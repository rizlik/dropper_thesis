\thispagestyle{empty}
\
\newpage
\clearpage{\pagestyle{empty}\cleardoublepage}
\mainmatter
\chapter{Introduzione}
%\addcontentsline{toc}{chapter}{Introduzione}


A seguito dell'introduzione di misure di sicurezza come l'isolamento
delle pagine di memoria, la casualizzazione dello spazio degli
indirizzi, e il controllo della consistenza dello stack le tecniche
per poter sfruttare gli errori presenti in un programma al fine di
ottenerne il controllo sono cambiate radicalmente.

La tecnica forse più abusata era lo \emph{stack smashing}
\cite{Phrack-96}, con la quale si compromisero centinaia di
sistemi. La tecnica consisteva nello sfruttare un errato controllo dei
limiti di un buffer memorizzato sullo stack per poter corrompere lo
stack stesso, errore identificato come \emph{stack buffer
  overflow}. In questo modo poteva essere ``iniettato'' nello stack
codice arbitrario, solitamente definito \emph{shellcode} per via del
suo obiettivo più comune: ottenere una shell. 

Nello stack sono presenti informazioni chiave per il funzionamento del
programma. Ad esempio, al momento di una chiamata a funzione,
l'indirizzo dell'istruzione successiva viene memorizzato sullo stack,
in questo modo, una volta terminato il lavoro della funzione, il
programma sa da dove riprendere l'esecuzione. L'attaccante, una volta
preso il controllo dello stack, può sovrascrivere il valore
dell'indirizzo di ritorno, controllando da dove il programma
continuerà. L'attaccante quindi può, compromettendo lo stack,
redirezionare il flusso del programma in modo che esegua il codice da
lui stesso iniettato.

Sono state introdotte diverse misure di protezione per mitigare questo
tipo di attacco. Con l'introduzione dell'isolamento delle pagine di
memoria è stato possibile adottare una politica dei permessi in
memoria con granulità una singola pagina. Il tentativo di eseguire
delle istruzioni presenti in una pagina non eseguible, così come il
tentativo di scrivere su una pagina non scrivbile causa l'arresto del
programma. Il codice leggittimo viene mappato in memoria in pagine
eseguibili ma non scrivibili, mentre stack e dati che il programma
deve manipolare vengono mappati su pagine scrivibili ma non
eseguibili. In questo modo non si può più semplicemente iniettare del
codice, in quanto le aree di memoria scrivibili (tra cui l'area di
memoria riservata allo stack) non potranno essere eseguite. Un'altra
misura di protezione per rendere più difficile questo tipo di attacco
è la casualizzazione dell'indirizzo di memoria nel quale risiede lo
stack. In questo modo, non sapendo dove il codice iniettato
effettivamente risiede, non si sa dove dover redirezionare il
controllo del flusso per eseguire il codice iniettato.

Tuttavia l'isolamento delle pagine di memoria e la casualizzazione
dello stack non evita la compromissione nè del flusso del programma nè
dello schema dello stack. Una delle tecniche utilizzata per ottenere
il controllo di un programma nonostante lo stack non sia marcato come
eseguibile si chiama \emph{return-to-libc}\cite{c0ntex}. In un utilizzo
tipico di questa tecnica il flusso del programma non viene dirottato
su un payload iniettato dall'attaccante ma su una funzione di una
qualche libreria utilizzata dal programma. Su un architettura a 32bit
gli argomenti vengono passati alle funzioni posizionandoli secondo un
ordine preciso (definito dalla \emph{calling convention}) sullo
stack. Dato che anche lo schema dello stack può essere compromesso
l'attaccante ha di fatto il controllo degli argomenti da passare alla
funzione. Potendo, ad esempio, richiamare la funzione system della
libc con argomenti arbitrari è possibile eseguire un qualiasi comando
sulla macchina in cui si sta eseguendo il programma, oppure chiamando
la funzione mprotect (sempre nella libc) con gli opportuni argomenti
si possono modificare i permessi della pagina di memoria dove risiede
lo stack, rendendola eseguibile. A quel punto redirezionare il flusso
del programma sul codice iniettato sullo stack non ne provocherà
l'arresto. La tecnica funziona perchè il codice ad essere eseguito
risiede in un area di memoria eseguibile. Una tecnica per mitigare
questo tipo di attacco è la casualizzazione dello spazio degli
indirizzi (\emph{Address Space Layout Randomization}). Le librerie
vengono mappate ad indirizzi di memoria casuali, che cambiano da
esecuzione ad esecuzione del programma. In questo modo non si
conoscono gli indirizzi delle funzioni e un attacco di tipo
return-to-libc naive non risulta possibile. Tuttavia su 32bit e sotto
alcune condizioni questi indirizzi possono essere facilmente ricavati
con un attacco di tipo bruteforce\cite{Shacham:2004}. Anche se su
un'architettura 64bit un attacco bruteforce non risulta fattibile è
tuttavia possibile che si riesca a sfruttare qualche altro errore di
programmazione per ottenere l'indirizzo della libreria, riuscendo così
a montare comunque l'attacco.

Una generalizzazione della \emph{return-to-libc} è la tecnica che
prende il nome di \emph{Return Oriented
  Programming}\cite{Shacham-2007,Roemer-2012}. In questo caso il
flusso viene redirezionato non più all'inzio di una funzione, ma
all'inizio di una piccola serie di istruzioni, già presenti in un area
di memoria del programma marcata come eseguibile, seguita da
un'instruzione ret. L'istruzione ret preleva dalla memoria dello stack
un indirizzo e ridireziona il flusso del programma a quel dato
indirizzo. Viene utilizzata al termine dell'esecuzione di una funzione
per tornare alla porzione di codice che l'ha richiamata. Potendolo
controllare l'attaccante può sistemare lo schema dello stack in modo
che una volta che il programma arrivi all'istruzione ret, sullo stack
sia presente l'indirizzo di una seconda serie dalle caratteristiche
simili alla prima. Ripetendo lo stesso discorso più volte è possibile
concatenare un numero arbitrario di queste piccole serie di istruzioni
(che prendono il nome di \emph{gadget}). In \cite{roemer-12} è stato
dimostrato che, quando sono presenti opportunee tipologie di gadget,
utilizzando il ROP si possono effettuare computazioni
turing-complete. Utilizzando opportunee sequenze di gadget è possibile
applicare tecniche per eludere varie misure di protezione \cite{roglia:2009}.

Un programma può essere compilato affinchè il suo codice sia
indipendente dalla posizione in cui viene mappato in memoria
(\emph{Position Indipendent Code}), in questo modo non solo le
librerie, ma anche il programma stesso può essere caricato in un
indirizzo casuale di memoria, che cambia tra un'esecuzione e
l'altra. In questo caso non è più facilmente possibile montare un
attacco che sfrutti i gadget, in quanto la posizione degli stessi non
è nota. Altre tecniche sono stati sviluppate per mitigare attacchi di
tipi ROP, anche se non risultano efficaci o l'overhead è troppo grande
\cite{Davi-2014}.

La \emph{Satisfability Modulo Theories}\cite{Barrett-14} è
un'estensione della logica del primo ordine, che viene estesa con
teorie come quella dei bitvector e degli array, rendendolo un ottimo
strumento per la modellazione molto dell'architettura di un
elaboratore (specialmente dei registri e della memoria). Negli ultimi
tempi è stata dedicata molta attenzione all'utilizzo di risolutori di
formule SMT con applicazione nell'ambito della sicurezza\cite{}
informatica. Riuscendo infatti a modellare le istruzioni e il contesto
di esecuzione di un programma in formule SMT, il risolutore può essere
utilizzato, ad esempio, per verificare che un programma aderisca ad
una data semantica o, per estrapolarne la semantica stessa.

Anche la simulazione simbolica viene utilizzata sempre di più
nell'ambito della sicurezza. Ad esempio può essere utilizzata per
mettere in relazione dati in ingresso e comportamento del
programma. Semplificando, una volta trovato un modo per poter simulare
il programma, ad un valore concreto di ingresso viene sostituito un
valore simbolico. Ogni volta che questo valore viene manipolato viene
tenuta traccia delle operazioni effettuate su di esso. Le strutture
dati del programma possono essere espresse in funzione di quel dato
valore simbolico. E' possibile poi ricavare in modo automatico una
formula SMT che rappresenti queste relazioni, in modo da poter
interrogare un risolutore per chiedere, ad esempio, per quali valori
in ingresso il programma segua o meno una branch o per quali valori
una struttura dati abbia o no un dato valore.

Nell'ambito della ROP un risolutore SMT può essere utilizzato sia per
estrapolare in maniera automatica la semantica di un gadget, sia per
ottenere le condizioni di partenza che consentino ad un gadget di
eseguire una determinata funzione (ad esempio i valori di registro per
i quali un dato gadget scrivi in una specifica locazione di memoria un
dato valore).

Lo scopo di questo lavoro di tesi è lo sviluppo di uno strumento che
possa aiutare nel compito di generare una catena di gadget che sia in
grado di prendere controllo di un programma, premesse alcune
condizioni. In particolare lo strumento riesce a generare catene di
gadgets per eludere misure di sicurezza come stack non eseguibile,
casualizzazione dello spazio degli indirizzi e casualizzazione dello
stack applicando, in maniera automatica, tecniche come \emph{got
  patching} e \emph{return to plt} \cite{roglia:2009}. Lo strumento
utilizza per l'analisi del binario il recente framework di sviluppo
BARF \cite{Heitman-14} e supporta l'architettura intel a 32bit e a
64bit.

Descrizioni dei capitoli...

%% NX, ASRL, ROP

%% SMT, AEG, 

%% STRUMENTO

%%% Local Variables: 
%%% mode: latex
%%% TeX-master: "tesi"
%%% End: 
